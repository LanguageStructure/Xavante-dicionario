\chapter*{Como usar o dicionário}

Este dicionário apresenta uma seleção de palavras da língua Xavante com suas respectivas traduções e informações gramaticais em português. Cada entrada foi formatada para facilitar o acesso e a compreensão por parte de falantes, professores, estudantes e pesquisadores.

\section*{Formato das entradas}

As entradas seguem o seguinte modelo:

\begin{quote}
\textbf{wéré} \textit{s.} -- pássaro preto, anu preto
\end{quote}

\begin{itemize}
  \item \textbf{Forma em negrito}: representa a palavra em Xavante.
  \item \textit{Abreviação em itálico}: indica a classe gramatical (por exemplo, \textit{s.} para substantivo, \textit{v.} para verbo) -- ver abreviações.
  \item Após os dois traços (\texttt{--}), encontra-se a tradução ou explicação da palavra em português.
  \item Quando pertinente, formas variantes ou flexionadas são indicadas com exemplos adicionais em negrito.
  \item A tradução pode ser seguida de uma frase em Xavante, que por sua vez é seguida da tradução em português. Essa frase serve como exemplo do uso do verbete em questão.

  
\end{itemize}

\section*{Prefixo humano \textit{da-}}

Na língua Xavante, o prefixo \textit{da-} indica a categoria semântica de “ser humano” e pode aparecer como parte de muitos substantivos. Neste dicionário, entretanto, optou-se por registrar as palavras sem esse prefixo.

\begin{quote}
\textbf{da’wéra} “pessoa velha” aparece no dicionário como \textbf{’wéra}
\end{quote}

Portanto, ao procurar uma palavra que contenha o prefixo \textit{da-}, recomenda-se buscar a forma sem ele, pois o dicionário organiza as entradas de acordo com a raiz.

%#####################################################
\section*{Verbos com supleção de número}

No dicionário, pode-se observar que em alguns verbos número gramatical é especificado por uma supleção ou modificação da raiz. Em intransitivos essa mudança na raiz indica o número do sujeito (singular, dual, ou plural); em verbos transitivos a mudança diz respeito ao número do objeto. A tabela (\ref{tab:suppletiveIntrVerbs}) lista a supleção dos verbos intransitivos e a tabela (\ref{tab:suppletiveTransVerbs}) a supleção dos verbos transitivos.

\begin{table}
\begin{tabular}{llll}\toprule
Objeto singular & Objeto dual & Objeto plural & Significado  \\\midrule
\textit{damorĩ} & \textit{dane} & \textit{dasiʔabaʔre} & ir, caminhar \\
\textit{dawara} & \textit{dasisamro} & \textit{dasisaʔre} & correr \\
\textit{daza} & \textit{dasimaʔwara} & \textit{dasimasa} & estar em pé \\
\textit{danhamra} & \textit{dasimasisi} & \textit{dasimro, daʔubumro} & sentar-se, ficar\\
\textit{dazebre} & \textit{dazasi} & \textit{danhisisi} & entrar \\
\textit{dawatobro} & \textit{dapusi} & \textit{dawairebe} & sair, emergir\\
\textit{danomro} & \textit{dazaʔwari, daʔwa} & \textit{dabaʔwara, simiʔwara} & deitar\\
\textit{dawisi} & \textit{dasimasisi} & \textit{dasihutu} & chegar \\
\textit{dawaptãrã} & \textit{dawaptã} & \textit{derereʔe} & cair; nascer \\
\textit{daʔo} & \textit{dazaʔo} & \textit{danhimnhatã} & pendurar\\
\textit{dawawa} & \textit{dawawa} & \textit{daʔryry} & chorar \\\bottomrule
\end{tabular}
\caption{Verbos intransitivos com mudança na raiz}
\label{tab:suppletiveIntrVerbs}
\end{table}

\begin{table}
\begin{tabular}{llll}\toprule
 Sujeito singular & Sujeito Dual & Sujeito Plural & Significado  \\\midrule
\textit{öri} & \textit{mrami} & \textit{waibu} & pegar\\
\textit{hiri} & \textit{nomri} & \textit{saʔra} & colocar, deixar \\
\textit{me} & \textit{wabzuri} & \textit{sãmra} & jogar \\
\textit{sẽrẽ} & \textit{za} & \textit{sẽme} & inserir, colocar\\
\textit{tari} & \textit{rĩ} & \textit{sinarĩ} & colher \\
\textit{wĩrĩ} & \textit{pãrĩ} & \textit{simro} & matar \\ 
\textit{duri} & \textit{ʔwapé} & \textit{ʔwasari} & levar, carregar\\
\textit{ʔrẽne} & \textit{si} & \textit{huri} & comer \\
\textit{azöri} & \textit{ahöri} & \textit{hö zaʔra} & bater \\
\textit{wanherẽ} & \textit{ʔwaza} & \textit{suwa} & cozinhar \\
\textit{wẽʔẽ} & \textit{ẽ} & \textit{mamaʔẽ} & quebrar \\
\textit{hötari} & \textit{hösirĩ} & \textit{hösinari} & rasgar \\
\textit{wasi} & \textit{wasisi} & \textit{wasisi} & atar, amarrar \\
\textit{piʔra} & \textit{pizari} & \textit{pizari} & girar \\
\textit{waʔrã, döri} & \textit{waʔrãmi, höri} & \textit{waʔrãmi, höri} & responder \\
\textit{waʔré} & \textit{waʔré} & \textit{hözu} & machucar com um objeto \\
\textit{höʔrẽ} & \textit{hösi} & \textit{öhu} & beber \\\bottomrule
\end{tabular}
\caption{Verbos transitivos com mudança na raiz}
\label{tab:suppletiveTransVerbs}
\end{table}



%###################################################
\section*{Observações adicionais}

\begin{itemize}
  \item Algumas traduções apresentam mais de um termo em português para abranger diferentes usos ou significados da palavra Xavante.
  \item As formas com prefixos como \textbf{ĩ-} indicam flexões pessoais e são incluídas como exemplos quando relevantes.
  \item Entradas compostas ou com significados culturais importantes podem incluir observações adicionais em edições futuras.
\end{itemize}

Esperamos que este dicionário seja útil para a aprendizagem, o ensino e a valorização da língua Xavante.
