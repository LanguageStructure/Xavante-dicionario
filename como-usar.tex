\chapter*{Como usar o dicionário}

Este dicionário apresenta uma seleção de palavras da língua Xavante com suas respectivas traduções e informações gramaticais em português. Cada entrada foi formatada para facilitar o acesso e a compreensão por parte de falantes, professores, estudantes e pesquisadores.

\section*{Formato das entradas}

As entradas seguem o seguinte modelo:

\begin{quote}
\textbf{wéré} \textit{s.} -- pássaro preto, anu preto
\end{quote}

\begin{itemize}
  \item \textbf{Forma em negrito}: representa a palavra em Xavante.
  \item \textit{Abreviação em itálico}: indica a classe gramatical (por exemplo, \textit{s.} para substantivo, \textit{v.} para verbo).
  \item Após os dois traços (\texttt{--}), encontra-se a tradução ou explicação da palavra em português.
  \item Quando pertinente, formas variantes ou flexionadas são indicadas com exemplos adicionais em negrito.
\end{itemize}

\section*{Prefixo humano \textit{da-}}

Na língua Xavante, o prefixo \textit{da-} indica a categoria semântica de “ser humano” e pode aparecer como parte de muitos substantivos. Neste dicionário, entretanto, optou-se por registrar as palavras sem esse prefixo.

\begin{quote}
\textbf{da’wéra} “pessoa velha” aparece no dicionário como \textbf{’wéra}
\end{quote}

Portanto, ao procurar uma palavra que contenha o prefixo \textit{da-}, recomenda-se buscar a forma sem ele, pois o dicionário organiza as entradas de acordo com a raiz.


\section*{Observações adicionais}

\begin{itemize}
  \item Algumas traduções apresentam mais de um termo em português para abranger diferentes usos ou significados da palavra Xavante.
  \item As formas com prefixos como \textbf{ĩ-} indicam flexões pessoais e são incluídas como exemplos quando relevantes.
  \item Entradas compostas ou com significados culturais importantes podem incluir observações adicionais em edições futuras.
\end{itemize}

Esperamos que este dicionário seja útil para a aprendizagem, o ensino e a valorização da língua Xavante.