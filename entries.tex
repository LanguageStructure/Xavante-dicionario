\section*{A}


\textbf{a hã} \textit{pron. pessoa/enf.} -- tu, você.

\textbf{a na} \textit{v.} -- \textbf{wede a na} derrube a árvore.

\textbf{a'a wai'o} \textit{s.} -- cátarro, pigarra.

\textbf{a'a'a bö} \textit{s.} -- penas de rabo de mutum.

\textbf{a'a'a pré} \textit{s.} -- mutum vermelho.

\textbf{a'a'apré} \textit{s.} -- mutum.

\textbf{a'ama} \textit{s.} -- advogado, defensor.

\textbf{a'amo na mrozé} \textit{s.} -- elenco dos meses.

\textbf{a'amo za'ru} \textit{s.} -- halo da lua.

\textbf{a'amonamrozé} \textit{s.} -- calendário.

\textbf{a'amo} \textit{s.} -- lua.

\textbf{a'a} \textit{s.} -- mutum  "\textit{sada'ro}" antigo.

\textbf{a'a} \textit{s.} -- tosse.

\textbf{a'eta} \textit{s.} -- primavera.

\textbf{a'ẽta'a wi} \textit{s./posp.} -- início de primavera.

\textbf{a'ẽta'a} \textit{s.} -- \textbf{a'ẽta} primavera.

\textbf{a'odo} \textit{s.} -- coco, coco de bocaiúva.

\textbf{a'raté} \textit{s. ad.} -- mãe de primeiro filho.

\textbf{a'rezé} \textit{s.} -- lumbago.

\textbf{a're} \textit{s.} -- anca, lombo.

\textbf{a're} \textit{s.} -- semente, planta pequena.

\textbf{a're} \textit{v.} \textbf{ĩ're} = \textbf{rob're} -- plantar, semear.

\textbf{a'ri'ri} \textit{s.} -- bronquite.

\textbf{a'rã'ru bö} \textit{s.} -- rabo de mutum preto.

\textbf{a'rã'ru} \textit{s.} -- mutum.

\textbf{a'rãmi} \textit{v. du.} -- admirar, assustar-se, comover-se, impressionar-se.

\textbf{a'rã} \textit{v. sing.} = \textbf{a'rãmi} \textit{v. du.} -- assustar-se, impressionar-se, comover-se, vencer, dominar, admirar.

\textbf{a'ré} \textit{adv.} -- quase, já.

\textbf{a'u'ö} \textit{v.} -- secar, enxugar  \textbf{dasi'u'ö} enxugar-se.

\textbf{a'ubuni} \textit{s.} -- arbusto, matagal, mato denso, privada.

\textbf{a'urĩrĩ} \textit{v.} -- limpar, encerar, acercar, dar volta.

\textbf{a'uréiwi} \textit{adv.} -- nu, sem roupa, exposto, descoberto.

\textbf{a'uré} \textit{adj.} -- coberto, fechado.

\textbf{a'utepré} \textit{s.} -- bebê, bebé, nenê, nené, criança recém-nascida.

\textbf{a'uwẽ nhib'ri} \textit{s.} -- casa do índio, cabana.

\textbf{a'uwẽ nhibro} \textit{s.} -- lugar do índio, colônia indígena, terra de índios.

\textbf{a'uwẽ nhimirob'ru} \textit{s.} -- democracia.

\textbf{a'uwẽ pisudu} \textit{s. aa.} -- cabeça.

\textbf{a'uwẽ siwazari} \textit{s.} -- homem que se mistura, caboclo.

\textbf{a'uwẽ ubumro} \textit{s.} -- reunião dos homens.omunidade, povo.

\textbf{a'uwẽ wa're zapu} \textit{s. aa.} -- furo da orelha.

\textbf{a'uwẽ zéré} \textit{s.} -- cabelo de índio.

\textbf{a'uwẽ} \textit{s.} -- homem, pessoa, gente, cidadão, índio.

\textbf{a'ãma sipi'ö} \textit{s.} advogada, defensora.

\textbf{a'é} \textit{s.} -- semente de capim navalha, colar de semente de capim navalha, pedra.

\textbf{a'öiwãtizu} \textit{s.} -- incenso, resina.

\textbf{a'öpé} \textit{interj.} -- atenção! espere então!.

\textbf{a'öza} \textit{interj.} -- espera!, es bem!, faça isso! deixe assim, por enquanto.

\textbf{a'ö} \textit{interj.} -- espera!, ainda, per enquanto.

\textbf{a'ö} \textit{s.} -- jatobá.

\textbf{a'ö} \textit{v. sing.} = \textbf{a'öri} = \textbf{mrami} \textit{v. du.} -- carregar, pegar, tomar.

\textbf{a'ö} \textit{v.} -- \textbf{tãma a'ö mono} dar água.

\textbf{a'ö} \textit{v.} -- pegar água.

\textbf{a-} \textit{pref. pessoa 2ªpessoa} -- ana tua mae, sua mãe (de você).

\textbf{a-} \textit{pref.} -- indicando muitas coisas, muitas pessoas (\textit{ai-} -- \textit{ab-} -- \textit{am-}).

\textbf{ab're} \textit{s.} -- buraco, cova, cavidade.

\textbf{ab're} \textit{v.} -- cavar, fazer buraco, enterrar.

\textbf{ab'ruti'i} \textit{v.} -- enfurecer, zangar.

\textbf{ab'ru} \textit{adj.} -- zangado, irado \textbf{ab'ru ti} le está zangado, está furioso.

\textbf{ab'rã} \textit{s.} -- última casa da aldeia.

\textbf{ab'é'rã} \textit{s.} -- mussurana.

\textbf{aba'ẽ} \textit{v.} -- convencer  \textit{da'ãma} -- \textit{aba'ẽ} convencer alguém.

\textbf{aba'wa} \textit{s.} -- caçador.

\textbf{abada} \textit{s.} -- esposa, mulher casada sem filhos.

\textbf{abahi 'rãi pré} \textit{s.} -- fruta vermelha do mato.

\textbf{abahi 'ãiprére} \textit{s. dim} -- cereja.

\textbf{abahi wede ro} \textit{s.} lugar de árvore de fruta.

\textbf{abahiwedero} \textit{s.} -- pomar.

\textbf{abahi} \textit{s.} -- fruta do mato.

\textbf{abahu} \textit{s.} -- padrinho no "wa'i".

\textbf{abamere} \textit{s.} -- cestinho redondo.

\textbf{abame} \textit{s.} -- cesto redondo.

\textbf{abanhinheme ubumro} \textit{s.} -- bagagem.

\textbf{abanhinhemere} \textit{s.} -- cestinho com tampa.

\textbf{abanhizé} (\textbf{abazizé}) \textit{s.} --  baquité com tampa, mala.

\textbf{abare'u} \textit{s.} -- piqui,  nome de grupo etário.

\textbf{abare} \textit{s.} -- \textbf{ĩ'abare} bucho de ruminador.

\textbf{abare} \textit{s.} -- piqui.

\textbf{abarudu} \textit{s.} -- árvore de embira.

\textbf{abawazi} \textit{s.} -- timbó.

\textbf{abaze zahi} \textit{s.} -- animal ferroz, fera.

\textbf{abazei hö} \textit{s.} -- pele de animal de caça.

\textbf{abazeihö} \textit{s.} -- couro.

\textbf{abaze} \textit{s.} -- caça, animal de caça.

\textbf{abazi dumnarĩ} \textit{s.} -- algodão sem caroço.

\textbf{abazi zanhamri} \textit{s.} -- tecido tecido de algodão.

\textbf{abazinheme} \textit{s.} = \textbf{abazizé} -- cesto con tampa, baquité com tampa.

\textbf{abaziparazu} \textit{s.} -- rede.

\textbf{abazipara} \textit{s.} -- rede.

\textbf{abazizé} = \textbf{abanhizé} \textit{s.} -- cesto com tampa, baquité com tampa.

\textbf{abazi} \textit{s.} -- algodão, fio de algodão, linha.

\textbf{aba} \textit{s.} -- caça.

\textbf{abzaihö} \textit{v.} -- espirrar terra.

\textbf{abzuma} \textit{adv.} -- ao meio dia, de meio dia.

\textbf{abzuma} \textit{s.} -- meio dia.

\textbf{abzé} \textit{s.} -- veneno de feitiço, pó de casca para feitiço, doença, epidemia.

\textbf{abzö} \textit{s.} -- tipo de marimbondo.

\textbf{adabasa} \textit{s.} -- comida de jovem casada, casamento.

\textbf{adubdö} \textit{s.} -- pintura preta da barriga.

\textbf{adubzönhomri} \textit{s.} -- brincadeira peculiar.

\textbf{adumhöprã} \textit{s.} -- mulher recém-grávida.

\textbf{adu} \textit{s.} -- barriga, estômago, abdômen.

\textbf{adö'ö 'rai hi} \textit{s.} -- osso da cabeça do defunto.

\textbf{adö'ö'ru} \textit{s.} -- cemetério, sepultura.

\textbf{adö'ö'ru} \textit{s.} -- lugar dos falecidos.

\textbf{adö'ö'rãihi} \textit{s.} -- caveira, crãnio.

\textbf{adö} \textit{s.} -- defunto, falecido, cadáver, morte.

\textbf{ahu na} \textit{v./des.} -- pegue água.

\textbf{ahu'rã} \textit{s.} -- pintura de carvão.

\textbf{ahupré} \textit{s.} -- final de uma festa.

\textbf{ahömhö amoi wa} \textit{adv} -- ante ontem.

\textbf{ahömhö} \textit{adv.} -- ontem.

\textbf{ahö} \textit{adj.} -- \textbf{ĩ'ahö} bastante, muito, mais.

\textbf{ahö} \textit{v.} -- aumentar, aumentar gente, acrescentar.

\textbf{ahö} \textit{v.} = \textbf{ahöri} \textit{v. du./pl.} -- bater, surrar, castigar batendo.

\textbf{ai'a're} \textit{s.} -- lado do tórax de uma pessoa.

\textbf{ai'aba'ré} \textit{v. pl.} \textbf{dasi'aba'ré} = \textbf{dane} \textit{v. du.} -- ir, andar, caminhar s. aa.

\textbf{ai'are} \textit{ s. aa.} -- dente.

\textbf{ai'ratare} \textit{s. adol. dim.} -- velho, idoso.

\textbf{ai'raté} \textit{s.} -- mulher de um filho.

\textbf{ai'repudu} \textit{s.} -- menino que ainda não é "wapté".

\textbf{ai'rere} \textit{s.} -- arbusto de coco  grupo etário.

\textbf{ai'ro} \textit{s. aa.} -- fezes.

\textbf{ai'ré} \textit{v.} -- ir, afastar-se, separar, progredir, crescer, avançar.

\textbf{ai'ubuni} \textit{s.} -- mato, campo  \textbf{da'ubuni} virgem.

\textbf{ai'udu} \textit{v.} -- contestar, retrucar, mentir.

\textbf{ai'udö} \textit{v.} -- enganar, mentir.

\textbf{ai'uirĩ} \textit{v.} -- rodar.

\textbf{ai'uiréiwa} \textit{adv.} -- abertamente, claramente.

\textbf{ai'uiréiwi} \textit{adv.} -- abertamente, claramente.

\textbf{ai'uiré} \textit{adj.} -- claro, aberto.

\textbf{ai'uté mreme tétére} \textit{s./v./s.} -- balbuciar.

\textbf{ai'uté nhanapré} \textit{s.} -- verme de criança.

\textbf{ai'uté podo} \textit{s.} -- convívio matrimonial, coito, relacionamento sexual.

\textbf{ai'uté potore} \textit{s.} -- boneca.

\textbf{ai'uté wa} \textit{s.} -- vencedor, campeão = \textbf{da'ãma ai'uté wa}.

\textbf{ai'uté za'warizé} \textit{s.} -- berço.

\textbf{ai'uté} \textit{s.} -- criança.

\textbf{ai'uté} \textit{v.} -- alcançar, conseguir, vencer  \textbf{wa ãma ai'uté} nós o alcançamos, nós o conseguimos, nós o vencemos.

\textbf{ai'utörĩ} \textit{v.} -- terminar, definhar, morrer, acabar.

\textbf{ai'utö} \textit{v. sing.} -- \textbf{ai'utörĩ} -- terminar, definhar, morrer, acabar.

\textbf{ai'uza} \textit{s. aa.} -- não índio, "civilizado".

\textbf{ai'wapé} \textit{v. du.} -- combater, lutar, guerrear, concorrer.

\textbf{ai-} = \textbf{a-} -- perfixo de 2ª pessoa.

\textbf{ai-} \textit{pref.} -- indicando muitas coisas, muitas pessoas  = \textbf{a-} = \textbf{ab-} = \textbf{am-}.

\textbf{aiba} \textit{v.} -- -- compadecer-se, ter amor, ter compaixão, ter pena, ter paciência.

\textbf{aibö nhimihö ze} \textit{s. adj.} -- homem que gosta de briga, homem briguento.

\textbf{aibö toire} \textit{s.} -- homem que gosta de brincadeira.

\textbf{aibönhimihöze} \textit{s.} -- brigão.

\textbf{aibötoire} \textit{s.} -- palhaço, brincalhão.

\textbf{aibö} \textit{s.} -- homem, macho.

\textbf{aihini} \textit{pron. ind} -- todos, todos juntos.

\textbf{aihutu} \textit{v. pl.} -- chegar, alcançar.

\textbf{aihu} \textit{s. aa.} -- pauzinho da orelha, brinco.

\textbf{aihö hö} \textit{s.} -- couro de veado.

\textbf{aihö nhimi'wa} \textit{s.} -- unha de veado.

\textbf{aihö'ubuni} \textit{s.} -- \textit{wapté} de orelha furada, \textit{wapté} que é chefe do grupo.

\textbf{aihöi'ré} \textit{s.} -- jacaré.

\textbf{aihö} \textit{s.} -- campeiro, veado, cervo.

\textbf{aihö} \textit{v.} -- rir, sorrir.

\textbf{aima'uri} \textit{v.} -- esconder, esconder-se.

\textbf{aima'wara} \textit{v. du.} -- fazer subir, empoleirar, ficar de pé.

\textbf{aima'wa} \textit{v. sing./du.} = \textbf{aima'wara} = \textbf{dasima'wara} -- fazer subir, empoleirar, ficar de pé.

\textbf{aimani} \textit{v.} -- fugir, perder-se.

\textbf{aimasa} \textit{v. pl.} \textbf{dasimasa} = \textbf{dasima'wara} \textit{v. du.} -- ficar de pé, encontrar-se.

\textbf{aimasisi} \textit{v. du.} -- estar, ficar, chegar, ser, estar, comparecer, apresentar-se.

\textbf{aimawi a'uwẽ} \textit{adj./s.} -- índio não xavante.

\textbf{aimawi} \textit{adj.} -- diferente, outro.

\textbf{aimawi} \textit{s.} -- cláusula.

\textbf{aime} \textit{v. sing.} \textbf{dasime} = \textbf{dasiwabzuri} \textit{v. du.} -- deitar-se, jogar-se, meter-se  \textbf{da'ãma dasiwãri} \textit{v.} ajudar  \textbf{aime na} ajude-me!.

\textbf{aiparapisudu} \textit{s. aa. comp.} -- cavalo.

\textbf{aipi'ra} \textit{v. sing.} \textbf{dasipi'ra} = \textbf{dasipizari} \textit{v. du./pl.} -- virar-se, voltar-se.

\textbf{aipo'o} \textit{v.} = \textbf{po'o} -- quebrar, partir, rachar, romper-se.

\textbf{aipuru} \textit{v.} -- \textbf{ĩpru} quebrar-se, romper.

\textbf{aipé} \textit{v.} -- dividir, cortar em pedaços, espalhar, dissipar.

\textbf{aire} \textit{v. dim.} \textbf{aire na} = \textbf{a na} -- dê! entregue!.

\textbf{airãzapotore} \textit{s. aa.} -- macaco.

\textbf{aiwa ĩnomri} \textit{s.} -- compensação.

\textbf{aiwa'aba} -- no meio de dois.

\textbf{aiwa'aba} \textit{v.} -- condenar, executar.

\textbf{aiwa'ro} \textit{s. aa.} -- sol.

\textbf{aiwa'u} \textit{s. aa.} -- água.

\textbf{aiwa'öno} \textit{s.} -- choro.

\textbf{aiwa'öno} \textit{s.} -- separação.

\textbf{aiwa'öno} \textit{v} -- chorar.

\textbf{aiwa'ötöre} \textit{s. aa.} -- olho.

\textbf{aiwa'ö} \textit{s.} -- choro.

\textbf{aiwa'ö} \textit{v.} -- chorar.

\textbf{aiwab zö ĩpibu} comparação.

\textbf{aiwahi'a} \textit{s.} -- \textbf{aiwahi'ãsi} \textit{s.} ceremöia para curar doença.

\textbf{aiwahudu} \textit{v.} -- levantar-se, erguer-se.

\textbf{aiwahutu} \textit{v} -- levantar-se, erguer-se.

\textbf{aiwamri} \textit{v.} = \textbf{dawamri} -- sossegar, acalmar-se, ter calma.

\textbf{aiwanarĩwarĩrĩzé} \textit{s. aa.} -- sal.

\textbf{aiwanarĩzeire} \textit{s. aa. dim.} -- biscoito, bolacha.

\textbf{aiwanhizari} \textit{v.} -- -- separar, afastar, apartar.

\textbf{aiwanhiza} \textit{v. sing.} = \textbf{aiwanhi zari} -- separar, afastar, apartar.

\textbf{aiwapsari} \textit{v.} -- resmungar, murmurar, criticar, queixar-se.

\textbf{aiwapsisi} \textit{v. du./pl.} -- dançar.

\textbf{aiwapsi} = \textbf{aiwa} \textit{adv.} -- igualmente, semelhante, assim.

\textbf{aiwazure} \textit{s. aa.} -- cabelo.

\textbf{aiwa} = \textbf{aiwapsi} \textit{adv.} -- igual, semelhante, assim.

\textbf{aiwẽ'ẽ} \textit{v.} -- inclinar-se, sair às pressas, disparar em corrida.

\textbf{ai} \textit{v} = \textbf{a}  \textbf{ĩ'ai mo} -- dar comida, alimentar  \textbf{a na} dê-me.

\textbf{aj'are} \textit{s} -- {dente}

\textbf{aj'are} \textit{s} -- {dente}

\textbf{aj'rowi} \textit{s} -- {capivara}

\textbf{aj'rãjhõrõ} \textit{s} -- {urubu}

\textbf{aj'rãzapotore} \textit{s} -- {macaco}

\textbf{aj'uza} \textit{s} -- {homem branco}

\textbf{aj'wapesere} \textit{s} -- {rato}

\textbf{aj'waza'é} \textit{s} -- {cachorro}

\textbf{aj'wañirã} \textit{s} -- {tamanduá-bandeira}

\textbf{aj'ãpu} \textit{v} -- {falar}

\textbf{aj'ãpu} \textit{v} -- {falar}

\textbf{ajböjhöbö} \textit{s} -- {quati}

\textbf{ajparapisudu} \textit{s} -- {cavalo}

\textbf{ajparapo} \textit{s} -- {pato}

\textbf{ajpesere} \textit{s} -- {queixada}

\textbf{ajpe} \textit{s} -- {porco-do-mato}

\textbf{ajpi} \textit{s} -- {chuva}

\textbf{ajwa'õtõre} \textit{s} -- {olho}

\textbf{ame!} \textit{excl. infa}. mãe!.

\textbf{amere} \textit{excl. infa. dim.} -- maezinha!.

\textbf{amhuri} \textit{s.} -- lepra.

\textbf{amhözépa sina marĩ 'ru} \textit{s./posp./pron./s.} chantagem.

\textbf{amhözépa} \textit{s.} -- ameaça, tribulação.

\textbf{amiro} \textit{s.} -- centopéia.

\textbf{amiruru'wa} \textit{s.} -- lagarta de fogo.

\textbf{amnhana} \textit{v. pl} \textbf{danhimnhana} = \textbf{daza'o} \textit{v. du.} -- pender, pendurar.

\textbf{amnhasi} \textit{v.} -- pedir a quem tem, desconfiar.

\textbf{amnhatã} \textit{v. pl. c.} \textbf{danhimnhana} = \textbf{daza'o} \textit{v. du.} -- pender, pendurar.

\textbf{amnho za'warizé} \textit{s.} -- armazém.

\textbf{amnho'a} \textit{s.} -- trigo.

\textbf{amnho'rã} \textit{s.} -- centeio.

\textbf{amnho'rézé} \textit{s.} -- secador de cereais.

\textbf{amnho'ubumrozé} \textit{s.} -- lugar de reunir os cereais, silo, celeiro.

\textbf{amnhonhimnhorö} \textit{s.} -- cevada.

\textbf{amnhotehöri} \textit{vmp.} -- ceifar, colher.

\textbf{amnhotepezé} \textit{s.} -- março.

\textbf{amnho} \textit{s.} -- cereal.

\textbf{amo'wa} \textit{s.} -- outro, companheiro falecido.

\textbf{amo} \textit{adj.} -- \textbf{ĩ'amo} outro.

\textbf{amo} \textit{s.} -- companheiro, colega, camarada.

\textbf{anarowa} \textit{s.} -- grupo etário.

\textbf{ana} \textit{prep. pessoa/s.} -- tua mãe, sua mãe (de você).

\textbf{ana} \textit{v.} \textbf{da'ãma ana} = \textbf{da'ãma dasina} -- ajudar, servir.

\textbf{ane} \textit{s. aa.} -- veado, cervo.

\textbf{anhana'rãtömri} \textit{s.} -- lagarta preta.

\textbf{anhana'upi} \textit{s.} -- besouro.

\textbf{anhana} \textit{s.} -- esterco, estrume.

\textbf{anhidazari} \textit{v.} -- matar, acabar com.

\textbf{anhisisi} \textit{v.} -- afastar, livrar.

\textbf{anhiwasi} \textit{v.} -- afastar, livrar, perdoar, abrir.

\textbf{anhiza} \textit{v.} -- chegar ao grupo, alcançar alguém.

\textbf{anhomri} \textit{v.} -- \textbf{sömri} engolir, dar muito.

\textbf{anhoré} = \textbf{danhoré} \textit{v.} -- enfileirar  \textbf{romnhoré} \textit{v.} estudar.

\textbf{anhĩsu} = \textbf{sisu} -- colocar penas em flecha.

\textbf{anomri} \textit{v. aa.} -- ir.

\textbf{apa} \textit{s.} -- lagartixa.

\textbf{api'ra} \textit{v. sing.} = \textbf{dasipiza} \textit{v. du.} -- tombar, virar-se.

\textbf{api} \textit{v.} -- cozinhar.

\textbf{apsaihuri} \textit{v.} -- roubar.

\textbf{apsi nhoro} \textit{s.} -- fibra de broto de gravatá.

\textbf{apsi} \textit{s.} -- gravatá, abacaxi.

\textbf{apto'oré} \textit{adv.} -- mais tarde.

\textbf{aptorore} \textit{s.} -- nhambu.

\textbf{aptoro} \textit{v.} -- fazer flechas para caça.

\textbf{aptö'özé} \textit{s.} -- cansaço, sono.

\textbf{aptö'ö} \textit{adj. c.} -- \textbf{aptö} cansado, com sono, exausto.

\textbf{aptömri'a} \textit{s.} -- cera branca.

\textbf{aptömriro'o zabzé} \textit{s.} -- lugar de pör a luz de cera, castiçal.

\textbf{aptömriro} \textit{s.} -- luz de cera, vela.

\textbf{aptömri} \textit{s.} -- cera preta.

\textbf{aptö} \textit{s.} -- sono, cansaço.

\textbf{apö sihöiba} \textit{adv./v.} -- \textbf{apö dasihöiba} ressuscitar, voltar à vida.

\textbf{apöre} \textit{adv. dim.} -- outra vez.

\textbf{apösi} \textit{adv.} -- depois, mais tarde, só na volta.

\textbf{apö} \textit{adv.} -- outra vez, de volta, depois.

\textbf{ari'iwi} \textit{adv.} -- quieto, silencioso.

\textbf{arãrã nhi'u} \textit{s.} -- suco que o beija-flor tira, néctar, suco de flor.

\textbf{arãrã'rã} \textit{s.} -- beija-flor preto.

\textbf{arãrãre} \textit{s.} -- colibri.

\textbf{arãrã} \textit{s.} -- beija-flor.

\textbf{asa'ẽtẽ nhi} \textit{s. aa.} -- carne de anta.

\textbf{asabutéi'wa} \textit{s. aa.} -- moço iniciado.

\textbf{asabu} \textit{s. aa.} -- casa.

\textbf{asada'a tei hi} \textit{s./s.} -- osso de canela de onça parda.

\textbf{asada'a} \textit{s.} -- \textbf{asada} puma, onça parda.

\textbf{asada'rã} \textit{s.} -- lobo guará.

\textbf{asada} \textit{s.} -- puma, onça parda.

\textbf{asadö} \textit{s.} -- camaleão, tiú.

\textbf{asahöpöre} \textit{s. aa. dim.} -- unha.

\textbf{asamrohöbö} \textit{s.} -- ouriço chato.

\textbf{asamro} \textit{s.} -- ouriço, ouriço cacheiro.

\textbf{asamro} \textit{v. du.} -- correr.

\textbf{asanaze} \textit{s.} -- preã.

\textbf{asarob'rãihörizé} \textit{s.} -- cois.om que s.orta a cacho do cereal, colheitadeira.

\textbf{asaro} \textit{s.} -- arroz.

\textbf{asarébé'a} \textit{s. aa.} -- tipo de peixe.

\textbf{asaröno} \textit{v.} -- pular muito.

\textbf{asarötöpe} \textit{s. aa.} -- bola.

\textbf{asimro} \textit{v. pl.} \textbf{da'ubumro} = \textbf{dasimasisi} \textit{v. du.} -- unir-se, ficar, sentar-se.

\textbf{asisi} \textit{v. pl.} \textbf{danhisisi} = \textbf{dazsi} \textit{v. du.} -- entrar, seguir.

\textbf{asã'ẽne} \textit{s. aa.} -- anta.

\textbf{asérére umo} \textit{s.} -- pauzinho de buritirana.

\textbf{asérére} \textit{s.} -- buritirana.

\textbf{ati} \textit{v.} \textbf{ĩti} = \textbf{dati} -- escolher.

\textbf{ato} \textit{s.} -- corcunda.

\textbf{atébré } \textit{v.} -- proliferar, ter filhos.

\textbf{atébré} \textit{s.} -- orvalho.

\textbf{atéma} \textit{adv.} -- devagar.

\textbf{awa'öbö} \textit{v.} -- pagar, recompensar.

\textbf{awaibu} \textit{v. pl.} = \textbf{mrami} \textit{v. du.} -- pegar muito.

\textbf{awaru 'mapraba'wa ubumro} \textit{s./s.} -- cavalaria.

\textbf{awaru 'mapraba'wa} \textit{s.} -- cavaleiro.

\textbf{awaru 'mapraba} \textit{s./v.} -- cavalgar, montar cavalo.

\textbf{awaru ba'ömore} \textit{s.} -- jumentinho.

\textbf{awaru butu zéré} \textit{s./s.} -- cabelo do pescoço do cavalo, crina.

\textbf{awaru po're pa} \textit{s.} -- cavalo de orelha comprida.

\textbf{awarubati} \textit{s.} -- camelo.

\textbf{awarupo'repa} \textit{s.} -- burro.

\textbf{awarupo'repore} \textit{s.} -- jumentinho.

\textbf{awarupo'repo} \textit{s.} -- jumento, burro.

\textbf{awaru} \textit{s. neo.} -- cavalo.

\textbf{awatis} \textit{v.} -- amarrar.

\textbf{awati} \textit{v.} -- prensar, endireitar, subjugar, vencer.

\textbf{awa} \textit{s.} -- nome de árvore.

\textbf{awa} \textit{s.} -- sombra, assombração.

\textbf{awẽ} \textit{adv.} -- amanhã.

\textbf{awẽ} \textit{v.} -- amanhecer  \textbf{ma tö ti'awẽ} amanheceu.

\textbf{awã'awi} \textit{adv.} -- já, logo, agora.

\textbf{awãrã} \textit{s.} -- mato, nome de árvore, arbusto reto.

\textbf{awã} \textit{s.} -- arbusto que faz sombra.

\textbf{awẽmhã} \textit{adv} -- {amanhã}

\textbf{aza'ra} \textit{v.} -- \textbf{ĩsa'ra} amontoar, arrumar, deixar, ficar, colocar, botar, pôr, agir.

\textbf{aza'ré} \textit{v.} -- \textbf{ĩsa're} dar prazo, dar tempo.

\textbf{aza'ré} \textit{v.} -- entregar, trair, ceder, afaster.

\textbf{azahu pisudu} \textit{s.} -- penacho de arara com pouca pena.

\textbf{azahu} \textit{s.} -- penacho de arara e outros.

\textbf{azani} \textit{v.} -- \textbf{ĩsani} tirar, afastar, livrar, afugentar.

\textbf{azarudu} \textit{s.} -- moça.

\textbf{azarutu} \textit{s.} -- \textbf{azarudu} moça.

\textbf{aze} \textit{s.} -- açúcar.

\textbf{azidi} \textit{v.} -- fiar.

\textbf{azo} \textit{s. neo} -- anzol.

\textbf{azöri} \textit{v. sing.} -- \textbf{ahöri} \textit{v. du./pl.} bater, surrar, castigar batendo.

\textbf{a} \textit{adj.} -- branco.  \textit{wede a} madeira branca.

\textbf{a} \textit{pron.} -- marca de 2 pessoa.

\textbf{a} \textit{s.} -- tosse.

\textbf{a} \textit{v.} -- \textit{da'a} tossir, pingar, cair pingando  \textit{ö ha, te ti'a'a} ele tosse, ele pinga.

\textbf{a} \textit{v.} -- dar  \textit{ö hã, ma tãma ti'a} ele lhe deu  \textit{ima a na} dê-me, ajude-me.

\textbf{ãhã na} \textit{pron. dem./posp.} -- por aqui, seguindo por aqui.

\textbf{ãhã ta, wa ĩhöiba} -- eu estou aqui, eu estou presente, presente!.

\textbf{ãhã ta} -- aqui! está aqui!.

\textbf{ãhãna} \textit{adv.} -- hoje, agora.

\textbf{ãhã} \textit{pron. dem.} -- este, esta.

\textbf{ãma ĩmro} \textit{v.} -- calcular.

\textbf{ãma} \textit{interj.} -- afasta um pouco, aproximar de!.

\textbf{ãma} \textit{posp.} -- de, em, a (ao, à), afastado de, para.

\textbf{ãme} \textit{adv.} -- aqui, cá.

\textbf{ãna} \textit{posp.} -- sem.

\textbf{ãne u'ö} \textit{adv./adj.} -- sempre assim, constância.

\textbf{ãne wa} -- por isso.

\textbf{ãnere} \textit{conj.} -- então.

\textbf{ãne} \textit{adv.} -- assim (aqui).

\textbf{ãpu} \textit{v.} -- cansar-se  \textbf{ãma ĩ'ãpu}.

\textbf{ãrewa} \textit{s.} -- cunhado que é irmão da esposa.

\textbf{ãté} \textit{adv.} -- talvez.

\textbf{ãwa} \textit{adv.} -- aqui, este lugar aqui, cá, para cá.

\textbf{ãwẽ amo na} \textit{adv.} -- depois de amanhã.

\textbf{ãwẽ u} -- de madrugada.

\textbf{ãwẽ wi} -- de madrugada.

\textbf{ãwẽ'ö} \textit{v.} -- recusar, renunciar, não gostar, desprezar.

\textbf{ãwẽpsi} \textit{adv.} -- amanhã somente.

\textbf{ãwẽ} \textit{v.} -- anunciar, falar, apresentar por palavras.

\textbf{ãwisi} \textit{v.} -- trazer, comunicar.

\textbf{ãwi} \textit{v. sing.} = \textbf{ãwisi} -- trazer, comunicar.

\textbf{ãzé} \textit{v. sing.} = \textbf{dazasi} \textit{s. du.} -- entrar  = \textbf{'mazasi} introduzir  \textbf{wa ãma ãzé} eu entro para junto de você.

\textbf{ã} \textit{interj.} -- eis! eis aí! tome!





%#############################################################
\section*{B}


\textbf{ba} \textit{s.} -- costas da gente.

\textbf{ba} \textit{posp.} -- para, rumo a  \textbf{öi ba} para o rio, ao rio.

\textbf{ba} \textit{v.} -- \textbf{ĩba} repartir, tirar parte, cortar, diminuir.

\textbf{ba} \textit{s.} -- \textbf{ĩba} parte, corte.

\textbf{ba'a} \textit{v} -- \textbf{ĩba} repartir, tirar parte, cortar, diminuir.

\textbf{ba'are} \textit{s.} -- \textbf{ĩba'are} partezinha  \textbf{dapara'uza ba'are} chinelo.

\textbf{baba} \textit{posp.} -- por, em  \textbf{'rinho'rei baba} pelas casas, pela fileira das casas, nas casas.

\textbf{babarã} \textit{posp.} -- \textbf{ĩbabarã} atrás de.

\textbf{babati} \textit{s.} -- caçula masculino.

\textbf{bazadö} \textit{s.} -- que fica nas costas do animal, arreio.

\textbf{bazadö} \textit{vmp.} -- arrear.

\textbf{bazadözé} \textit{s.} -- \textbf{ĩbazadözé} cobertura das costas do animal.

\textbf{baihö} \textit{s.} -- \textbf{ĩbahbö} rolo de papel, escrita, escritura, tabuleta.

\textbf{baihö} \textit{adj.} -- \textbf{ĩbahbö} raso, fino, brando.

\textbf{ba'ömo} \textit{s.} = \textbf{ba'u} -- as costas que tem chifre  \textbf{awaru ba'ömo} \textit{s.} jumento.

\textbf{ba'ömore} \textit{s.} -- as costas que tem chifre  \textbf{awaru ba'ömore} \textit{s.} jumentinho.

\textbf{ba'öno} \textit{s.} -- menina, mocinha.

\textbf{ba'ötö} \textit{s.} -- \textbf{ba'öno} menina, mocinha.

\textbf{bara} \textit{s.} -- barata.

\textbf{barahö} \textit{s.} -- cicatriz.

\textbf{baraire} \textit{s.} -- baratinha.

\textbf{barana} \textit{adv.} -- de noite.

\textbf{barana dasa} \textit{adv./s.} -- ceia.

\textbf{bati} \textit{s.} -- corcunda.

\textbf{ba'u} \textit{s.} -- \textbf{ĩba'u} que tem costas com chifre.

\textbf{batiwaru} \textit{s.} -- camelo.

\textbf{bato} \textit{s.} -- \textbf{ĩbato} corcunda.

\textbf{bété} \textit{intj.} -- então, mesmo, será!.

\textbf{biriwawẽ} \textit{s.} -- bigorna, bilhão.

\textbf{bö} \textit{adj.} -- \textbf{ĩbö} certo, apto, exato, coerente, assim.

\textbf{bö} \textit{posp.} -- todo, repetidamente  \textbf{bötö bö} todos os dias, diariamente  \textbf{da bö} então.

\textbf{bö} \textit{posp.} por acaso  \textbf{e niha bö} por acaso? por que?.

\textbf{bö} \textit{s.} -- \textbf{ĩbö} rabo, cauda.

\textbf{bö} \textit{s.} -- pênis.

\textbf{bö} \textit{s.} -- urucum.

\textbf{bödi} \textit{s. ad.} -- filho, neto.

\textbf{bödö} \textit{s.} -- sol, dia hora, hora de relógio.

\textbf{bödödi} \textit{s.} -- caminho, estrada.

\textbf{bödödi nhitobzé} \textit{s.} -- coisa que fecha o caminho, porteira, cancela.

\textbf{bödödi nhitbozé ĩsi'uwazi na} \textit{s./s./posp.} -- coisa que fecha o caminho com arame, colchete.

\textbf{bödödinho'u} \textit{s.} -- avenida, rodovia.

\textbf{bödödĩ'rã} \textit{s.} -- asfalto.

\textbf{bödödire} \textit{s.} -- trilho.

\textbf{bödödisipa} \textit{s.} -- bifurcação.

\textbf{bödödi su'u'wa} \textit{s.} -- alisador de estrada, patrola.

\textbf{bödö ma dörö} \textit{s./pron./v.} -- eclipse solar.

\textbf{bödöra} \textit{s.} -- seio endurecido, abcesso mamário.

\textbf{bö ĩza} \textit{adj./v. du.} -- \textbf{ĩbö ĩza} bem ajustado, caber.

\textbf{böi hö} \textit{s.} -- casca de urucum.

\textbf{böi waihi} \textit{s.} -- galho de urucum.

\textbf{böre} \textit{adj.} -- \textbf{ĩböre} conveniente, certo, apto, exato, certinho, coerente  \textbf{tô ĩböre} conformar  \textbf{éré ĩböre} convir, convém, quase.

\textbf{bötö} \textit{s.} -- \textbf{böbö} sol, dia, hora.

\textbf{bötödö} \textit{s.} -- eclipse solar.

\textbf{bötöza'ẽtẽwẽ!} -- boas festas!.

\textbf{bötö za'ru} \textit{s.} -- phalo do sol.

\textbf{bötö zasizé} \textit{s.} -- pôr do sol, oeste, ocidente.

\textbf{bötö nhimi'ahöri} \textit{s.} -- insolação.

\textbf{bötö pusizé} \textit{s.} -- nascer do sol, leste, oriente.

\textbf{bötö warö} \textit{s.} -- eclipse solar.

\textbf{bötöwawi} \textit{s.} -- minuto.

\textbf{bötöwawire} \textit{s. dim} -- segundo.

\textbf{bösi} \textit{s.} -- crítica, calúnia, potoca, mentira.

\textbf{bösi} \textit{v.} -- criticar, caluniar, falar mal.

\textbf{brudu} \textit{s.} -- borduna de aroeira.

\textbf{brutu} \textit{s.} -- \textbf{brudu} borduna de aroeira.

\textbf{brutu'amo} \textit{s.} -- outra aroeira, carvalho.

\textbf{bu} \textit{s.} -- \textbf{ĩbu} taquara oca, cano.

\textbf{budu} \textit{s.} -- pescoço.

\textbf{buze} \textit{s.} -- cana, caniço, cana de açúcar.

\textbf{bu're} \textit{s.} -- \textbf{ĩbu're} cano.

\textbf{buru} \textit{s.} -- roça.

\textbf{buro a'ötöre} \textit{s.} -- flecha sem ponta.

\textbf{buru'öno} \textit{s.} -- andorinha.

\textbf{buru'ötöre} \textit{s.} -- andorinha, andorinha pequena.

\textbf{buru'ötöwawẽ} \textit{s. aum.} -- andorinha grande.

\textbf{buruteihi} \textit{s.} -- cana, erva especial, capim.

\textbf{butömo} \textit{s.} -- botão.

\textbf{butömore} \textit{s.} -- vegalume.

\textbf{butu} \textit{s.} -- pescoço.

\textbf{butudazpu} \textit{s.} -- torcicolo.

\textbf{butuzé} \textit{s.} -- dor de pescoço.

\textbf{butu nihöri} \textit{s./v.} -- cortar pescoço, degolar.

\textbf{butupawaru} \textit{s.} -- girafa.

\textbf{butupo} \textit{s.} -- gravata do "wai'a".

\textbf{butu'ré} \textit{adj.} -- \textbf{ĩbutu'ré} estarrecido, pescoço longo.

\textbf{butu'ré} \textit{vmp.} -- fazer pescoço comprido, por a cabeça para fora, espiar.

\textbf{butu'ré} \textit{s.} -- pescoço duro, torcicolo.


%#####################################################
\textbf{D}.



\textbf{da} \textit{posp.} -- para.

\textbf{da} \textit{conj.} -- para, para que, afim de que.

\textbf{da-} \textit{pref. pessoa} -- da pessoa  da gente (sentido genérico).

\textbf{da'ãmawai'a} \textit{s.} -- iniciação ao "wai'a".

\textbf{da'ãmawai'a'wa} \textit{s.} -- iniciador ao "wai'a".

\textbf{dabasa} \textit{s.} = \textbf{adabasa} -- caça de casamento.

\textbf{daba'wara} \textit{v. pl.} = \textbf{daza'wari} \textit{v. du.} -- deitar-se.

\textbf{da bö} -- então, em seguida.

\textbf{dahu} \textit{adj.} -- amontoado.

\textbf{dahu nho'u} \textit{s.} -- muita gente, povoado.

\textbf{danhimbö} \textit{s.} -- caminho, trilho.

\textbf{danhimite} \textit{s.} -- bom espírito que é portador de vida, mensageiro, anjo.

\textbf{danhipai'wa} \textit{s.} -- chefe, senhor, Deus, superior.

\textbf{danhipai'wa nhimbödö} \textit{s.} -- dia do Senhor, domingo.

\textbf{danho'rehipãrĩro} \textit{s.} -- lugar de sangrar pescoço de gente  Aldeia Sangradouro.

\textbf{da'o} \textit{v. sing.} -- \textbf{sa'o} \textit{v. du.} -- pender, suspender, suspender a tampa, elevar.

\textbf{darẽ} \textit{v.} -- ganhar jogo  \textbf{rẽme} abandonar, largar, deixar.

\textbf{daro mono bö} \textit{s./adv./posp.} -- em todo o lugar.

\textbf{dati'ö} \textit{s. ad.} -- mãe, tia (ao chamar).

\textbf{dati'ö'u} \textit{s.} -- jaó.

\textbf{dasina damrozé} \textit{s.} -- casamento.

\textbf{da'u} \textit{s.} -- piolho.

\textbf{da'wa} \textit{v. du.} = \textbf{daza'wari} -- deitar-se.

\textbf{dasiré romhuri} -- \textit{v.} cooperar.

\textbf{dawa} \textit{s.} -- abertura, entrada  \textbf{ö dawa} beira do mar, praia  \textbf{'ri dawa} abertura da casa, janela da casa, entrada da casa  \textbf{'ridawa} porta  \textbf{dazadawa} boca.

\textbf{di} \textit{s.} -- barriga, ventre.

\textbf{di} \textit{c.} = \textbf{ti} -- ser, estar, ter, haver  \textbf{wẽ di} é bom.

\textbf{di} \textit{v.} = \textbf{robdi} -- molhar, aguar, borrifar.

\textbf{di'i} \textit{v} = \textbf{robdi} -- molhar, aguar, borrifar.

\textbf{di'i} \textit{s.} = \textbf{robdi} -- humidade.

\textbf{di'i} \textit{s.} -- útero, ventre, entranhas, seio (palavra genérica para não usar nome próprio).

\textbf{di'izé} \textit{s.} -- dor de barriga  \textbf{di'ize pire} \textit{s.} \textbf{dadi'izé pire} cólica.

\textbf{di'iwai'u} \textit{s.} -- vermes intestinais.

\textbf{di'iwa'u} \textit{s.} -- diarréia, disenteria.

\textbf{dö}  \textbf{udazö} \textit{s.} -- calça  \textbf{bazadö} \textit{s.} -- cobertura das costas de animal.

\textbf{dö} \textit{s.} -- morte.

\textbf{dö'ö} \textit{v. du./pl.} -- \textit{v. du.} -- morrer.

\textbf{dö'özé} \textit{s.} -- morte.

\textbf{dö'ö'õ} \textit{adj.} -- \textbf{ĩdö'öõ} imortal.

\textbf{dö'ösina} \textit{conj.} -- justamente por, por causa de.

\textbf{dörö} \textit{v. sing.} -- v. du. -- morrer.

\textbf{du} \textit{s.} -- capim, grama, erva, fogo de caça.

\textbf{du} \textit{s.} -- barriga, estômago, abdômen.

\textbf{du} \textit{s.} -- \textbf{ĩdu} bolsa, saco.

\textbf{du'a} \textit{s.} -- capim gordura.

\textbf{dub'rada} \textit{s.} -- irmão mais velho.

\textbf{dub'rata} \textit{s.} -- irmão mais velho.

\textbf{duzapari'wa} \textit{s.} -- caçador que fica na espera.

\textbf{du zõ} \textit{s./posp.} -- "para o capim" = para a caça, para a queimada.

\textbf{dumnarĩ} \textit{adj.} -- sem caroço  \textbf{abazi dumnarĩ} algodão sem caroço.

\textbf{du nho'u} \textit{s.} -- chá.

\textbf{duptede} \textit{s.} -- prisão de ventre.

\textbf{dupto} \textit{s.} -- inchaço, inflamação  \textbf{da'u'aze dawi ĩdupto} contusão.

\textbf{dupto} \textit{v.} -- inchar.

\textbf{dupu} \textit{s.} -- inchaço, gordura, flato.

\textbf{dupu} \textit{adj.} -- \textbf{ĩdupu} gordo, inchado, corpulento.

\textbf{dupurewawẽ} \textit{s.} -- \textbf{ĩdupurewawẽ} balão.

\textbf{duré} \textit{conj.} -- e, mais.

\textbf{duréi hã} \textit{adv.} -- antigamente.

\textbf{duréi hé} \textit{adv.} -- hoje.

\textbf{duréi pese} \textit{adv./adv.} -- muito tempo atrás.

\textbf{duri} \textit{v. sing.} -- \textbf{'wapé} \textit{v. du.} carregar, levar.

\textbf{dusu} \textit{s.} -- queimada.

\textbf{dusuzé} \textit{s.} -- julho.

\textbf{du'wa} \textit{s.} -- palha para a casa, capim alto, sapé.

%###########################################################


\textbf{E}.



\textbf{e} -- interrogativo  introduz toda a interrogação  \textbf{e wa há} Quem?.

\textbf{ẽ} \textit{v. du.} -- quebrar, romper  \textbf{õ hã}, \textbf{e na ti'ẽ} ele quebrou.

\textbf{ẽzé} -- . . .  \textbf{sib'ẽzé} faca  \textbf{si'aböri'ẽzé} espada  \textbf{sitob'ẽzé} -- chave.

\textbf{é'é} \textit{s.} -- \textbf{ĩ'é'é} zumbido.

\textbf{ẽne} \textit{s.} -- pedra, mineral.

\textbf{éré} \textit{adv.} -- quase.

\textbf{ẽtẽ} \textit{s.} -- \textbf{ẽne} pedra, mineral.

\textbf{ẽtẽ} \textit{s.} -- pedra branca.

\textbf{ẽtẽ'awaipo} \textit{s.} -- pedra branca que brilha, prata.

\textbf{ẽtẽ zazu} \textit{s.} -- pedra cinzenta.

\textbf{ẽtẽzazuptire} \textit{s.} -- pedra cinzenta pesada, chumbo.

\textbf{ẽtẽzahöpö'ri} \textit{s.} -- cas.ercada pelos morros  Aldeia São José (R. I. São Marcos).

\textbf{ẽtẽzub'a} \textit{s.} -- pó branco de pedra, cal.

\textbf{ẽtẽzubtede} \textit{s.} -- pó duro de pedra, cimento.

\textbf{ẽtẽzuhiptede} \textit{s.} -- pó de pedra que se torna duro, cimento, concreto.

\textbf{ẽtẽnho'repré} \textit{s.} -- morro de garganta vermelha (Aldeia São Marcos).

\textbf{ẽtẽna'rada} \textit{s.} -- bigorna.

\textbf{ẽtẽ pa} \textit{s.} -- pedra comprida.

\textbf{ẽtẽpa} \textit{s.} -- grupo etário.

\textbf{ẽtẽ po} \textit{s.} -- pedra estensa, pedra larga.

\textbf{ẽtẽpo} \textit{s.} -- laje, alicerce.

\textbf{ẽtẽprétede} \textit{s.} -- pedra vermelha dura, cobre.

\textbf{ẽtẽprétete wazari} \textit{s.} -- bronze.

\textbf{ẽtẽpréwaipo} \textit{s.} -- pedra vermelha brilhante, ouro.

\textbf{ẽtẽpro} \textit{s.} -- carvão de pedra.

\textbf{ẽtẽ'rãihö} \textit{s.} -- pedra alta, monte, montanha.

\textbf{ẽtẽ'ãirã} \textit{s.} -- pedra branca.

\textbf{ẽtẽ'ãire} \textit{s.} -- pedrinha, cascalho.

\textbf{ẽtẽ're} \textit{s.} -- buraco na pedra, gruta, caverna.

\textbf{ẽtẽtede} \textit{s.} -- pedra dura, ferro.

\textbf{ẽtẽtedehöbö} \textit{s.} -- chapa de ferro.

\textbf{ẽtẽtete wa'õno} \textit{s.} -- metade, pedaço de ferro, barra.

\textbf{ẽtẽ si'ubuzi} \textit{s.} -- pedra que brilha, pedra preciosa.

\textbf{ẽtẽsi'ubuzi-s.} -- diamante.

\textbf{ẽtẽsiwasi} \textit{s.} -- diamante.

\textbf{ẽtẽwaipo} \textit{s.} -- cristal.

\textbf{ẽtẽwano} \textit{s.} -- pedra que faz ruído, urânio.

\textbf{ẽtewa'õtõ na bödödi} \textit{s./posp.} -- calçada.

\textbf{ẽtẽwapu} \textit{s.} -- alumínio.

\textbf{ẽtẽwari} \textit{s.} -- aço.


%##############################################
\section*{H}.



\textbf{ha} \textit{interj.} -- de exclamação.

\textbf{hã} \textit{palavra enfática}.

\textbf{hã} \textit{s.} -- \textbf{ĩhã} cheiro.

\textbf{hadu} \textit{adj.} -- ainda.

\textbf{hãihã} \textit{v.} -- \textbf{ĩhãihã} chiar.

\textbf{ha'ö} \textit{v. sing.} = \textbf{po'reha'öri} = \textbf{dapo'rehaimrami} \textit{v. du.} -- esquecer.

\textbf{hapa} \textit{v.} -- enganar, mentir, fazer constrangido.

\textbf{hape} \textit{v.} -- enganar, mentir.

\textbf{hapese} \textit{v} -- \textbf{hape} enganar, mentir  \textbf{sadawa hape}.

\textbf{haré} \textit{adv.} -- assim, isso aí.

\textbf{haré} \textit{adv.} -- unicamente  \textbf{õne haré} -- diretamente.

\textbf{hato} \textit{v.} -- enfraquecer a vista, não ver bem, fechar os olhos  \textbf{ma tõ tomo hato} -- ele não viu bem, ele está com vista fraca.

\textbf{hãsi} \textit{v. pl.} -- = \textbf{dahörö} \textit{v. du.} chamar, clamar, gritar, convidar (uma pessoa).

\textbf{hãsi} \textit{s. pl.} -- -- grito de chamar, chamada, convite, clamor.

\textbf{hawi} \textit{posp.} -- de, de onde (lugar).

\textbf{hawimhã} -- sendo de lá.

\textbf{hawipsi} -- somente de.

\textbf{he'ere} \textit{s.} -- canto dos "wapté".

\textbf{he'ere za'ru} \textit{s.} -- lugar do "he'ere".

\textbf{hepãrĩ} \textit{interj. excl.} -- de elogio, obrigado!.

\textbf{hereroi'wa} \textit{s.} -- "wapt̃é" após a furação das orelhas.

\textbf{hi} \textit{s.} -- perna, osso.

\textbf{hi} \textit{s.} -- \textbf{ĩhi} velho, antigo, ancião.

\textbf{hidiba} \textit{s.} -- irmã do homem.

\textbf{hizé} \textit{s.} -- dor de perna (osso).

\textbf{hi'ẽ} \textit{vmp.} -- quebrar osso.

\textbf{hipãrĩ} \textit{v.} -- sangrar para matar  \textbf{ĩsõ're hipãrĩ} -- sangrar no pescoço.

\textbf{hipopo} \textit{s.} -- dança noturna.

\textbf{hipu} \textit{s.} -- lesão de perna.

\textbf{hi'rada} \textit{s.} -- antepassados, antepassados falecidos.

\textbf{hi'rata} \textit{s.} -- antepassados, antepassados falecidos.

\textbf{hi'rãti} \textit{s.} -- joelho.

\textbf{hi'rãtitõ} \textit{adj.} -- de joelhos, ajoelhado.

\textbf{hire} \textit{s.} -- \textbf{ĩhire} velhinho.

\textbf{hi'ré} \textit{s.} -- osso seco, paralítico.

\textbf{hiri} \textit{v. sing.} = \textbf{noarĩ} \textit{v. du.} -- colocar, pôr, botar, deixar.

\textbf{hitébré} \textit{s.} -- irmão da mulher.

\textbf{hisihöri} \textit{s.} -- fratura.

\textbf{hisiwairĩ} \textit{s.} -- distorção.

\textbf{hi'wa} \textit{s.} -- "wapté" ao grupo etário anterior.

\textbf{hiwĩrĩ} \textit{s. sing.} -- \textbf{hipãrĩ} \textit{v. du.} sangrar para matar.

\textbf{hö} \textit{s.} -- casa do wapté.

\textbf{hö} \textit{s.} -- \textbf{ĩhö} casca.

\textbf{hö} \textit{s.} -- pele.

\textbf{hö} \textit{s.} -- \textbf{ĩhö} frio.

\textbf{hö} \textit{adj.} -- frio  \textbf{hö di} -- está frio.

\textbf{hö} \textit{adj.} -- mal-humorado, zangado  \textbf{hö'ö di} -- está zangado, está mal-humorado.

\textbf{hö} \textit{v. pl.} = \textbf{da'ahöri} \textit{v. du.} -- bater  \textbf{hö za'ra na} batam! \textbf{sina hö} aumentar, produzir.

\textbf{hö} \textit{s.} -- capricho, pirraça.

\textbf{hö} \textit{v.} = \textbf{romhõ} criar, renovar.

\textbf{hö'a} \textit{s.} -- coruja.

\textbf{hö'a} \textit{s.} -- \textbf{ĩhö'a} chefe, dirigente, cacique.

\textbf{hö'aprã} \textit{s.} -- \textbf{ĩhö'aprã} clérigo, quase-chefe.

\textbf{ho'are} \textit{s.} -- coruja.

\textbf{höbö} \textit{adj.} -- \textbf{ĩhöbö} chato, plano, baixo.

\textbf{höbö} \textit{v.} -- achatar, aplainar, baixar, abaixar, rebaixar.

\textbf{hödö} \textit{s.} -- machado de pedra, basalto, granito.

\textbf{hözahutu} \textit{s.} -- fim do frio, fim da friagem.

\textbf{hözano} \textit{v.} -- beliscar.

\textbf{höza'õno} \textit{adj.} -- grosso, pele grossa.

\textbf{hözapré} \textit{vmp.} -- avermelhar.

\textbf{hözé} \textit{s.} -- doença, dor, febre.

\textbf{hözu} \textit{v. pl.} = \textbf{wa'ré} \textit{v. sing. du.} -- ferir com projéteis, flechas, dar injeção.

\textbf{höiba s.} -- vida, existência, corpo pessoa, corpo vivo.

\textbf{höiba} \textit{v.} -- viver.

\textbf{höiba v.} -- \textbf{ĩhöiba} existir, ter, haver.

\textbf{höiba'ahö} \textit{s.} -- plural (gramatical).

\textbf{höiba amo} \textit{s.} -- \textbf{ĩhöiba amo} outro, outra pessoa.

\textbf{höibabarĩ} \textit{s.} -- \textbf{ĩhöibabarĩ} costura, remendo de roupa.

\textbf{höibabarĩ} \textit{v.} -- \textbf{ĩhöibabarĩ} costurar, consertar roupa.

\textbf{höibabarĩ} \textit{v.} -- costurar pele humana, fazer sutura, suturar.

\textbf{höibabarĩ'wa} \textit{s.} -- \textbf{ĩhöibabarĩ'wa} costureiro, costureira, alfaiate.

\textbf{höibabarĩ'wa} \textit{s.} -- -- cirurgião.

\textbf{höiba bö} \textit{s./adv.} -- \textbf{ĩhöiba bö} cada, todo, para todos.

\textbf{höibazahu} \textit{s.} -- dual (gramatical).

\textbf{höibazé poré} \textit{s.} -- dor geral.

\textbf{höiba'é} \textit{s.} -- cirurgia.

\textbf{höiba'é'wa} \textit{s.} -- cirurgião.

\textbf{höibamisi} \textit{s.} -- singular (gramatical).

\textbf{höibane} \textit{adj. comp.} -- semelhante.

\textbf{höibanhomrĩ} \textit{s.} -- comunhão.

\textbf{höiba pibuzé} \textit{s.} -- consulta.

\textbf{höibaprozé} \textit{s.} -- crematório.

\textbf{höiba'ré} \textit{s.} -- paralisia.

\textbf{höibarezu} \textit{s. comp.} -- micróbio.

\textbf{höiba'rénhipãrĩ} \textit{s.} -- tétano.

\textbf{höibarére} \textit{s.} -- célula.

\textbf{höibaresimapa'wa} \textit{s.} -- \textbf{ĩhöibaresimapa'wa} anticorpo.

\textbf{höibaresimapa'wadö} \textit{s.} -- \textbf{ĩhöibasimapa'wadö} AIDS.

\textbf{höibarĩ} \textit{s.} -- imagem, pintura, fotografia.

\textbf{höibarĩzaro} \textit{s.} -- slide.

\textbf{höibarĩzé} \textit{s.} -- máquina fotográfica.

\textbf{höibasi'rãmi} \textit{s.} -- ataque (doença), falha.

\textbf{höiba'uptabi} \textit{s.} -- espírito.

\textbf{höiba'uptabi ĩpe} \textit{s.} -- \textbf{Ĩhöiba'uptabi Ĩpe} Espírito Santo.

\textbf{höiba'uptabire} \textit{s.} -- alma.

\textbf{höiba warõ} \textit{s.} -- corpo inanimado, corpo sem vida, cadáver.

\textbf{höibödö} \textit{s.} -- pele riscada, pele aranhada.

\textbf{höimana} \textit{s.} -- vida, existência, cultura, conduta  \textbf{ĩ'reb höimana} conteúdo, conter (\textbf{da-} de gente  \textbf{ĩ}- outros).

\textbf{höimanazahi waima} \textit{s.} -- barbaridade.

\textbf{höimana} \textit{v.} -- viver, existir, estar presente.

\textbf{höimanazé} \textit{s.} -- vida, vivência, cultura, comodidade.

\textbf{höimanazé ĩpisudu} \textit{s.} -- caráter.

\textbf{höimanazé pipa} \textit{s.} -- brutal.

\textbf{höimanazé pipa} \textit{s.} -- tragédia.

\textbf{höimanazé pire} \textit{s.} -- drama.

\textbf{höimana'madö'özé} \textit{s.} -- televisor, TV.

\textbf{höimananhizu} \textit{s.} -- teatro.

\textbf{höimananhizusizé} \textit{s.} -- imitação.

\textbf{höimanaprédu} \textit{s.} -- comportamento, boa educação, respeito.

\textbf{höimanarĩ} \textit{s.} -- filme.

\textbf{höimanarĩzé} \textit{s.} -- filmadora.

\textbf{höimana'ruzahi} \textit{s.} -- legislação, carta magna, constituição.

\textbf{höimana'ruzahi 'manharĩ} \textit{s.} -- a constituinte.

\textbf{höimana'ruzahi 'manharĩ'wa} \textit{s.} -- o constituinte.

\textbf{höimana'rupisudu} \textit{s.} -- comunismo.

\textbf{höimana'rupisutu'wa} \textit{s.} -- comunista.

\textbf{höimana'ubuni} \textit{s.} -- castidade.

\textbf{Höimana'u'ö} \textit{s.} -- Deus (o que existe sempre).

\textbf{Höimana'u'öhö} \textit{s.} -- \textbf{Höimana'u'ö} Deus.

\textbf{höimana'uptabi nhomrizé} \textit{s.} -- sacramento.

\textbf{höimana wasu'uzé} \textit{s.} -- biografia.

\textbf{höimana wasu'u nhihödö} \textit{s.} -- certificado, certidão.

\textbf{höimanawẽ} \textit{s.} -- bondade.

\textbf{höimo} \textit{adv.} -- para cima, cima.

\textbf{höipe} \textit{adj.} -- \textbf{ĩhöipe} gordo.

\textbf{höirã} \textit{s.} -- roupa branca, pele branca.

\textbf{höi'ré} \textit{v.} -- aparecer, nascer.

\textbf{höi'ré} \textit{v.} -- \textbf{ĩhöi'ré} mostrar, revelar.

\textbf{höi'ro} \textit{v.} -- aquecer  \textbf{ma tihöi'ro} esquentou.

\textbf{höiwa} \textit{s.} -- céu.

\textbf{höiwaza} \textit{s.} -- crepúsculo.

\textbf{höiwa za'uri} \textit{s.} -- cerimônia no fim da caça, cerimônia contra a chuva.

\textbf{höiwazu} \textit{v.} -- \textbf{ĩhöiwazu} rasgar, descascar, esfolar.

\textbf{höiwahö} \textit{s.} -- tarde.

\textbf{höiwahö za'rã} \textit{s.} -- véspera da tarde.

\textbf{höiwa na} \textit{s./posp.} -- celeste.

\textbf{höiwa na'rada} \textit{s.} -- horizonte.

\textbf{höiwane'uzé} \textit{adj. comp.} -- azul.

\textbf{höiwa nhirõno} \textit{s.} -- nuvem.

\textbf{höiwa nhirõtõ} \textit{s.} -- \textbf{höiwa nhirõno} nuvem.

\textbf{höiwapré} \textit{vmp.} -- amanhecer, madrugar  \textbf{ma tihöiwapré} amanheceu.

\textbf{höiwara} \textit{s.} -- correia.

\textbf{höiwaratede} \textit{s.} -- \textbf{ĩhöiwaratede} estilingue.

\textbf{höiwari} \textit{s.} -- \textbf{ĩhöiwari} borracha.

\textbf{höiwarobo} \textit{s.} -- \textbf{ĩhöiwarobo} papel, livro, folha.

\textbf{höiwarobore} \textit{s.} -- \textbf{ĩhöiwarobore} caixa de papelão.

\textbf{höiwa u} \textit{s./posp.} -- para o céu.

\textbf{höiwa'u} \textit{s.} -- leite (de gente).

\textbf{höiwa'u} \textit{s.} -- \textbf{ĩhöiwa'u} leite de animal.

\textbf{höiwa'udu} \textit{vmp.} -- ressuscitar, ressurgir, renovar.

\textbf{höiwa'utuzé} \textit{s.} -- ressureição, renovação.

\textbf{höiwi} \textit{adv.} -- em cima, de pé, para cima.

\textbf{höiwi} \textit{v.} -- levantar-se  \textbf{te oto höiwi} ele levantou-se então.

\textbf{höiwi} \textit{s.} -- avião.

\textbf{höiwibduri} \textit{s.} -- avião de carga.

\textbf{höiwiza'o} \textit{s.} -- helicóptero.

\textbf{höiwi za'ru} \textit{s.} -- campo de aviação.

\textbf{höiwizé} \textit{s.} -- combustível de avião.

\textbf{homo} \textit{v.} -- \textbf{ĩhomo} chamuscar, queimar de leve, sapecar.

\textbf{hönhibudu} \textit{s.} -- bico do mamilo.

\textbf{hönhihöri} \textit{vmp. du. pl.} -- rasgar.

\textbf{hönhimi'rãmi} \textit{s.} -- arrepios.alafrios, alergia.

\textbf{hönhimnhirobo} \textit{s.} -- vento do sul, vento frio.

\textbf{hö'ö} \textit{v} -- aumentar, produzir.

\textbf{hö'ö} \textit{s.} -- raiva, pirraça, mal-humorado (adj. ).

\textbf{ho'orã} \textit{s.} -- espécie de macaco.

\textbf{höpa} \textit{adj. comp.} -- \textbf{ĩhöpa} comprido, comprido no corpo (roupa).

\textbf{höpãrĩ} \textit{vmp.} -- comer  \textbf{'mahöpãri} comer em excesso.

\textbf{höpö} \textit{adj. c.} -- \textbf{ĩhöpö} plaino, chato.

\textbf{höpö} \textit{v.} = \textbf{ĩhöbö} -- aplainar, achatar, baixar, abaixar a cabeça.

\textbf{höpö} \textit{s.} -- alergia.

\textbf{höpö'õno} \textit{s.} -- bolo grande  \textbf{nonhama höpö'õno} bolo grande de milho.

\textbf{höpö'ré} \textit{s.} -- omoplata.

\textbf{höpöré} \textit{s.} -- omoplata.

\textbf{höpösapodo} \textit{s.} -- disco.

\textbf{höpré} \textit{adj. comp.} -- \textbf{ĩhöpré} vermelho, púrpura.

\textbf{hö'rã} \textit{v.} -- beliscar.

\textbf{hö'rada} \textit{s.} -- roupa velha, pele enrugada, enrugado.

\textbf{hö'ratare} \textit{adj. comp. dim.} -- \textbf{ĩhö'ratare} velho, usado, passado.

\textbf{höre} \textit{s.} -- resfriado, calafrio.

\textbf{hö're} \textit{s.} -- bolso, bolsa.

\textbf{hö'rẽne} \textit{v. sing.} = \textbf{hösi} \textit{v. du.} -- beber.

\textbf{höri} \textit{v.} = \textbf{romhöri} acalmar, pacificar, fazer as pazes, defender, interromper, parar.

\textbf{hõrõ} \textit{v.} -- assobiar, apitar.

\textbf{hörö} \textit{v. du.} -- chamar, gritar, buzinar, bradar, berrar, chiar  \textbf{dazö dahörö} convidar, chamar alguém.

\textbf{hörö} \textit{s.} -- grito, chamada, apitada, assobiada, brado, berro.

\textbf{hörözé} \textit{s.} -- \textbf{ĩhörözé} buzina.

\textbf{höröre} \textit{s.} -- \textbf{ĩhöröre} apito.

\textbf{hörö'wa} \textit{s.} -- \textbf{ĩhörö'wa} juiz de jogo.

\textbf{höröwawã} \textit{s.} -- eco.

\textbf{höta} \textit{v. sing.} -- \textbf{hotari} = \textbf{dahösirĩ} \textit{v. du.} rasgar, esfolar  \textbf{ma höta} ele rasgou.

\textbf{hötari} \textit{v. sing.} -- \textit{v. du.} -- rasgar, esfolar.

\textbf{höté} \textit{adj. comp.} -- novo, jovem, recém-.

\textbf{hötede} \textit{s.} -- besouro.

\textbf{höti} \textit{v.} -- convalescer.

\textbf{höti} \textit{s.} -- convalescência (saliencia na pele).

\textbf{hõti} \textit{v.} -- jogar-se, lançar-se.

\textbf{höto} \textit{s.} -- calo, tipo de sarampo, varíola, varicela velha.

\textbf{hötö} \textit{adj.} -- \textbf{hötö'ö di} receioso, preocupado.

\textbf{hötoire} \textit{s.} -- eczema.

\textbf{hötoiwawẽ} \textit{s.} -- catapora.

\textbf{hötö'ö} \textit{adj. c.} -- \textbf{ĩhötö} receioso, preocupado.

\textbf{hötöra} \textit{s.} -- machado.

\textbf{hötörã} \textit{s.} -- peixe, peixe riscado  grupo etário.

\textbf{höto'rã} \textit{s.} -- cólera.

\textbf{hötörãmreme} \textit{s.} -- toca-fita.

\textbf{hötöra'rãpo} \textit{s.} -- enxada.

\textbf{hötöra'rãpo pisudu} \textit{s.} -- enxadão.

\textbf{hötöratama'a} \textit{s.} -- formão, espátula.

\textbf{hõsi} \textit{v} -- \textbf{hõ} = \textbf{romhõ} renovar, criar.

\textbf{hösi} \textit{v. du.} -- beber.

\textbf{hösi} \textit{s.} -- bebida, sede.

\textbf{hösizé} \textit{s.} -- objeto para beber, copo.

\textbf{hösizé baihö} \textit{s.} -- caneca.

\textbf{hösizé da'udö na} \textit{s./v./posp.} -- bebedouro.

\textbf{hösinarĩ} \textit{v. pl.} = \textbf{dahösirĩ} \textit{v. du.} -- rasgar pele, rasgar camisa, despedaçar.

\textbf{hösirĩ} \textit{v. du.} -- rasgar, esfolar.

\textbf{hösi wasédé} \textit{s.} -- bebedeira.

\textbf{hösu'u} \textit{vmp.} -- alisar, aplainar.

\textbf{hösu'u} \textit{vmp.} -- passar roupa.

\textbf{hö'uwa} \textit{s.} -- chorão.

\textbf{hö'wa} \textit{s.} -- "wapté!" (modo de chamar pelos pais).

\textbf{hu} \textit{s.} -- onça.

\textbf{hu} \textit{v. sing.} = \textbf{huri} -- praticar ato sexual.

\textbf{hunhi'a} \textit{s.} -- mosquito.

\textbf{hunhizeé} \textit{s.} -- bruma, neblina.

\textbf{hu'õ} -- (animal) não cortar, não partir.

\textbf{huri} \textit{s. pl.} -- comer.

\textbf{huri} \textit{s.} -- ato sexual.

\textbf{hutu} \textit{v. pl. c.} -- \textbf{dasimasisi} \textit{v. du.} -- vir, chegar.

\textbf{hutu} \textit{v.} = \textbf{tãma hutu} -- bater um pouco e parar logo.

\textbf{hu'uzérépa} \textit{s.} -- onça de cabelo comprido, leão.

\textbf{hu'u tede'wa} \textit{s.} -- dono da onça.

\textbf{hu'usiwa'rã} \textit{s.} -- pantera negra, onça preta.

\textbf{hu'u'upto} \textit{s.} -- onça pintada, jaguar, leopardo.

\textbf{hu'uwaratede} \textit{s.} -- leopardo.

\textbf{hu'uwaw̃i} \textit{s.} -- tigre.



\textbf{I}.



\textbf{i} \textit{s.} -- cupim.

\textbf{ĩ-} (longo) \textit{pref. pessoa} -- 1ª pessoa sing. .

\textbf{ĩ-} (breve) \textit{pref. pessoa} -- 3ª pessoa sing. .

\textbf{ĩda} \textit{abr.} = \textbf{ĩ'rada} -- tio! avô!.

\textbf{ĩhe} \textit{interj.} -- sim.

\textbf{ĩhöiba'uptabi} \textit{s.} -- espírito.

\textbf{ĩhöiba'uptabi Ĩpe} \textit{s.} -- Espírito Santo.

\textbf{ihu nho'u} \textit{s.} -- muito cupinzeiro.

\textbf{ĩmropö} \textit{num.} -- seis.

\textbf{ĩmrotõ} \textit{num.} -- cinco.

\textbf{ĩtehutunhisip'ri} \textit{s.} -- casas no campo  Aldeia Pizzato.

\textbf{Ĩ'ubumro Ĩpe} \textit{s.} -- Igreja, Povo de Deus.

\textbf{ĩwẽnhomri} \textit{s.} -- sacramento.


%&##################################################
\section*{M}.



\textbf{ma} \textit{posp.} -- para, a.

\textbf{ma} \textit{pron. rel.} -- você que, ele que  \textbf{õ hã, ma tirẽ} ele que abandonou, \textbf{a hã, ma aiwĩ} você que vem.

\textbf{mã} \textit{s.} -- ema (\textit{Rhea americana}).

\textbf{'ma-} \textit{pref.} -- a ação do verbo se realiza só uma vez  \textbf{'mahörö} \textit{v.} chamar uma só vez.

\textbf{ma'ãpé} -- vamos! então! avante!.

\textbf{'madö'ö} \textit{v} -- ver, olhar, observar.

\textbf{'madö'öze} \textit{adj. comp.} -- \textbf{ĩ'madö'özé} bonito, belo, agradável (para ver), vistoso.

\textbf{'madö'özé} \textit{s.} -- \textbf{ĩ'madö'özé} visão, aspecto, observação.

\textbf{'madö'ö'wa} \textit{s.} -- \textbf{ĩ'madö'ö'wa} fiscal, zelador, guarda, observador.

\textbf{'maza} \textit{v. du.} -- firmar, colocar dentro  \textbf{tame, te maza} -- ele firma.

\textbf{mazazöri} \textit{v. sing.} -- parar.

\textbf{mazahöri} \textit{v. du./pl.} -- parar.

\textbf{'mazapo} \textit{v.} -- abaixar-se, agachar-se.

\textbf{'mazasi} \textit{v. du.} -- entrar por dentro, colocar dentro.

\textbf{mazasizé} \textit{s.} -- lugar por dentro.

\textbf{mazasu} \textit{s.} -- pena de ema.

\textbf{maze} \textit{fem.} -- \textbf{maze di} não tem, não.

\textbf{mahã} \textit{pron. ind.} -- algo, alguém  \textbf{e mahã ta} onde está ele?.

\textbf{maha wi} \textit{pron. ind./posp.} -- de onde  \textbf{e maha wi} de onde?.

\textbf{'mahöpãri} \textit{v. du.} -- fartar-se.

\textbf{'mahörö} \textit{v. du.} -- chamar uma vez, chamar atenção.

\textbf{'mahörözé} \textit{s.} -- telefone.

\textbf{'mahörö'ru} \textit{s.} -- imperativo (modo gramatical).

\textbf{'mairéme} \textit{v.} -- omitir, abandonar, largar uma coisa.

\textbf{'mairẽme'õ'wa} \textit{s.} -- que não abandona uma s.oisa.

\textbf{'mai'rẽne} \textit{v.} -- comer uma s.oisa.

\textbf{'mai'u'a} \textit{v.} -- machucar, ferir uma só vez.

\textbf{'mai'utõrĩ} \textit{v.} -- passar uma s.oisa.

\textbf{'maiwẽ} \textit{v.} -- oferecer, dar, gostar uma coisa.

\textbf{mama} \textit{s.} -- pai, tio, padrinho.

\textbf{mama amo} \textit{s.} -- tio paterno.

\textbf{mama'ẽ} \textit{v. pl.} = \textbf{ẽ} \textit{v. du.} -- quebrar, romper.

\textbf{mama wapté} \textit{s.} -- tio materno.

\textbf{mame} \textit{pron.} -- onde  \textbf{e mame} onde?.

\textbf{mameni} \textit{v. sing.} = \textbf{dawabzuri} \textit{v. du.} -- jogar.

\textbf{mana} . . .  \textbf{pizo'omanane} -- serrote.

\textbf{manadö} \textit{adj.} -- pertencente a "pahöri'wa" e "tébé".

\textbf{manahö} \textit{s.} -- prepúcio.

\textbf{manahöire} \textit{s.} -- prepúcio.

\textbf{mana'rada} \textit{s.} -- pintura de riscos.

\textbf{manawa} \textit{posp.} -- em cima de  \textbf{ĩmanawa wi} de cima dele, por fora.

\textbf{manawazu'rã} \textit{s.} -- pintura preta de "wapté".

\textbf{manharĩ} \textit{v.} = \textbf{ro'manharĩ} -- fazer, realizar, cometer  \textbf{marĩ 'manharĩ} -- construir.

\textbf{'manharĩzé} \textit{s.} -- ação, realização, fato.

\textbf{'manharĩ'wa} \textit{s.} -- que realiza, que faz, realizador.

\textbf{'manharĩ'wa} -- \textit{pi'uruwi rob'uipra 'manhari'wa} contrabandista.

\textbf{'manharĩ'wa} \textit{s.} -- \textit{marĩ 'manharĩ'wa} construtor, trabalhador.

\textbf{'manherẽ} \textit{v. sing.} = \textit{ĩza} \textit{v. du.} -- enfiar, meter uma só vez.

\textbf{'manhi'ãsi} \textit{v.} -- fazer coroa, coroar.

\textbf{mapa} \textit{v. sing.} -- \textit{mapari} = \textit{danhimipari} esperar.

\textbf{maparane} \textit{num.} -- dois, bis.

\textbf{maparanesi'uiwana} \textit{num. comp.} -- quatro.

\textbf{mapari} \textit{v.} -- \textit{danhimipari} esperar.

\textbf{'mapãrĩ} \textit{v. du.} -- matar.

\textbf{'mapra} \textit{v. sing.} -- \textit{'mapraba} guiar, correr, carregar.

\textbf{'mapraba} \textit{v.} -- guiar, correr, carregar.

\textbf{'maprabazé} \textit{s.} -- \textit{da'maprabazé} ônibus.

\textbf{'mapraba'wa} \textit{s.} -- \textit{ĩ'mapraba'wa} motorista, guia.

\textbf{maprewa} \textit{s.} -- \textit{damaprewa} sogro, sogra, sogros.

\textbf{'mapru} \textit{v.} -- \textit{ĩpru} destruir, quebrar.

\textbf{mapu} \textit{s.} -- esponja, espuma  \textit{ĩwa'ré̃mapu} esponja  \textit{mapuza'éré} esponja.

\textbf{mara} \textit{s.} -- noite.

\textbf{mara} \textit{v.} -- anoitecer  \textit{ma tõ timara} anoiteceu.

\textbf{marã} \textit{s.} -- mata, floresta.

\textbf{marãire} \textit{s.} -- bosque.

\textbf{mararé} \textit{adv.} -- cedo, de manhã.

\textbf{mara wa'wa} \textit{s./posp.} -- meia-noite, à meia-noite.

\textbf{marawé!} -- boa noite!.

\textbf{mare} \textit{masc.} -- \textit{mare di} não tem, não.

\textbf{marĩ} \textit{s.} -- coisa, causa.

\textbf{marĩ} \textit{pron. ind.} -- que, o que, que coisa  \textbf{e marĩ} como? que? o que?.

\textbf{marĩ'ahö} \textit{adj. comp.} -- \textit{tãma ĩmarĩ'ahö} rico.

\textbf{marĩ ãna} \textit{pron. ind./posp.} -- porque, sem mais.

\textbf{marĩ bö} -- causa.

\textbf{marĩ da} \textit{pron. ind./posp.} -- para que  \textbf{marĩ da ĩwẽ} útil  \textbf{e marĩ da} para que.

\textbf{marĩ za'ra} -- carga.

\textbf{marĩ höimana} -- existe, vive algo, tem, há, é, está algo, caso.

\textbf{marĩ höimana pisudu} -- concluir, conclusão, marcação, indicação, existência fixa.

\textbf{marĩ ĩzazé} -- computador.

\textbf{marĩ ĩhöimana'õ} -- não ter nada, carecer.

\textbf{marĩ 'manharĩ} \textit{pron./v.} -- construir, fazer algo.

\textbf{mari na} -- causa.

\textbf{marĩ na dapo'repu'u} -- comemorar.

\textbf{marĩ na dapo'repu'uzé} -- comemoração.

\textbf{marĩ na saprĩ} -- comércio.

\textbf{marĩ'õ} \textit{adj. comp.} -- \textit{tãma ĩmarĩ'õ} não tem nada, pobre.

\textbf{marĩ parizé} \textit{pron. ind./s.} -- batedeira.

\textbf{marĩ pu} -- circunstância, caso, acontecimento.

\textbf{marĩ 'rare} -- alguma coisa pequena, bagatela.

\textbf{marĩ résizé} -- máquina.

\textbf{marĩ tobozé} \textit{pron./s.} -- cola.

\textbf{marĩ waihu'u pe} -- capaz, competência, capacidade.

\textbf{marĩ wasu'uzé nhoré} -- catálogo.

\textbf{marĩ wẽ} -- bem.

\textbf{marĩ wi ĩsito} -- certo.

\textbf{'maropãrĩ} \textit{v. du.} -- vencer, derrubar, espatifar, matar jogando.

\textbf{'marosimro} \textit{v. pl.} = \textbf{'maropãrĩ} \textit{v. du.} -- vencer, derrubar, espatifar, matar jogando.

\textbf{'marowĩ} \textit{v. sing.} -- derrubar, vencer, espatifar.

\textbf{'marowĩrĩ} \textit{v. sing.} = \textbf{'maropãrĩ} \textit{v. du.} -- vencer, derrubar, espatifar, matar jogando.

\textbf{'mata} \textit{v. sing.} \textbf{'matarĩ} = \textbf{'mairĩ} \textit{v. du.} -- pegar, tirar folha, fruta, galho  \textbf{ma 'mata} -- ele pega, tira.

\textbf{'matari} \textit{v. sing.} = \textbf{'mairĩ} \textit{v. du.} -- tirar, quebrar, pegar uma só vez.

\textbf{masa} \textit{s.} -- içá, fêmea de formiga içá (cortadeira).

\textbf{'masãmri} \textit{v.} -- encontrar algo.

\textbf{masa nhimhi} \textit{s.} -- símbolo da formiga.

\textbf{'masi} \textit{v. sing.} -- \textbf{masisi} encher, completar.

\textbf{'masi} \textit{v. du.} -- \textbf{si} \textit{v. du.} comer.

\textbf{'masisi} \textit{v. du./pl.} -- encher, completar.

\textbf{ma'u} \textit{s.} -- pato.

\textbf{ma'utubupa} \textit{s.} -- cisne.

\textbf{ma'uzahi} \textit{s.} -- ganso.

\textbf{'ma'wamarĩ} \textit{v.} -- limpar, arrumar algo, pintar algo.

\textbf{me!} \textit{interj.} -- não sei! talvez!.

\textbf{me} \textit{v. sing.} = \textbf{wabzuri} \textit{v. du.} -- jogar, meter, lançar, afastar.

\textbf{me} \textit{posp.} -- com  \textbf{ĩme} comigo, \textbf{aime} contigo, \textbf{wame} conosco.

\textbf{mi} \textit{s.} -- lenha.

\textbf{mimi} \textit{s.} = \textbf{mi} -- lenha.

\textbf{misi} \textit{num.} -- um.

\textbf{misi-zahu} \textit{num.} -- cem, cento.

\textbf{misi-zahu na ãma ĩsahu} -- cêntuplo.

\textbf{misi-zahu wahu} -- cem anos.entenário, século.

\textbf{misi haré} -- uma só vez, único.

\textbf{misi si} -- somente um, um por um, um por vez.

\textbf{mo} \textit{v. sing.} -- \textbf{morĩ} = \textbf{dane} \textit{v. du.} ir, andar.

\textbf{mo} \textit{adv.} = \textbf{mono} -- em toda a parte  \textbf{'re . . . mono} sempre . . . em toda a parte.

\textbf{mohõni} \textit{s.} -- abelha preta.

\textbf{momo} \textit{pron.} -- onde, para onde  \textbf{e momo} -- onde? para onde?.

\textbf{mono} \textit{adv.} -- em todo o lugar, em toda a parte  \textbf{'re . . . mono} -- sempre . . . em toda a parte.

\textbf{mo'õni} \textit{s.} -- cara.

\textbf{mo'õni'a} \textit{s.} -- batata branca.

\textbf{mo'õniwapru} \textit{s.} -- batata vermelha.

\textbf{morĩ} \textit{s. sing.} = \textbf{dane} \textit{v. du.} -- ir, andar, caminhar  \textbf{we morĩ} vir  \textbf{dawi morĩ} = \textbf{dawi dane} ir-se, afastar-se dele.

\textbf{morĩzé} \textit{s.} -- caminhada, ida  \textbf{we morĩzé} vinda.

\textbf{morĩ'rada} \textit{adj. comp.} -- primeiro.

\textbf{mra} \textit{s.} -- comida, fome.

\textbf{mrabre} \textit{s.} -- borrachudo (mosquito).

\textbf{mrami} \textit{v. du.} -- pegar, cativar.

\textbf{mrami'wa} \textit{s.} -- polícia, quem pega.

\textbf{mramre} \textit{s.} -- mosquito miúdo.

\textbf{mre} \textit{v.} -- \textbf{ĩmre} contrair.

\textbf{mre} \textit{s.} -- \textbf{ĩmre} contração, ruga.

\textbf{mre} \textit{v. sing.} = \textbf{mreme} -- falar.

\textbf{mreme} \textit{v. sing.} -- falar.

\textbf{mreme} \textit{s.} -- fala, língua.

\textbf{mremezahu} \textit{adj. comp.} -- bilíngue.

\textbf{mreme zahu} \textit{s./v.} -- repetir.

\textbf{mremezé} \textit{s.} -- língua, linguagem.

\textbf{mremezu} \textit{v. pl. comp.} -- falar conversando, orar, rezar, louvar.

\textbf{mremezusi} \textit{v. pl. c.} -- falar conversando, orar, rezar, louvar.

\textbf{mreme'rãihözé} \textit{s.} -- microfone.

\textbf{mremeta'azé} \textit{s.} -- rádio transmissor.

\textbf{mro} \textit{s.} -- esposo, esposa, marido, cônjuge.

\textbf{mro} \textit{v.} -- reunir, juntar  \textbf{ãma ĩmro} contar  \textbf{dasi na damro} unir.

\textbf{mrozahu} \textit{s.} -- casamento repetido, bigamia, concubinato.

\textbf{mrozé} \textit{s.} -- reunião, união, coleção  \textbf{ãma ĩmrozé} contabilidade  \textbf{dasina damrozé} casamento.

\textbf{mro'õ} \textit{adj. comp.} -- \textbf{ĩmro'õ} não-unido, não casado, celibatário.

\textbf{mrotõ} \textit{adj. comp.} -- \textbf{ĩmrotõ} sem par, sem esposo, sem esposa = \textbf{ĩmro'õ}.

\textbf{mrotõ} \textit{num.} -- \textbf{ĩmrotõ} cinco.

\textbf{mrotõ} \textit{s.} -- \textbf{ĩmrotõ} viúvo, viúva, desquitado, divorciado.

\textbf{mro'wa} \textit{s.} -- \textbf{ãma ĩmro'õ ĩmro'wa} contador.

\textbf{mro waihu'uzé} \textit{v./s.} -- censo.



\textbf{N}.



\textbf{na} \textit{posp.} -- em, de, dentro de.

\textbf{na} \textit{s.} -- mãe, tia, madrinha.

\textbf{na} \textit{s.} -- rola.

\textbf{naire} \textit{s.} -- rolinha.

\textbf{napa} \textit{s.} -- braços da mãe, colo da mãe, seio.

\textbf{napré} \textit{s.} -- juriti.

\textbf{na'rada} \textit{s.} -- \textbf{ĩna'rada} começo, início, base.

\textbf{na'ratasito} \textit{vmp.} -- \textbf{ĩna'ratasito} confluir.

\textbf{na'rata'wa} \textit{s.} -- \textbf{ĩna'rata'wa} iniciador, pioneiro.

\textbf{na'rata wasi} \textit{s./v.} -- \textbf{ĩna'rata wasi} amarrar uma ponta.

\textbf{nare} \textit{adv.} -- realmente, de verdade.

\textbf{na're} \textit{s.} -- nádegas, traseiro.

\textbf{nasi} \textit{adv.} -- sempre, costuma.

\textbf{na wapté} \textit{s.} -- tia materna.

\textbf{ne} \textit{adj.} -- semelhante, parecido, como.

\textbf{ne} \textit{v. du.} -- ir, andar, caminhar  \textbf{wa wanem ni} nós dois vamos.

\textbf{nebzé si'uhi} \textit{s.} -- corredor.

\textbf{neb ré} \textit{v.} -- andando, indo, caminhando.

\textbf{neme} \textit{adv.} -- neste lugar.

\textbf{nemo} \textit{adv.} -- muito.

\textbf{newa} \textit{adv.} -- eventualidade.

\textbf{nhada'öbö} \textit{v} -- \textbf{sada'öbö} responder, retrucar.

\textbf{nhama} \textit{s.} -- \textbf{ĩnhama} = \textbf{ĩzö} grão, semente.

\textbf{nhamare} \textit{s.} -- esperma, sêmen.

\textbf{nhamare} \textit{s.} -- \textbf{ĩnhamare} sementinha.

\textbf{nhamnha} \textit{s.} -- guacho.

\textbf{nhamra} \textit{v. sing.} -- \textit{v. du.} sentar-se, estar sentado, estar deitado, deitar-se  \textbf{te nhamra} ele está sentado, deitado.

\textbf{nhamra} \textit{v. pl.} = \textbf{sãmra} = \textbf{wabzuri} \textit{v. du.} -- jogar fora, despejar, lançar, destruir.

\textbf{nhamrazé} \textit{s.} -- lugar de permanecer, banco, cadeira, visita.

\textbf{nhamri} \textit{v.} -- trançar.

\textbf{nhana} \textit{s.} -- fezes, intestino, intestino grosso.

\textbf{nhana za'ẽne} \textit{s.} -- intestino grosso.

\textbf{nhanahi} \textit{s.} -- umbigo.

\textbf{nhanahözé} \textit{s.} -- indigestão.

\textbf{nhanahöri'wa} \textit{s.} -- mulher que corta cordão umbilical.

\textbf{nhananhihörizé} \textit{s.} -- taquara para cortar cordão umbilical.

\textbf{nhanapré} \textit{s.} -- barriga vermelha  \textbf{ai'uté nhanapré} verme de criança.

\textbf{nhanarãpré} \textit{s.} -- verme.

\textbf{nhana'rãsuture} \textit{s.} -- apêndice (intestino).

\textbf{nhana're} \textit{num.} -- segundo (2º).

\textbf{nhana'rézé} \textit{s.} -- dificuldade de defecar, parar disenteria, dispepsia.

\textbf{nhana'remhã} \textit{num.} -- estando em segundo lugar.

\textbf{nhana syry} \textit{s.} -- intestino delgado.

\textbf{nhana'u} \textit{s.} -- diarréia.

\textbf{nhanawai'u} \textit{s.} -- vermes.

\textbf{nharĩ} \textit{v.} -- falar, dizer, contar.

\textbf{nharĩzé} \textit{s.} -- fala, conversa, conto.

\textbf{nheme} \textit{v. pl. c.} -- \textbf{sẽme} = \textbf{ĩza} \textit{v. du.} meter, colocar.

\textbf{nheme'wa} \textit{s.} -- = \textbf{dazai'wa} \textit{s. du.} -- metedor, colocador.

\textbf{nherẽ} \textit{conj.} -- apesar de, mesmo que, pois.

\textbf{nherẽ} \textit{v. sing.} = \textbf{sẽrẽ} = \textbf{daza} \textit{v. du.} -- meter, colocar.

\textbf{nherẽzé} \textit{s. pl.} -- = \textbf{dazazé} \textit{s. du.} pousada, túmulo.

\textbf{nhi} \textit{s.} -- carne, músculo.

\textbf{nhi'ã} \textit{v.} -- -- coroar.

\textbf{nhi'ãsi} \textit{v} = \textbf{si'ãsi} -- coroar.

\textbf{nhi'a'wa} \textit{s.} -- -- chuva.

\textbf{nhi'a'waire} \textit{s.} -- -- chuvisco.

\textbf{nhibzari} \textit{s.} -- oferta, bondade, presente, dom.

\textbf{nhibzu} \textit{s.} -- poeira da gente.

\textbf{nhib'ö} \textit{s.} -- água pertencente à pessoa.

\textbf{nhib'rã} \textit{s.} -- cordão umbilical.

\textbf{nhib'rada} \textit{s.} -- mão.

\textbf{nhib'ratabto} \textit{vmp.} -- bater palmas.

\textbf{nhib'ratahi} \textit{s.} -- dedo.

\textbf{nhib'ratapo} \textit{s.} -- palma da mão.

\textbf{nhib'rata prabazé} \textit{s.} -- corrimão.

\textbf{nhib'ri} \textit{s.} -- casa que ele fez  \textbf{a'uwẽ nhib'ri} casa xavante.

\textbf{nhibro s.} -- lugar dele, pertences.olônia, propriedade dele, bens.

\textbf{nhibro 'manharĩ} \textit{s./v.} -- colonizar, instalar-se.

\textbf{nhibrobuduri} \textit{s.} -- carona.

\textbf{nhibrob'o} \textit{adj. comp.} -- pobre, pobre sem nada, vagabundo, cigano.

\textbf{nhib'ru} \textit{s.} -- ira, raiva, coragem, braveza.

\textbf{nhib'ru} \textit{adj.} -- corajoso, irado, raivoso.

\textbf{nhib'uwa} \textit{adj.} -- fraco, delicado.

\textbf{nhib'uwa} \textit{s.} -- fraqueza, delicadeza.

\textbf{nhib'uware} \textit{adj. dim.} -- fraco, delicado.

\textbf{nhidöpösi} \textit{adj.} = \textbf{sidöpösi} tudo, sempre, toda vez que  \textbf{bötö nhidöpösi} todos os dias.

\textbf{nhize} \textit{s.} -- \textbf{ĩnhize} carne não gostosa.

\textbf{nhizé} \textit{s.} -- fumaça, neblina  \textbf{unhama nhizé} fumaça.

\textbf{nhize'õ} \textit{s.} -- \textbf{ĩnhize'õ} carne não gostosa, não comestível.

\textbf{nhizöri} \textit{v. sing.} -- \textbf{sizöri} = \textbf{sihöri} v. du. cortar  \textbf{ti'a nhizöri} região limitada.

\textbf{nhihöbö} \textit{v.} -- encurvar-se.

\textbf{nhihödö} \textit{s.} -- riscado, escrito, carta, livro.

\textbf{nhihözé} \textit{s.} = \textbf{dawa're nhihözé} -- remédio.

\textbf{nhihöiwada} \textit{s.} -- prepúcio.

\textbf{nhihöri} \textit{v. du./pl.} -- \textbf{sihöri} cortar.

\textbf{nhihöri'wa} \textit{s.} = \textbf{sihöri'wa} -- cortador.

\textbf{nhihötö} \textit{v} -- escrever.

\textbf{nhihötö'wa} \textit{s.} -- \textbf{ĩsihötö'wa} escritor, escrivão  \textbf{rowasu' nhihötö'wa} escritor.

\textbf{nhihudu} \textit{s.} -- neto, descendente.

\textbf{nhihutu} \textit{s.} -- neto, descendente.

\textbf{nhimhmatã} \textit{s.} -- padrinho do sobrinho.

\textbf{nhim'apito} \textit{s. neo.} -- cacique, chefe.

\textbf{nhimhi} \textit{s.} -- \textbf{simhi} formiga.

\textbf{nhimizadaihu} \textit{v.} -- entender, atender por palavras.

\textbf{nhimizadaihu'u} \textit{v} -- entender, atender por palavras.

\textbf{nhimizaze} \textit{s.} -- fé, crença, confiança, convicção.

\textbf{nhimizazezé} \textit{s.} -- fé, credo.

\textbf{nhimizaze nhiptetezé} \textit{s.} -- confirmação.

\textbf{nhimizazöri} \textit{v. sing.} -- \textit{v. du.} parar.

\textbf{nhimizahöri} \textit{v. du./pl.} -- parar.

\textbf{nhimizahörizé} \textit{s.} -- lugar de pousada, parada.

\textbf{nhimizama} \textit{s.} -- animal doméstico, escravo, submisso.

\textbf{nhimiza'õno} \textit{s.} -- cotovelo.

\textbf{nhimiza're} \textit{adj.} -- esperto, perito.

\textbf{nhimiza're} \textit{s.} -- razão, sabedoria, cautela.

\textbf{nhimiza're} \textit{adv.} -- juntamente, prontamente, imediatamente.

\textbf{nhimiza're} \textit{s.} -- caução.

\textbf{nhimiza'rese} \textit{s.} -- razão, sabedoria, cautela.

\textbf{nhimizawi} \textit{s.} -- caridade, amor, benevolência.

\textbf{nhimizawipe} \textit{adj. comp.} -- manso, benigno, cordial, clemente.

\textbf{nhimizu} \textit{s.} -- pulso da mão.

\textbf{nhimizutoto} \textit{s.} -- pulso (batida).

\textbf{nhimizusi} \textit{v} -- colocar de pé.

\textbf{nhimi'e} \textit{adj.} -- esquerda, a mão esquerda.

\textbf{nhimi'ẽ} \textit{adj.} -- aplicado, dedicado, diligente, trabalhador.

\textbf{nhimihö} \textit{s.} -- briga, rixa, discórdia.

\textbf{nhimimnha} \textit{s.} -- perigo, perigoso.

\textbf{nhiminho're} \textit{s.} -- glândula.

\textbf{nhimnhasi} \textit{s.} -- perigo, perigoso.

\textbf{nhim'ĩhö'a} \textit{s.} -- chefe, cacique, liderança, dirigente.

\textbf{nhim'ĩhö'a nhimhu} \textit{s.} -- cetro.

\textbf{nhimihöze} \textit{s.} -- briga, rixa, discórdia.

\textbf{nhimihöze} \textit{adj. comp.} -- briguento.

\textbf{nhiminho'ru} \textit{s.} -- inveja, ciúme.

\textbf{nhimini} \textit{v.} -- perder-se, desviar, omitir, desrespeitar.

\textbf{nhimipari} \textit{v.} -- esperar.

\textbf{nhimiparizé} \textit{s.} -- esperança, lugar de esperar.

\textbf{nhimi'rãmi} \textit{v.} -- enrugar  \textbf{dahö nhimi'rãmi} alergia, arrepio.

\textbf{nhimire} \textit{adj.} -- a direita.

\textbf{nhimirowa'a} \textit{s.} -- luz da pessoa.

\textbf{nhimi'ru} \textit{s.} -- ordem.

\textbf{nhimi'ru} \textit{v.} -- instar, insistir, mandar.

\textbf{nhimisõ} \textit{s.} -- lavação, ablução das mãos.

\textbf{nhimi'ubumro} \textit{s.} -- discípulo.

\textbf{nhimi'uri} \textit{v.} -- camuflar, esconder, extraviar.

\textbf{nhimi'wa} \textit{s.} -- \textbf{ĩsimi'wa} unha do animal, casco.

\textbf{nhimiwazé} \textit{s.} -- respeito.

\textbf{nhimiwanho} \textit{s.} -- padrinho, afilhado.

\textbf{nhimiwãrã} \textit{s.} -- sangue do parto, dores de parto.

\textbf{nhimi'wara} \textit{v.} -- parar, pousar, estar de pé.

\textbf{nhimiwẽ} \textit{s.} -- benevolência.

\textbf{nhimnhana} \textit{v. pl.} -- = \textbf{daza'o} \textit{v. du.} pendurar.

\textbf{nhimnhatã} \textit{v. pl. c.} -- = \textbf{daza'o} \textit{v. du.} pendurar.

\textbf{nhimnhasi} \textit{s.} -- confiança, desconfiança.

\textbf{nhimhasi} \textit{v.} -- pedir a quem tem, querer, desconfiar.

\textbf{nhimnhi'ã} \textit{s.} -- arma, arco.

\textbf{nhimnhihörö} \textit{s.} -- concerto, melodia.

\textbf{nhimnhohu} \textit{s.} -- padrinho, madrinha.

\textbf{nhimnho'rebzu} \textit{s.} -- afilhado.

\textbf{nhimro} \textit{v. pl.} -- \textbf{asimro} \textbf{ubumro} = \textbf{dasimasisi} \textit{v. du.} reunir, unir, sentar-se, ficar.

\textbf{nhimro} \textit{v. pl.} -- = \textbf{dapãri} \textit{s. du.} matar.

\textbf{nhine} \textit{adj.} = \textbf{anhine} = \textbf{sine} = \textbf{ne} -- semelhante, igual.

\textbf{nhinha} \textit{v.} -- dobrar  \textbf{hönhinha} dobrar roupa, pano.

\textbf{nhinihã} \textit{s.} -- cedro  \textbf{wede nhinihã} cedro.

\textbf{nhi'odo} \textit{v.} -- mandar de volta, voltar.

\textbf{nhi'oto} \textit{v} -- mandar de volta, voltar.

\textbf{nhi'oto} \textit{adj. c.} -- torto, defeituoso.

\textbf{nhi'õtõ} \textit{s.} = \textbf{nhi'u} -- . . .  \textbf{bödödi nhi'u} rodovia, avenida.

\textbf{nhipa} \textit{adj.} -- \textbf{nhi'u} acima de, superior, por cima de.

\textbf{nhipai u} \textit{adj./posp.} -- acima de, por cima de, além de.

\textbf{nhipai'u} \textit{adv.} -- sobretudo.

\textbf{nhipai'wa} \textit{s.} -- chefe, senhor.

\textbf{nhipe} \textit{adj.} -- engenhoso.

\textbf{nhipi} \textit{v.} = \textbf{sipi} -- cozinhar.

\textbf{nhipisu} \textit{s.} -- que faz as coisas depressa.

\textbf{nhipo} \textit{s.} = \textbf{sipo} -- unha, unha da gente.

\textbf{nhipta höto} \textit{s.} -- calo.

\textbf{nhiptede} \textit{s.} -- força.

\textbf{nhiptetezé} \textit{s.} -- força, fortificante.

\textbf{nhiptete'wa} \textit{s.} -- que dá força, que dá apoio.

\textbf{nhipti} \textit{s.} -- \textbf{ĩsipti} -- ramal, galho, ramo, corrego, cabeceira  \textbf{wedepa nhipti} galho.

\textbf{nhipti} \textit{adj.} -- \textbf{ĩsipti} acrescentado, amais, ramificação.

\textbf{nhipto} \textit{s.} -- dedo.

\textbf{nhiptõmo} \textit{s.} -- dedo.

\textbf{nhiptõmohi} \textit{s.} -- osso do dedo  \textbf{danhiptõmohi wasisizé} anel do dedo.

\textbf{nhiptoro} \textit{s.} -- flecha.

\textbf{nhiptoro} \textit{v.} -- fazer flechas.

\textbf{nhipsaihuri} \textit{v.} -- roubar, furtar.

\textbf{nhipsi} \textit{s.} -- pulseira de corda.

\textbf{nhipsisizé} \textit{s.} -- braçalete, amarras do pulso.

\textbf{nhirã} \textit{s.} -- rasto, pegada.

\textbf{nhirã} \textit{s.} -- gengiva, céu da boca.

\textbf{nhir'a} \textit{v.} = \textbf{ĩsi'ra} -- descer, baixar, abaixar.

\textbf{nhirãrã} \textit{s.} -- \textbf{ĩsirãrã} flor  \textbf{romnhirãrã} flor.

\textbf{nhirẽ} \textit{s.} -- coroa dos cabelos, tonsura.

\textbf{nhi'riti} \textit{v.} -- andar desviando, driblar.

\textbf{nhirobo} \textit{s.} -- espuma, mofo  \textbf{si nhi'robo} penugem.

\textbf{nhirobo} \textit{s.} -- resfriado, constipação, catarro.

\textbf{nhiromduri} \textit{s.} -- carona.

\textbf{nhirobo'u'õzé} \textit{s.} -- cois.om que se enxuga o catarro de gente, lenço.

\textbf{nhirõno} . . .  \textbf{höiwa nhirõno} nuvem.

\textbf{nhirõtõ} \textit{s.} . . .  \textbf{höiwa nhirõno} nuvem.

\textbf{nhiti} \textit{posp.} -- longe de, fora de, afastado de.

\textbf{nhiti'ru} \textit{s.} -- raiva, ira.

\textbf{nhiti'ru} \textit{adj.} -- zangado, mal-humorado.

\textbf{nhito} \textit{adj.} -- = \textbf{ĩsito} fechado.

\textbf{nhitobzé} \textit{s.} -- cárcere, cadeia, prisão.

\textbf{nhitobzé} \textit{s.} -- \textbf{ĩsitobzé} fechadura, lugar fechado.

\textbf{nhitobzébré} \textit{s.} -- cadeia, prisão, cárcere.

\textbf{nhitob'ru} \textit{s.} -- inimigo.

\textbf{nhitõ'wa} \textit{adj.} -- último grupo etário.

\textbf{nhitowa} \textit{adj.} -- \textbf{ĩsitowa} aberto.

\textbf{nhisé} \textit{s.} -- pudor, vergonha, respeito.

\textbf{nhisé} \textit{s.} -- ombro.

\textbf{nhiséb'õ} \textit{adj. comp.} -- desrespeitoso, sem vergonha.

\textbf{nhisi} \textit{posp.} -- em cima de, por cima de  \textbf{nhisi u} \textit{posp./posp.} ao topo de.

\textbf{nhisi} \textit{s.} -- nome.

\textbf{nhisi'ã} \textit{s.} -- \textbf{ĩsisi'ã} curva, coroa, colar de algodão.

\textbf{nhisi'ãsi} \textit{s.} -- \textbf{ĩsisi'ã} curva, coroa, colar de algodão.

\textbf{nhisi're} \textit{s.} -- nariz.

\textbf{nhisi're wapru} \textit{s.} -- sangramento do nariz.

\textbf{nhisiri} \textit{v.} -- espirrar.

\textbf{nhisisi} \textit{v. pl.} -- = \textbf{dazasi} \textit{v. du.} -- entrar.

\textbf{nhisisizé} \textit{s.} -- porta, entrada.

\textbf{nhisi nhipti} \textit{s./posp.} -- sobrenome.

\textbf{nhisiwi} \textit{posp.} -- em cima de, por cima de.

\textbf{nhisu} \textit{s.} = \textbf{sisu} folha de buriti.

\textbf{nhi'u} \textit{s.} -- esperma, sêmen.

\textbf{nhi'u} \textit{s.} -- \textbf{ĩsi'u} suco, caldo  \textbf{arãrã nhi'u} suco de beija-flor, néctar.

\textbf{nhi'umadtõ} \textit{num. c.} -- três  \textbf{mara nhi'umadtõ} três noites.

\textbf{nhi'uwa} \textit{adj.} -- leve, tenro, suave (pele, corpo, carne).

\textbf{nhiwã} \textit{s.} -- capacete.

\textbf{nhi'wa} \textit{s.} -- ferramenta de corte e ponta.

\textbf{nhiwairĩ} \textit{s.} -- distorção.

\textbf{nhiwari} \textit{v.} -- pedir coisas.

\textbf{nhiwasari} \textit{s.} -- reumatismo.

\textbf{nho} \textit{s.} -- mantimento, comida de gente, fruto, colheita, cereal.

\textbf{nho'a} \textit{posp.} -- na presença de.

\textbf{nhozahö} \textit{s.} -- náusea.

\textbf{nhozaré} \textit{s.} -- caçula.

\textbf{nhozarési} \textit{s.} -- caçula.

\textbf{nhohu} \textit{v.} -- ser padrinho.

\textbf{nhohui'wa} \textit{s.} -- padrinho.

\textbf{nhomo'a} \textit{adj. c.} -- \textbf{ĩsõmo'a} aberto, afunilado.

\textbf{nhomri} \textit{v} -- \textbf{sõmri} dar.

\textbf{nhomrizé} \textit{s.} -- dom, dádiva.

\textbf{nhomri'wa} \textit{s.} -- \textbf{ĩsõmri'wa} doador.

\textbf{nhonhi'ã} \textit{s.} -- clavícula.

\textbf{nhonhi'ã} \textit{s.} -- colar de algodão de "wapté".

\textbf{nhonhihöbö} \textit{s.} -- lado do peito.

\textbf{nhono} \textit{v.} -- dormir.

\textbf{nhono} \textit{s.} -- sono, festa, festa de furar orelha, iniciação para a vida adulta.

\textbf{nho'o} \textit{s.} -- vômito.

\textbf{nho'o} \textit{v.} -- vomitar.

\textbf{nho'õmo} \textit{posp.} -- \textbf{ĩsõ'õmo} em direção de, rumo a.

\textbf{nho'õmo} \textit{s.} -- \textbf{ĩsõ'õmo} = \textbf{nho'u} esteio  \textbf{ri nho'õmo} esteio da casa.

\textbf{nho'õmo} \textit{adj. c.} -- \textbf{nho'u} = \textbf{danho'u} todos.

\textbf{nhopo} \textit{s.} -- . . .  \textbf{tonhopa} carnaval.

\textbf{nhopẽtẽ} \textit{v.} -- \textbf{sõpẽne} encontrar.

\textbf{nhopré} \textit{v.} -- procurar, refletir, olhar, ver.

\textbf{nhopre} \textit{s.} -- . . . .

\textbf{nhopre wairĩ} \textit{s.} -- corda na cabeça.

\textbf{nhopru} \textit{adj.} -- generoso.

\textbf{nhoprubzé} \textit{s.} -- generosidade.

\textbf{nho'rã} \textit{s.} -- risco preto na barriga.

\textbf{nho'rada} \textit{s.} -- sobra da comida, resto da comida, migalha.

\textbf{nho're} \textit{s.} -- oração, reza, canto.

\textbf{nho're} \textit{v.} -- orar, rezar, celebrar, ler, enumerar.

\textbf{nho're} \textit{posp.} -- em frente de  \textbf{nho'remhã} estando em frente de.

\textbf{nho'ré} \textit{s.} -- fila, leitura.

\textbf{nho're'a} \textit{s.} -- rouquidão.

\textbf{nho'rebzö} \textit{v.} -- arrasar, extinguir, matar a todos.

\textbf{nho'rebzu} \textit{s.} -- colar.

\textbf{nho'rebzu'a} \textit{s.} -- colar de algodão.

\textbf{nho'rebzuzé} \textit{s.} -- medalha.

\textbf{nho'rebzu'wa} \textit{s.} -- irmão da mãe com função de padrinho.

\textbf{nhorezé} \textit{s.} -- proclamação, fila de gente.

\textbf{nho'reptu} \textit{v.} -- salvar, resgatar.

\textbf{nho'reptuzé} \textit{s.} -- salvação.

\textbf{nho'reptu'wa} \textit{s.} -- salvador.

\textbf{nho're'ré} \textit{s.} -- gola, garganta.

\textbf{nho're'rézé} \textit{s.} -- dor de garganta.

\textbf{nho'reti} \textit{s.} -- estrangulamento.

\textbf{nho'reti} \textit{v.} -- estrangular, enforcar.

\textbf{nho're'wa} \textit{s.} -- puxador de canto.

\textbf{nho'rewazari} \textit{s.} -- coro.

\textbf{nhorõ} \textit{s.} -- veia, corda.

\textbf{nhorõ} \textit{v.} -- torcer corda, trançar corda.

\textbf{nho'rõni} \textit{s.} -- jaguatirica.

\textbf{nho'rõnire} \textit{s.} -- gato.

\textbf{nhorõtede} \textit{s.} -- tendão.

\textbf{nhorõwa} \textit{s.} -- casa.

\textbf{nho'ru} \textit{s.} -- inveja, ciúme, ira.

\textbf{nhoti} \textit{adj.} -- avarento, mesquinho.

\textbf{nhoti} \textit{s.} -- avareza.

\textbf{nhoto} \textit{s.} -- língua.

\textbf{nhotõ} \textit{s. (v. ) c.} -- sono, dormir.

\textbf{nhotõzé} \textit{s.} -- cobertor, dormitório, cama, lugar de dormir.

\textbf{nhotõ'uza} \textit{s.} -- pijama, camisola.

\textbf{nhosi} \textit{s.} -- irmão segundo-nascido.

\textbf{nho'u} \textit{s.} -- esteio  \textbf{ri nho'u} esteio da casa, coluna central, lugar de morada.

\textbf{nho'u} \textit{adj.} -- \textbf{nho'uma} \textit{adj. c.} todos, muitos  \textbf{ri nho'uma} \textit{s.} metrópole.

\textbf{nho'udu} \textit{s.} -- peito.

\textbf{nho'uma} \textit{s.} -- toda a gente, todos os povos, todos.

\textbf{nho'utu} \textit{s.} -- peito, coração.

\textbf{nho'utu'ru} \textit{s.} -- coração.

\textbf{nhowa} \textit{posp.} -- diante de, em frente de.

\textbf{nhowamhã} \textit{posp.} -- estando na frente, primeiro da fila.

\textbf{nhowi} \textit{s.} -- cigarra.

\textbf{nhowire} \textit{s.} -- cigarrinha.

\textbf{ni!} \textit{interj. excl.} -- cuidado!.

\textbf{ni'ã!} \textit{interj.} -- cuidado! aí! é isso aí!.

\textbf{niha} \textit{pron. ind.} -- quanto, como  \textbf{e niha} quanto? como?.

\textbf{nihane} \textit{s.} -- objeto, coisa, algo, como assim.

\textbf{nima} \textit{pron. ind.} -- para qualquer, para quem, alguém  \textbf{e nima ha ma} para alguém  \textbf{nima me} com alguém.

\textbf{nimomo} -- para onde, algum lugar.

\textbf{nimosi} -- agorinha, novo, recém-.

\textbf{ni'wa} \textit{pron. ind.} -- alguém.

\textbf{niwa} \textit{pron.} -- quando, casualmente, certa vez.

\textbf{niwamhã} -- qualquer dia.

\textbf{niwapsi} -- quando somente, mais tarde.

\textbf{niwi} \textit{posp.} -- de lado de  \textbf{danhimire niwi} à direita.

\textbf{no} \textit{s.} -- \textbf{ĩno} irmão mais novo.

\textbf{nozö} \textit{s.} -- milho xavante.

\textbf{nozö'a'upé} \textit{s.} -- antigo canto.

\textbf{nozöb'a} \textit{s.} -- milho xavante branco.

\textbf{nozöb'awawi} \textit{s.} -- milho rajado.

\textbf{nozöb'rã} \textit{s.} -- milho xavante preto.

\textbf{nozöbraretopré} \textit{s.} -- milho vermelho.

\textbf{nozö'u} \textit{s.} -- milho xavante preto, grupo etário.

\textbf{noire} \textit{s.} -- \textbf{ĩnoire} irmãozinho.

\textbf{nomo} \textit{s.} -- \textbf{dadu} = barriga, estômago.

\textbf{nomowa're} \textit{s.} -- úlcera de estômago.

\textbf{nomri} \textit{v. du.} -- colocar, botar, deixar, conceber  \textbf{romharé nomri} conciliar, fazer as pazes.

\textbf{nomrisiséré} -- composição, compor.

\textbf{nomro} \textit{v. sing.} = \textbf{ĩsa'warĩ} v. du. -- estar, deitar, permanecer, parar ficar, colocar.

\textbf{nonhama} \textit{s.} = \textbf{nozö} milho xavante.

\textbf{nonhama höpö'õno} \textit{s.} -- bolo de milho.

\textbf{nonhama'ubuté} \textit{s.} -- milho amarelo.

\textbf{nonhamawa'ãzé} \textit{s.} -- fevereiro.

\textbf{noni} \textit{s.} -- feixe de buriti.

\textbf{noniza'ozé} \textit{s.} -- lugar do "noni".

\textbf{noni mrami'wa} \textit{s.} -- carregador do "noni".

\textbf{noni nhonhi'ãsizé} \textit{s.} -- pau dentro do "noni".

\textbf{norĩ} \textit{des.} pronominal e substantiva do dual e plural.

\textbf{norõ} \textit{s.} -- coco de indaiá.

\textbf{norõzö} \textit{s.} -- coco da babaçu.

\textbf{norõzu} \textit{s.} -- castanha de babaçu.

\textbf{norõzupeetépe} \textit{s.} -- castanha dura mesmo  Aldeia N. S. Auxiliadora.

\textbf{norõihö nhorõ} \textit{s.} -- corda para arco.

\textbf{norõipo} \textit{s.} -- palmito de babaçu.

\textbf{norõ'rã} \textit{s.} -- coco.

\textbf{norõre} \textit{s.} -- coco de indaiá.

\textbf{norõre waipo} \textit{s.} -- broto de palmeira indaiá.

\textbf{norõsu'rã} \textit{s.} -- babaçu de folha preta  Aldeia Couto Magalhães.

\textbf{norõwada} \textit{s.} -- tucano (\textit{Ramphastos toco}).

\textbf{norõwede} \textit{s.} -- babaçu.

\textbf{norõwedezahö} \textit{s.} -- estojo peniano.



\textbf{O}.



\textbf{õ} \textit{adv.} -- não.

\textbf{õ} \textit{v.} -- dar; \textbf{ö ĩma õ na} dẽ-me água; \textbf{õ da} para dar.

\textbf{õ} \textit{pron. pessoa} = \textbf{õ hã} -- ele, ela, \textbf{õ norĩ hã} -- eles, elas.

\textbf{ö} \textit{s.} -- água (que corre).

\textbf{ö'a} \textit{s.} -- cachoeira, cascata.

\textbf{ö'a'a} \textit{s.} -- \textbf{õ'a} cachoeira, cascata.

\textbf{ö'a rãihö} -- catarata.

\textbf{ö'amramizé} \textit{s.} -- balde.

\textbf{öbduri} \textit{s.} -- navio.

\textbf{öduri nhimizahörizé} \textit{s.} -- lugar de parada de navio, cais.

\textbf{odo} \textit{s.} -- cigarra.

\textbf{öza} \textit{antj.} -- \textbf{a'öza} recusa de pedido.

\textbf{özada'ré} \textit{s.} -- limite da água, praia, costas do mar, barranco.

\textbf{özaihö} \textit{s.} -- beira-d'água.

\textbf{özaipro} \textit{s.} -- cerveja, chope.

\textbf{özaipro 'manharĩzé} \textit{s.} -- cervejaria.

\textbf{öze} \textit{s.} -- refresco (água gostosa).

\textbf{özé} \textit{s.} -- bebida alcoólica (água doída).

\textbf{özéze} \textit{s.} -- vinho.

\textbf{özéze'rãire} \textit{s.} -- uva.

\textbf{õ hã} \textit{pron. pessoa/enf.} -- ele, ela, \textbf{õ norĩ hã} -- eles, elas.

\textbf{õhõ} \textit{pron. dem.} -- aquele, aquela.

\textbf{õhõ ta} -- ele está lá.

\textbf{öhuri} \textit{vmp. pl.} -- beber.

\textbf{oi'o} \textit{s.} -- luta de meninos batendo-s.om raíz.

\textbf{õme} -- com aquele, lá.

\textbf{õmo} \textit{s.} -- \textbf{ĩ'õmo} \textit{s.} = \textbf{ĩ'u} chifre.

\textbf{öna'awaru} \textit{s.} -- hipopótamo.

\textbf{önadasa} \textit{s.} -- soro.

\textbf{õne} -- semelhante àquele, daquele jeito, assi (lá).

\textbf{õne} -- \textbf{öne di} não achar bom, não gostar.

\textbf{õne haré} -- todos juntos, unidos, logo, direto.

\textbf{õne u'ö} \textit{adv.} -- sempre.

\textbf{õne u'ösi} \textit{adv} = \textbf{õne u'ö} -- sempre.

\textbf{önhana'rada} \textit{s.} -- cabeceira.

\textbf{önhidototzé} \textit{s.} -- chaleira.

\textbf{önhizé} \textit{s.} -- vapor.

\textbf{önhimiruru} \textit{s.} -- usina hidroelétrica.

\textbf{önhi'ré pibuzé} \textit{s.} -- bóia.

\textbf{önhi'rudu} \textit{s.} -- nome de aldeia.

\textbf{önhitobzé} \textit{s.} -- comporta.

\textbf{õniwi} -- daquele lado.

\textbf{öporé} \textit{s.} -- mar, oceano.

\textbf{öporé zaihö} \textit{s.} -- beira-mar, costas, praia.

\textbf{öpraba} \textit{s.} -- cascata.

\textbf{öpraba 'manharĩ} -- canal.

\textbf{ö pré} \textit{s.} -- água vermelha.

\textbf{öpré} \textit{s.} -- enchente.

\textbf{ö'rada} \textit{s.} -- urina.

\textbf{ö're} \textit{s.} -- Aldeia Dom Bosco.

\textbf{ö'rehö} -- água profunda, profundidade, baía.

\textbf{ö'rẽne} \textit{vmp. sing.} = \textbf{ösi} \textit{v. du.} -- beber.

\textbf{öri} \textit{v. sing.} = \textbf{mrami} \textit{v. du.} -- pegar, apanhar.

\textbf{oro} \textit{interj.} -- exclamação: por acaso! Puxa!.

\textbf{öro'ozé} \textit{s.} -- querosene.

\textbf{öro'ope} \textit{s.} -- álcool.

\textbf{ö tede'wa} \textit{s.} -- dono das águas.

\textbf{oti} \textit{s. adu.} -- filha, neta.

\textbf{oto} \textit{interj.} -- vamos!.

\textbf{oto} \textit{adv.} -- então.

\textbf{ötõmoza'rob'a} \textit{s.} -- Aldeia São Luís.

\textbf{ösi} \textit{vmp. du.} -- beber.

\textbf{ösipe} \textit{s.} -- ilha.

\textbf{ösisasi} \textit{s.} -- ondas do mar.

\textbf{õ'umnhasite} \textit{s.} -- possibilidade.

\textbf{õwa} \textit{adv.} -- lá.

\textbf{öwazarize} \textit{s.} -- refrigerante.

\textbf{öwahö'u'ẽne} \textit{s.} -- gelo.

\textbf{öwapso} \textit{vmp.} -- chapinhar, lavar com a mão, borbulhar.

\textbf{öwapso} \textit{s.} -- borbulho, agitação da água.

\textbf{ö wapu} \textit{s.} -- água leve, água morna.

\textbf{öwapu} \textit{s.} -- água quente, fonte térmica; Aldeia N. S. Aparecida (R. I. São Marcos).

\textbf{öwara} \textit{s.} -- córrego, água corrente.

\textbf{öwa'roze} \textit{s.} -- chá.

\textbf{öwawẽ} \textit{s.} -- rio grande, Rio das Mortes.lã.

\textbf{ö wi} \textit{s./posp.} -- afastado da água, longe da água.

\textbf{öwi} \textit{s.} -- rego d'água.



\textbf{P}.



\textbf{pa} \textit{s.} -- fígado.

\textbf{pa} \textit{s.} -- \textbf{ĩpa} galho, raiz, espuma, verdura, fígado de animal.

\textbf{pa} \textit{adj.} -- \textbf{ĩpa} comprido.

\textbf{pa} \textit{s.} -- córrego, riacho; \textbf{pa 'rata} perto do córrego, ao lado do córrego.

\textbf{pã} \textit{s.} -- alegria, satisfação pela boa caçada, sucesso, melodia monótona.

\textbf{padi} \textit{s.} -- tamanduá bandeira.

\textbf{padi} \textit{s.} -- (i longo) nome dele.

\textbf{pa zababa} \textit{s./posp.} -- beirando o córrego.

\textbf{pa zaihö} \textit{s.} -- beira do córrego.

\textbf{pazamarĩpese'wa} \textit{s.} -- computador.

\textbf{pazatobro} \textit{s.} -- \textbf{ĩpazatobro} raiz para remédio.

\textbf{pazé} \textit{s.} -- bilis, fel.

\textbf{pazépu} \textit{s.} -- iterícia.

\textbf{pahi} \textit{s.} -- medo, temor.

\textbf{pahi} \textit{adj.} -- \textbf{pahi di} está com medo, medroso.

\textbf{pahi} \textit{v.} -- ficar com medo, temer.

\textbf{pahi'õ} \textit{adj. comp.} -- \textbf{ĩpahi'õ} corajoso.

\textbf{pahiwasa} \textit{s.} -- urubu caçador.

\textbf{pahiwasédé} \textit{s.} -- covarde.

\textbf{pahõ} \textit{s. neo.} -- pão.

\textbf{pahö} \textit{v.} -- andar longe, prolongar, atirar longe; \textbf{wapahö za'ra nii} nós vamos longe.

\textbf{pahö} \textit{v. sing.} -- cortar galho.

\textbf{pahöri} \textit{v. du./pl.} -- cortar galho.

\textbf{pahöri'wa} \textit{s.} -- adorador do sol.

\textbf{pahöri'wa manadö} -- comida do "pahöri'wa" (carne da caça).

\textbf{paihi} \textit{s.} -- braço, membro.

\textbf{paihidi} \textit{s.} -- sabiá (\textit{Turdus leucomelas}).

\textbf{pa'madö'ö} \textit{v} -- olhar, ver o trabalho das mãos de alguém.

\textbf{pamrami} \textit{v. du.} -- pegar no braço, na mão.

\textbf{panhihö} \textit{s.} -- baço.

\textbf{panhihödö} \textit{s.} -- risco preto na braço.

\textbf{panhihöne} \textit{s.} -- tipo de arbusto.

\textbf{panhipti} \textit{s.} -- \textbf{ĩpanhipti} ramo, lateral, ramal.

\textbf{panhisihu} \textit{s.} -- capim para brincadeira.

\textbf{panho'õmo} \textit{s.} -- \textbf{panho'u} córrego.

\textbf{panho'u} \textit{s.} -- córrego.

\textbf{pano} \textit{s.} -- braço.

\textbf{pano pa} \textit{s.} -- \textbf{ĩpano pa} braço comprido.

\textbf{pamrami} \textit{v. du.} -- pegar no braço, na mão, conduzir.

\textbf{pa'o} \textit{s.} -- banana.

\textbf{pa'ö} \textit{v. sing.} -- \textbf{pa'öri} \textit{v. sing.} = \textbf{pamrami} \textit{v. du.} pegar na mão, no braço, conduzir.

\textbf{pa'öri} \textit{v. sing.} -- \textbf{pamrami} \textit{s. du.} pegar na mão, no braço, conduzir.

\textbf{papara} \textit{adj.} -- \textbf{ĩpapara} aos pés de, do.

\textbf{papo} \textit{adj.} -- \textbf{ĩpapo} braço aberto, braço estendido.

\textbf{papré} \textit{s.} -- \textbf{ĩpapré} raiz vermelha, cenoura.

\textbf{para} \textit{s.} -- pé, roda.

\textbf{para} \textit{posp.} -- dentro de; \textbf{ri para} dentro de casa.

\textbf{para} \textit{posp.} -- de acordo com, segundo.

\textbf{parabubu} \textit{s.} -- raiz para comer, batata azeda; Aldeia Parabubu.

\textbf{paraza} \textit{s.} -- padrinho no "wai'a".

\textbf{parazahu} \textit{s.} -- \textbf{ĩparazahu} bípede.

\textbf{parazu} \textit{s.} -- \textbf{ĩparazu} ponta, extremidade; \textbf{abazi parazu} ponta do algodão.

\textbf{parahi} \textit{s.} -- dedo do pé.

\textbf{parahizu} \textit{vmp.} -- pisar nos pés durante o "wai'a".

\textbf{parahö} \textit{s.} -- video-fita, videotape.

\textbf{parahö na ĩpodo} \textit{s./posp.} -- filme de videocassete.

\textbf{parahö podo} \textit{s.} -- filme de videocassete.

\textbf{para'idi'idi} \textit{s.} -- formigamento do pé.

\textbf{paranhipo} \textit{s.} -- dedo do pé.

\textbf{paranhipo'ore} \textit{s.} -- \textbf{ĩparanhipo'ore} garfo.

\textbf{paranhito} \textit{s.} -- tornozelo.

\textbf{para'õno} \textit{s.} -- calcanhar.

\textbf{para'õtõwe} \textit{num.} -- nove.

\textbf{parapo} \textit{s.} -- sola.

\textbf{para'ra} \textit{s.} -- \textbf{ĩpara'ra} ruído dos pés, sapatear.

\textbf{para'rasi} \textit{v} -- \textbf{ĩpara'ra} sapatear.

\textbf{pararĩrĩ} -- capinar, carpir roça, acerar.

\textbf{paratehö} \textit{s.} -- fita de som.

\textbf{para sipo} \textit{s.} -- rachadura do pé, pé rachado.

\textbf{para'ubuzé} \textit{s.} -- meia.

\textbf{para'uza} \textit{s.} -- sapato, calçado.

\textbf{para'uzaba'are} \textit{s.} -- chinelo.

\textbf{para'uza za} \textit{s./v.} -- calçar.

\textbf{para'uzazé} \textit{s.} -- calçado, sapato.

\textbf{para'uzazadazöri} \textit{s.} -- botina.

\textbf{para'uzahötede} \textit{s.} -- calçado de lona.

\textbf{para'uzanhi'õmore} \textit{s.} -- chinelo.

\textbf{para'uzapa} \textit{s.} -- bota, botina.

\textbf{para'uzatepa} \textit{s.} -- bota.

\textbf{para'upti} \textit{s.} -- bicho do pé.

\textbf{para'wa} \textit{s.} -- lutadores.

\textbf{parawã} \textit{s.} -- tição, pedaço de lenha.

\textbf{parawãza'razé} \textit{s.} -- ajuntamento de lenha; Aldeia Santa Maria.

\textbf{parawaihi} \textit{v.} -- molhar o pé.

\textbf{parawihõ} \textit{s.} -- \textbf{ĩparawaihõ} câmera de ar.

\textbf{pare} \textit{s.} -- córrego.

\textbf{pari} \textit{v.} -- apagar, limpar.

\textbf{pari} \textit{posp.} -- depois; \textbf{romhuri pari} depois do trabalho.

\textbf{pãrĩ} \textit{v. du.} -- matar.

\textbf{parinai'a} \textit{s.} -- menino milagroso.

\textbf{paripsi} \textit{adv.} -- depois somente.

\textbf{pãrĩ'wa} \textit{s.} -- matador, assassino, carrasco.

\textbf{pãrĩ wasédé} \textit{v./s.} -- chacina.

\textbf{patede} \textit{s.} -- raízes comestíveis.

\textbf{pati} \textit{s.} -- \textbf{padi} tamanduá.

\textbf{patire} \textit{s.} -- tamanduá-mirim, mixila.

\textbf{patiwa} \textit{s.} -- gordura de tamanduá.

\textbf{pa'ubumro} \textit{vmp.} -- amarrar, ajuntar braços.

\textbf{pa'uwati} \textit{v.} -- dissuadir (de brigar, de beber), desaconselhar.

\textbf{pawaibu} \textit{vmp. pl.} -- \textit{s. du.} pegar nos braços.onduzir.

\textbf{pawa'ö} \textit{s.} -- \textbf{ĩpawa'ö} pagamento, salário.

\textbf{pawa'ö} \textit{v. sing.} -- \textbf{ĩpawa'ö} pagar serviço, recompensar.

\textbf{pawa'öbö} \textit{v.} -- \textbf{ĩpawa'öbö} pagar serviço, recompensar.

\textbf{pawapto} \textit{v.} -- servir, fazer o bem, ajudar.

\textbf{pawapto'wa} \textit{s.} -- ajudante, servidor, servo.

\textbf{pawapu} \textit{s.} -- pulmão.

\textbf{pawapuhö're} \textit{s.} -- tuberculose.

\textbf{pawasi} \textit{v. sing.} -- amarrar, amarrar o pulso, prender.

\textbf{pawasi} \textit{adj.} -- amarrado, preso.

\textbf{pawasi} \textit{s.} -- cativeiro.

\textbf{pawasisi} \textit{v. du./pl.} -- amarrar, prender, amarrar os pulsos.

\textbf{pawasisizé} \textit{s.} -- amarras.

\textbf{pawi} \textit{s.} -- guatambu, cachimbo.

\textbf{pé!} \textit{interj.} -- vamos! então!.

\textbf{pe!} \textit{interj. fem.} -- excl. de admiração.

\textbf{pe} \textit{adj.} -- \textbf{ĩpe} perfeito, correto, santo.

\textbf{pe} \textit{v.} -- corrigir, consertar; \textbf{marĩ pe} corrigir, correção.

\textbf{pẽ} \textit{s.} -- abdômen, entranhas, saudade, pensamento.

\textbf{pe'a} \textit{s.} -- peixe.

\textbf{pezai'y} \textit{s.} -- boto.

\textbf{peza'i'y} \textit{s.} -- boto.

\textbf{peza'ywawẽ} \textit{s. aum.} -- baleia.

\textbf{pezapodo} \textit{s.} -- pacu.

\textbf{pẽ'ẽ} \textit{s.} -- pensamento, saudade.

\textbf{pẽ'ẽ} \textit{adj. c.} -- \textbf{ĩpẽ} saudoso, triste.

\textbf{pẽ'ẽzanĩ} \textit{s.} -- respiro, respiração, bafo, fôlego, aspiração, cheiro, espírito, Espírito Santo, consolador.

\textbf{pẽ'ẽzanĩ} \textit{v.} -- respirar, aspirar.

\textbf{pẽ'ẽzé} \textit{s.} -- saudade, tristeza, nostalgia.

\textbf{pẽ'ẽzé} \textit{adj.} -- \textbf{ĩpẽ'ẽdée} triste, saudoso.

\textbf{pẽ'ẽzépu} \textit{s.} -- dor de estômago.

\textbf{pẽ'ẽnomri} \textit{s.} -- vantagem.

\textbf{pẽ'ẽwara} \textit{v. sing.} = \textbf{pẽ'ẽsisamro} \textit{v. du.} -- assustar-se, entristecer-se.

\textbf{pehö} \textit{s.} -- couro de peixe.

\textbf{pehöire} \textit{s.} -- matrinxão.

\textbf{pẽne} \textit{v.} = \textbf{pẽtẽ} -- procurar.

\textbf{pepa} \textit{s.} -- peixe bicudo.

\textbf{pẽtẽ} \textit{v} = \textbf{pẽne} -- procurar.

\textbf{petob'rã} \textit{s.} -- peixe papatene.

\textbf{pese} \textit{v} -- \textbf{ĩpe} curar, aperfeiçoar, consertar, arrumar.

\textbf{pese} \textit{adv.} -- mesmo; \textbf{uburé pese} tudo mesmo.

\textbf{pese'wa} \textit{s.} -- curador, mestre, quem conserta.

\textbf{pe'wanhipti} \textit{s.} -- peixe cachorro.

\textbf{pe'watõ} \textit{s.} -- pacu.

\textbf{pĩ} \textit{s.} -- \textbf{ĩpĩ} mel.

\textbf{pi'ã} \textit{s.} -- anu vermelho.

\textbf{pibu} \textit{s.} -- tomar conta de, provar, experimentar; \textbf{marĩ pibui wẽ} conservar.

\textbf{pibui'wa} \textit{s.} -- protetor, guarda, anjo, polícia.

\textbf{pidu} \textit{s.} -- mutuca.

\textbf{piza} \textit{s.} -- cerâmica, panela, vasilha.

\textbf{piza'a zadazöri} \textit{s.} -- pote.

\textbf{piza'a höpöre} \textit{s./adj. dim.} -- bandeja.

\textbf{piza'a} \textit{s.} -- \textbf{piza} cerâmica, panela, vasilha.

\textbf{piza'a nhomo'a} \textit{s.} -- vasilha afunilada, aberta.

\textbf{piza'anhomo'a} -- bacia, copo, cálice.

\textbf{piza'apo} \textit{s.} -- caco.

\textbf{piza'apru} \textit{s.} -- caco.

\textbf{piza'a waipo} \textit{s.} -- copa, cálice.

\textbf{pizaiba} \textit{s.} -- cerâmica.

\textbf{pizari} \textit{v. du.} -- virar-se, voltar-se.

\textbf{pizawatapa} \textit{s.} -- bule.

\textbf{pizo'omanane} \textit{s.} -- serrote de mão.

\textbf{pizuré} \textit{s.} -- tesoura.

\textbf{pineiré} \textit{s.} -- abanador de arroz, peneira.

\textbf{pineiré ĩsi'uwazi na} -- batedeira.

\textbf{pĩni} \textit{s.} -- \textbf{pĩ} mel.

\textbf{pi'õ} \textit{s.} -- mulher, fêmea.

\textbf{pi'õiwapru a'amo bö} -- menstruação.

\textbf{pi'õ si'ĩhö'a} -- dama.

\textbf{pipa} \textit{s.} = \textbf{ropipa} -- perigo.

\textbf{pipa} \textit{adj.} -- \textbf{ĩpipa} perigoso, chocante.

\textbf{pi'ra} \textit{v. sing.} = \textbf{pizari} \textit{v. du.} -- virar, voltar-se.

\textbf{pire} \textit{adj.} -- \textbf{ĩpire} pesado, complicado, importante.

\textbf{pi'reba} \textit{adv.} -- embaixo.

\textbf{pire pibuzé} \textit{adj./s.} -- \textbf{ĩpire pibuzé} balança.

\textbf{piro} \textit{s.} -- borboleta.

\textbf{piro'o} \textit{s.} -- \textbf{piro} borboleta.

\textbf{pisudu} \textit{s.} -- indicação, determinação; \textbf{dazawi pisudu} aliança, \textbf{romhuri si'uiwa pisudu} convênio.

\textbf{pisudu} \textit{v.} -- indicar, determinar, escolher, marcar, tratar de, combinar; \textbf{dazawi pisudu} fazer aliança; \textbf{marĩ pisudu} concluir, conclusão.

\textbf{pisudu} \textit{s. aa.} -- cabeça; \textbf{a'uwẽ pisudu} cabeça de xavante.

\textbf{pisutu} \textit{v} -- \textbf{pisudu} indicar, determinar, escolher, marcar, tratar de, combinar.

\textbf{pisutu} \textit{s.} -- \textbf{pisudu} indicação, determinação.

\textbf{pi'u} -- abelha.

\textbf{pi'um nhiptoro} -- flecha de "pi'u".

\textbf{pi'u pĩ} \textit{s.} -- mel de abelha.

\textbf{pi'uri} . . . .

\textbf{pi'uriwi} \textit{adv.} -- escondido.

\textbf{pi'uriwi rob'uipra 'manharĩ'wa} -- contrabandista.

\textbf{po} \textit{s.} -- taquara fendida ou rachada, flecha de ponta larga; \textbf{wedepo} tábua.

\textbf{po} \textit{adj.} -- comprido; \textbf{po'repo} burro, jumento.

\textbf{pö} \textit{c.} -- \textbf{bö} por acaso.

\textbf{po} -- . . . ; \textbf{powawẽ} gado, vaca.

\textbf{po} \textit{v.} -- dividir, quebrar.

\textbf{podo} \textit{v.} -- criar, procriar, desenhar, fazer projetos.

\textbf{podo} \textit{v.} -- criatura, criação, imagem, fotografia, desenho.

\textbf{pozahi} \textit{s.} -- búfalo americano.

\textbf{pozasu} \textit{s.} -- llama.

\textbf{pozé} \textit{s.} -- cervo.

\textbf{pozé nhimi'wa} \textit{s.} -- casco de cervo.

\textbf{pozé'õmopo} \textit{s.} -- folga.

\textbf{pozé to} \textit{s.} -- olho de cervo.

\textbf{pozé'u} \textit{s.} -- . . . .

\textbf{pozéwasédé} \textit{s.} -- gado, boi, vaca, touro.

\textbf{pozéwasété zapu} \textit{s.} -- bezerro, bezerra.

\textbf{pozéwasété nhimiwa'ra} \textit{s.} -- curral.

\textbf{pone} \textit{s.} -- veado mateiro.

\textbf{pone'ẽbö} \textit{s.} -- cabra, bode, cabrito.

\textbf{pone'ẽböire aibö} \textit{s.} -- bode.

\textbf{pone'ẽböire pi'õ} \textit{s.} -- cabra.

\textbf{pone'ẽböre} \textit{s.} -- bode, cabra.

\textbf{pone'ẽnhimizawi} \textit{s.} -- ovelha.

\textbf{pone'ẽre} \textit{s.} -- veadinho, cabrito.

\textbf{po'o} \textit{v.} -- \textbf{po} dividir, quebrar.

\textbf{po'o} \textit{s.} -- \textbf{po} taquara.

\textbf{po'ore} \textit{s.} -- flecha de ponta larga.

\textbf{popaihö} \textit{s.} -- nome de árvore.

\textbf{popanone} \textit{s.} -- foice.

\textbf{popara} \textit{s.} -- enfeite da perna.

\textbf{popo} \textit{v.} -- tremer.

\textbf{popo} \textit{s.} -- tremor; \textbf{ti'ai popo} tremor de terra, terremoto.

\textbf{po're} \textit{s.} -- orelha, ouvido.

\textbf{poré} \textit{adv.} -- em todo o lugar, geral.

\textbf{po'reza'õno} \textit{s.} -- girino; clã.

\textbf{po're zapu} \textit{s.} -- -- furo da orelha.

\textbf{po're za'ru} \textit{s.} -- brinco.

\textbf{po'rezé} \textit{s.} -- otite, dor de ouvido.

\textbf{po'rehaimrami} \textit{vmp. du. pl.} -- esquecer.

\textbf{po'reha'ö} \textit{s. sing.} -- = \textbf{dapo'rehaimrami} \textit{v. du.} -- esquecer, esquecer-se de.

\textbf{po're'õ} \textit{adj. comp.} -- desobediente.

\textbf{po'repe} \textit{adj. comp.} -- obediente.

\textbf{po're po} \textit{s.} -- orelha comprida; \textbf{awaru po'repo} burro.

\textbf{po'repo} \textit{s.} -- burro, jumento.

\textbf{po'reptõ} \textit{adj. comp.} -- \textbf{ĩpo'reptõ} surdo.

\textbf{po'repu} \textit{v.} -- avisar, recordar.

\textbf{po'repuru} \textit{v. sing.} -- lembrar de pessoa.

\textbf{po'repu'u} \textit{v du. pl.} -- avisar, recordar.

\textbf{po'resimrami} \textit{v. du.} -- esquecer.

\textbf{po'resi'öri} \textit{v. sing.} -- \textit{v. du./pl.} -- esquecer.

\textbf{po'rewaza} \textit{vmp.} -- convidar.

\textbf{po're wairĩ} \textit{} -- torcer as orelhas.

\textbf{po'rewa'u} \textit{s.} -- brinco, palito da orelha.

\textbf{põrĩ} \textit{v.} -- ventilar, soprar do vento.

\textbf{poto} \textit{v} -- \textbf{podo} criar, procriar, desenhar, fazer projetos.

\textbf{potozé} \textit{s.} -- criação, sexo.

\textbf{poto hawi} \textit{s./posp.} -- \textbf{ĩpoto hawĩ} congênito.

\textbf{poto na'rada} \textit{s./v.} -- conceber, engravidar.

\textbf{poto na'rada} \textit{s.} -- concepção, gravidez.

\textbf{poto'wa} \textit{s.} -- criador, Deus.

\textbf{powawẽ} \textit{s.} -- gado, boi, vaca.

\textbf{powawẽ aibö} \textit{s.} -- boi, touro.

\textbf{powawẽ pi'õ} \textit{s.} -- vaca.

\textbf{powawẽ zapu} \textit{s.} -- bezerro, vitelo.

\textbf{powawẽ dezérépa} \textit{s.} -- búfalo.

\textbf{powawẽ höiwa'u} \textit{s.} -- leite de vaca.

\textbf{powawẽ ma wa'ra} \textit{s./posp.} -- curral.

\textbf{powawẽ umhöbö} \textit{s.} -- búfalo.

\textbf{pra!} \textit{interj. masc.} -- exclamação de surpresa.

\textbf{prã} \textit{adv./adj.} -- menos, quase.

\textbf{pra} \textit{v. sing.} = \textbf{praba} -- correr, avançar; \textbf{du, te pra} o fogo avança.

\textbf{praba} \textit{v.} -- correr, avançar.

\textbf{praba} \textit{v.} -- dança de passo.

\textbf{prabare} \textit{s.} -- \textbf{ĩprabare} zipper.

\textbf{prã ti} -- \textbf{ĩprã} mais ou menos.

\textbf{pré} \textit{adj.} -- \textbf{ĩpré} vermelho.

\textbf{pré'a} \textit{adj. comp.} -- amarelo.

\textbf{pré'a} \textit{s.} -- brasa.

\textbf{pré'a} \textit{s.} -- brasileiro.

\textbf{pré'a nhib'a'uwẽ} \textit{} -- cidadão, brasileiro.

\textbf{pré'anhib'ri} \textit{} -- Brasília.

\textbf{pré'anhib'ri'ahö} \textit{} -- Brasília.

\textbf{pré'anhibro} \textit{} -- Brasil.

\textbf{pré'are} \textit{s.} -- centelha, faísca.

\textbf{prédu} \textit{adj.} -- \textbf{ĩprédu} adulto.

\textbf{prédu} \textit{v.} -- crescer, tornar-se adulto, amadurecer; \textbf{ma tõ tiprédu} ficou adulto.

\textbf{prẽdubzé} \textit{s.} -- amadurecimento, madureza.

\textbf{pré'é} \textit{v.} -- bater, surrar, chicotear.

\textbf{prẽ ro'o} -- vermelho forte, rubro.

\textbf{pré'ubumro} \textit{s.} -- \textbf{ĩpré'ubumro} reunião dos vermelhos.omunismo.

\textbf{pré'uzé} \textit{adj. comp.} -- \textbf{iprẽ'uzé} colorido.

\textbf{pro} \textit{adj.} -- derivado de, caldo de, saĩado de, pó de; \textbf{wede pro serragem}; \textbf{dazadaipro} saliva, cuspe, escarro.

\textbf{pro} \textit{s.} -- pó queimado, fuligem.

\textbf{prore} \textit{s.} -- \textbf{ĩprore} lápis.

\textbf{prorodo} \textit{s.} -- coruja.

\textbf{prorotore} \textit{s.} -- coruja pequena.

\textbf{pru} \textit{v.} -- \textbf{ĩpru} quebrar.

\textbf{pu} \textit{v. sing.} = \textbf{pusi} \textit{v. du.} -- sair; \textbf{sena ĩpu} concretizar; \textbf{ĩsarina pu} consequência; \textbf{si'uihöna ĩpu} tirar a sorte.

\textbf{pu} \textit{s.} -- lesão; \textbf{hĩpu} \textit{s.} -- lesão da perna.

\textbf{pu} \textit{s.} -- lago, mar.

\textbf{pupu} \textit{s.} -- barulho da água, ruído do terremoto, barulho.

\textbf{pupu'u} \textit{s.} = \textbf{pupu} -- barulho, ruído.

\textbf{puru} \textit{v. sing.} -- derramar, jorrar, furar.

\textbf{pusi} \textit{v. du.} -- sair.

\textbf{pusizé} \textit{s.} -- saída.

\textbf{pu'u} \textit{s.} = \textbf{pu} -- lago, mar.

\textbf{pu'u} \textit{v. du./pl.} -- riscar as pernas.

\textbf{pu'u} \textit{v. du./pl.} = \textbf{puru} -- furar, derramar, jorrar.

\textbf{pu'ure} \textit{s.} -- lagoa pequena.

\textbf{pu'uwawẽ} \textit{s.} -- mar limitado, lago grande.



\textbf{R}.



\textbf{ra} \textit{s.} -- lixeira.

\textbf{'ra} \textit{v.} -- \textbf{ĩ'ra} fazer ruído, fazer barulho; \textbf{para'ra} ruído dos pés.

\textbf{'ra} \textit{s.} -- filho, filha; \textbf{da'rawapté} \textit{s.} sobrinho, sobrinha.

\textbf{'ra} \textit{v.} -- criar filhos, procriar.

\textbf{rã} \textit{adj.} -- \textbf{ĩrã} \textbf{darã} branco, claro, limpo.

\textbf{rã} \textit{v.} -- tornar branco, caiar; \textbf{rã da} caiar.

\textbf{'rã} \textit{s.} -- cabeça.

\textbf{'rã} \textit{s.} -- \textbf{ĩ'rã} = \textbf{rob'rã} fruto, fruta, cacho; \textbf{ĩ'rãi'ré} fruto seco, fruto maduro.

\textbf{'rã} \textit{v.} -- dar fruto; \textit{ma ti'rã za'ra} eles deram fruto.

\textbf{'rã} \textit{adj.} -- \textbf{ĩ'rã} preto; \textbf{bödödi'rã} asfalto.

\textbf{'rada} \textit{s.} -- arara vermelha (\textit{Ara chloroptera}).

\textbf{'rada uzé} \textit{s.} -- arara vermelha (\textit{Ara chloroptera}).

\textbf{'rada} \textit{s.} -- antepassados, velho, virilha; \textbf{da'rada aĩbö} avô; \textbf{da'rada pii'õ} avó.

\textbf{'rada} \textit{adj.} -- \textbf{ĩ'rada} primeiro, antigo, velho.

\textbf{'rada} \textit{posp.} -- \textbf{ĩ'rada} antes, primeiro.

\textbf{'ra dadub 're} \textit{s./posp.} -- feto.

\textbf{'rãdö} \textit{adj.} -- \textbf{ĩ'rãdö} preto, escuro.

\textbf{rãdö'ö} \textit{adj. c.} -- \textbf{ĩ'rãdö'ö} preto escuro.

\textbf{'rãzaza} \textit{s.} -- chapéu, chapéu com aba larga.

\textbf{'rãzazazamo} \textit{s.} -- boné.

\textbf{'rãzazatede} \textit{s.} -- capacete.

\textbf{'rãza'idi} \textit{vmp.} -- cortar cabelo no meio, fazer tonsura.

\textbf{'rãza'iti} \textit{vmp. c.} -- cortar cabelo no meio, fazer tonsura.

\textbf{'ra za'õno} \textit{s.} -- parte traseira da cabeça.

\textbf{'rãza'õno} \textit{s.} -- recordar.

\textbf{'razarõno} \textit{vmp.} -- levantar a cabeça, soberba.

\textbf{'razé} \textit{s.} -- útero.

\textbf{'rãzé pire} \textit{s.} -- dor de cabeça, cefaléia.

\textbf{'rãzö'rã} \textit{s.} -- marimbondo preto.

\textbf{'razöri} \textit{v. sing.} -- \textbf{ro'razöri} = \textbf{ro'rahöri} \textit{v. du.} passar por, atravessar.

\textbf{'rãzöri} \textit{v. sing.} = \textbf{'rãhöri} \textit{v. du.} colher.

\textbf{'rãhiwanhana} \textit{s.} -- cérebro.

\textbf{'rãhiwanhanazapré} \textit{s.} -- derrame cerebral.

\textbf{'rahöri} \textit{v. du./pl.} = \textbf{ro'rahöri} passar por, atravessar.

\textbf{'rahöri} \textit{v. du.} -- colher.

\textbf{'rãihi} \textit{s.} -- crânio, caveira.

\textbf{'rãihiwanhana} \textit{s.} -- cérebro.

\textbf{'rãihö} \textit{adj. comp.} -- \textbf{ĩ'rãihö} alto, elevado, grande.

\textbf{'rãihö zahu} \textit{s.} -- \textbf{ĩ'raĩhözahu} corda de reserva no arco.

\textbf{'rãime} \textit{s.} -- canto antes da caça.

\textbf{'rãime} \textit{vmp. sing.} -- = \textbf{da'rãiwabzuri} \textit{vmp. du.} bater na cabeça, esbofetear.

\textbf{'rãi'ré} \textit{s.} -- \textbf{ĩ'rãi'ré} fruto seco, fruto maduro.

\textbf{'rãi'wa'rudu} \textit{s.} -- preocupação.

\textbf{'rãi'wa'rutu} \textit{vmp. c.} = \textbf{da'rãi'wa'rudu} preocupar-se.

\textbf{'rãiwi'a} \textit{s.} -- careca, calvo.

\textbf{'rame} \textit{s. infa.} -- mãe!.

\textbf{'rami} \textit{v.} -- queimar, cremar.

\textbf{'rãmi} \textit{v.} -- = \textbf{ĩwa'rãmi} assustar gente, discutir, disputar, responder, retrucar, brigar, contrariar, contestar.

\textbf{'ra nhimiza'rese'õre} \textit{vmp. dim.} -- caducar, bobinho.

\textbf{'rãnhipai'wa} \textit{s.} -- \textbf{ĩ'rãnhipai'wa} superior.

\textbf{'rã'õno} \textit{vmp.} -- unir-se, invocar, colher, dar nó, recolher.

\textbf{'rã'õno} \textit{s.} -- invocação, reunião, assembléia.

\textbf{'rãpari} \textit{vmp.} -- rapar a cabeça.

\textbf{'rãpari} \textit{adj. comp.} -- \textbf{ĩ'rãpari} calvo, cabeça rapada.

\textbf{'rãpo} \textit{adj. comp.} -- cabeça inclinada, cabisbaixo.

\textbf{'rãpo} \textit{vmp.} -- inclinar-se, abaixar a cabeça.

\textbf{'rãpré} \textit{adj. comp.} -- cabeça vermelha.

\textbf{'rãrã} \textit{v.} -- fazer barulho, trovoar.

\textbf{'rãra} \textit{s.} -- barulho; = \textbf{tãirãrã} trovão.

\textbf{'rãrã} \textit{v.} -- \textbf{ĩrãrã} = \textbf{robrãrã} fazer barulho.

\textbf{'rãrã'ã} \textit{v} -- \textbf{ĩrãrã} fazer barulho.

\textbf{'rare} \textit{s.} -- lixeira.

\textbf{'rare} \textit{adj. dim.} -- \textbf{ĩ'rare} magro, pequeno.

\textbf{'rarepa} \textit{s.} -- raiz de lixeira.

\textbf{'rã'ru} \textit{adj. comp.} -- \textbf{ĩ'rã'ru} cabelo ondulado, cabelo crespo, pixaim.

\textbf{'rata} \textit{posp.} -- \textbf{ĩ'rata} perto de, antes de.

\textbf{'ratamama} \textit{s.} -- bisavô, bisavó.

\textbf{'ratare} \textit{s.} -- \textbf{ĩ'ratare} coisas velhas, antiguidades.

\textbf{'rãtẽ} \textit{adj. comp.} -- coxo, manco, paralítico.

\textbf{'rãtẽ} \textit{vmp.} -- mancar.

\textbf{'rãtede} \textit{adj. comp.} -- cabeçudo, cabeça dura.

\textbf{'rãti} \textit{s.} -- cabeça molhada, batismo.

\textbf{'rãti} \textit{vmp.} -- molhar a cabeça, batizar.

\textbf{'rãti} \textit{s.} -- formiga.

\textbf{'rãti'i'õ} \textit{adj. comp.} -- não batizado, pagão.

\textbf{'rãti'i ré} \textit{adj./posp.} -- batizado, cristão.

\textbf{'rãti nhimhi} \textit{s.} -- casulo da formiga, mistério da formiga.

\textbf{'rãto} \textit{s.} -- \textbf{ĩ'rãto} morrinho no varjão, feitiço.

\textbf{'rãtõ} \textit{vmp.} -- \textbf{ĩ'rãtõ} cessar, concluir.

\textbf{'rasi} \textit{s.} -- \textbf{ĩ'ra} barulho dos pés, barulho do andar.

\textbf{'rasi} \textit{v.} -- \textbf{ĩ'ra} fazer barulho com os pés, fazer ruído.

\textbf{'rãsitõ} \textit{s.} -- feixe de folhas.

\textbf{'rasudu} \textit{s.} -- saracura.

\textbf{'rãsudu} \textit{s.} -- \textbf{ĩ'rãsudu} fim, término, ponta.

\textbf{'rãsatu} \textit{v} -- \textbf{'rãsudu} terminar, findar, encerrar.

\textbf{'rãsutuzé} \textit{s.} -- \textbf{ĩ'rãsutuzé} fim, término.

\textbf{'rã'upari'wa} \textit{s.} -- padrinho, madrinha.

\textbf{'rawa} \textit{s.} -- paca.

\textbf{'ra'wa} \textit{s.} -- pai, genitor.

\textbf{'rawapté} \textit{s.} -- sobrinho.

\textbf{ré} \textit{s.} -- resina.

\textbf{ré} \textit{v.} -- tremer, mexer-se, chocalhar.

\textbf{ré} \textit{posp.} -- \textbf{ĩ'ré} cheio de, com, contendo.

\textbf{rẽ} \textit{v. sing.} = \textbf{rẽme} -- deixar, abandonar, ceder; \textbf{õ hã, ma tirẽ} -- ele deixou, abandonou, cedeu.

\textbf{-re} \textit{suf. dim.} -- sufixo diminutivo.

\textbf{'rẽ} \textit{v. sing.} = \textbf{'rẽne} -- comer; \textbf{õ hã, ma ti'rẽ} -- ele comeu.

\textbf{'rẽ} \textit{s.} -- periquito (genérico); 

\textbf{'rẽre} \textit{s.} -- periquito.

\textbf{'re} \textit{s.} -- sanguessuga.

\textbf{'re} \textit{s.} -- -- testículo, vagina, vulva.

\textbf{'re} \textit{s.} -- \textbf{ĩ're} lugar, cavidade, planta, buraco, furo, carroceria, espaço, interior, conteúdo, gruta, ovo, cova.

\textbf{'re} \textit{ posp.} -- \textbf{ĩ're} em, dentro de.

\textbf{'re} \textit{adv.} -- sempre; \textbf{'re . . . mono} sempre . . . em todo o lugar.

\textbf{'ré} \textit{adj.} -- \textbf{ĩ'ré} seco.

\textbf{'rébé} \textit{s.} -- pauzinho para acender fogo.

\textbf{'rébéme} \textit{vmp.} -- acender, fogo com pauzinho.

\textbf{'reb ĩhöimana} \textit{adj./s.} -- \textbf{ĩ'reb ĩhöimana} conteúdo.

\textbf{'reb'ré'é'u} \textit{s.} -- urubu-rei.

\textbf{'redumhö} \textit{s.} -- escroto.

\textbf{'rezani} \textit{vmp.} -- \textbf{ĩ'rezani} castrar, capar.

\textbf{'rézé} \textit{s.} = \textbf{rob'rézé} -- lugar seco, secador.

\textbf{'rehi} \textit{s.} -- \textbf{ĩ'rehi} -- espião.

\textbf{'rehö} \textit{s.} -- \textbf{ĩ'rehö} côncavo, profundo; \textbf{ö 'rehö} -- baía.

\textbf{'Re Ĩ'höimana U'ösi Mono} \textit{adv./v./adv./adv.} -- que vive sempre, Deus.

\textbf{rẽme} \textit{v.} -- deixar, abandonar, ceder.

\textbf{rẽmezé} \textit{s.} -- abandono.

\textbf{rẽme'õ} \textit{s.} -- \textbf{ĩ'rẽme'õ} que não abandona, constância.

\textbf{'remhã} \textit{posp.} -- \textbf{ĩ'remhã} -- estando dentro, contendo.

\textbf{'rẽne} \textit{v. sing.} = \textbf{si} \textit{v. du.} -- comer.

\textbf{'renhamri} \textit{s.} -- pequena esteira.

\textbf{'renhipré} \textit{s.} -- periquito vermelho.

\textbf{'rẽnhö} \textit{s.} -- fruta do mato.

\textbf{repé'é} \textit{adv.} -- ansioso.

\textbf{'repu'u} \textit{v.} -- furar.

\textbf{rẽrẽ} \textit{v.} -- tremer.

\textbf{rẽrẽ} \textit{s.} -- choque.

\textbf{'rẽre} \textit{s.} -- periquito; \textbf{'rẽ} -- periquito (genérico).

\textbf{'rẽrezareba} \textit{s.} -- penas amarelas de periquito.

\textbf{rere'e} \textit{v. pl.} = \textbf{dawaptã} \textit{v. du.} -- cair, nascer.

\textbf{rési} \textit{v} -- \textbf{ré} tremer, mexer-se, chocalhar.

\textbf{rési} \textit{adj.} -- \textbf{ĩré} agitado.

\textbf{rési'õ} \textit{adj. comp.} -- \textbf{ĩrési'õ} -- imóvel, parado.

\textbf{'resito} \textit{s.} -- \textbf{ĩ'resito} -- ônibus.

\textbf{rési'wa} \textit{s.} -- motor.

\textbf{rési'wa upsibizé} \textit{s.} -- capota, capu.

\textbf{'resu} \textit{s.} -- palha de coqueiro, taquara de bambu.

\textbf{'resui'ré} \textit{s.} -- palha seca.

\textbf{'rẽwawẽ} \textit{s. aum.} -- periquito grande, periquitão (\textit{Psittacara leucophthalmus}).

\textbf{'ri} \textit{s.} -- casa, casa xavante, cabana, choupona, choça.

\textbf{'ri} \textit{v.} -- construir casa.

\textbf{'rĩ} \textit{v.} -- arrancar, cortar; \textbf{õ hã, ma ti'rĩ} ele corta, arranca.

\textbf{'ri ahö} \textit{s.} -- muitas casas.

\textbf{'ri'ahö} \textit{s.} -- povoado, cidade.

\textbf{'ri'ahö morĩ'rada} \textit{s.} -- primeira cidade, capital.

\textbf{'ri'ahötede'wa} \textit{s.} -- dono da cidade, prefeito.

\textbf{'ribi} \textit{v.} -- nadar.

\textbf{'ridawa} \textit{s.} -- porta.

\textbf{'ridawa nhitobzé} \textit{s.} -- chave.

\textbf{'ridi} \textit{s.} -- . . . \textbf{robhni'ridi} brecha.

\textbf{'ridi} \textit{s.} -- gafanhoto.

\textbf{'rizadazöri} \textit{s.} -- janela.

\textbf{'rizahöbö} \textit{s.} -- varanda.

\textbf{'rizapu} \textit{s.} -- buraco da casa, janela.

\textbf{'ri'manharĩ'rada} \textit{s.} -- castelo.

\textbf{'rina'rada} \textit{s.} -- início da casa, muro, muralha, parede.

\textbf{'rinhito} \textit{s.} -- casa fechada, claustro.

\textbf{'ri nhitowa} \textit{s.} -- casa aberta, porta, portão.

\textbf{'ri nho'õmo} \textit{s.} = \textbf{'ri nho'u} -- esteio do meio da casa.

\textbf{'rinho'õmo} \textit{s.} -- \textbf{'rinho'u} muitas casas, metrópole; \textbf{'ri nho'õmore} \textit{s.} povoado.

\textbf{'ri nho'u} \textit{s.} -- esteio do meio da casa.

\textbf{'rinho'u} \textit{s.} -- muitas casas.idade grande, metrópole.

\textbf{'rinho're} \textit{s.} -- páteo.

\textbf{rĩni} \textit{v.} -- procurar.

\textbf{rĩni} \textit{adj.} -- acordado.

\textbf{'ri'raĩhö} \textit{s.} -- casa alta, prédio.

\textbf{'ri'rewa'õno} \textit{s.} -- parte dentro da casa, quarto, compartimento.

\textbf{'ri'ribi} \textit{s.} -- grilo.

\textbf{'ri romnhorézé} \textit{s.} -- classe.

\textbf{'ritede} \textit{s.} -- fortificação, o forte.

\textbf{'ritéi'wa} \textit{s.} -- dono de casa nova, moço inciado à vida adulta.

\textbf{'riti} \textit{s.} = \textbf{'ridi} -- gafanhoto.

\textbf{'rĩtĩ} \textit{v} -- -- procurar.

\textbf{'rĩtĩ pe} \textit{adj./adj.} -- olhar fixo, curioso.

\textbf{'ritiré} \textit{s.} -- gafanhoto pequeno.

\textbf{'rito} \textit{s.} -- mangaba.

\textbf{'ritu} \textit{s.} -- casa abandonada, tapera.

\textbf{'rituwawẽ} \textit{s. aum.} -- casas maiores; Aldeiona.

\textbf{'ri uparizé} \textit{s.} -- apoio da casa, coluna.

\textbf{'ri upsibizé} \textit{s.} -- coberta da casa, telhado.

\textbf{'ri'wa} \textit{s.} -- construtor da casa.

\textbf{'ri wa'õtõre} \textit{s.} -- partezinha da casa, cela.

\textbf{'ri'waré} \textit{s.} -- torre.

\textbf{ro} \textit{s.} -- \textbf{ĩro} luz, energia elétrica.

\textbf{ro} \textit{s.} -- lugar, lugar de moradia, lugar de nascimento, terra nativa.

\textbf{ro-} \textit{pref.} = \textbf{rob-}, = \textbf{rom-} -- sentido amplo, intensificado, indefinido.

\textbf{'ro} \textit{s. aa.} -- irmão.

\textbf{'ro} \textit{s.} -- mau cheiro.

\textbf{'ro} \textit{adj.} -- \textbf{ĩ'ro} mal cheiroso.

\textbf{ro} \textit{adj.} -- áspero, peludo; \textbf{ĩhöiba roi wawẽ di} -- corpo muito peludo.

\textbf{rõ} \textit{s.} -- estojo peniano.

\textbf{rob-} \textit{pref.} = \textbf{ro-} \textbf{rom-} -- sentido amplo, intensificado, indefinido.

\textbf{robaba} \textit{adj.} -- \textbf{ĩrobaba} vazio, baldio.

\textbf{robdi} \textit{v.} -- molhar, borrifar, aspergir.

\textbf{robdi'i} \textit{v} -- \textbf{robdi} molhar, borrifar, aspergir.

\textbf{robduri} \textit{s.} -- veículo.

\textbf{robduriza'éré} \textit{s.} -- bicicleta.

\textbf{robduriza'éré 'mapraba'wa} \textit{s.} -- ciclista.

\textbf{robduriza'éré na dasi'wapé} \textit{s./posp.} -- ciclismo.

\textbf{robduir zapo} \textit{s.} -- camioneta.

\textbf{robdurizé} \textit{s.} -- combustível.

\textbf{robdurihöiba} \textit{s.} -- chassi.

\textbf{robduri nhimizasizé} \textit{s.} -- garagem.

\textbf{robduri para} \textit{s.} -- pneu.

\textbf{robduripré} \textit{s.} -- trator.

\textbf{robduri'rãihö} \textit{s.} -- caminhão.

\textbf{robduri're} \textit{s.} -- carroceria.

\textbf{robduri'una} \textit{s.} -- carreta.

\textbf{robduri wa'õtõ ahö} \textit{s.} -- trem de ferro.

\textbf{robduri wẽtẽnhamri} \textit{s.} -- trator de esteira.

\textbf{robzabu} \textit{v.} -- ver, meditar, espiar.

\textbf{robzadaze} \textit{s.} -- odor, cheiro do lugar.

\textbf{robzadamrimi} \textit{s.} -- cheirar.

\textbf{robzada'ré pibuzé} \textit{s.} -- baliza.

\textbf{robzahi} \textit{s.} -- região selvagem.

\textbf{robzanhamri} \textit{s.} -- conversa, colóquio, bate-papo.

\textbf{robzanhamri} \textit{v.} -- conversar, falar.

\textbf{robzanhamrize} \textit{adj. comp.} -- comunicativo, conversador.

\textbf{robzapodo} \textit{s.} -- círculo, circo.

\textbf{robzapoto na dato} \textit{s./posp.} -- circo.

\textbf{robzapru} \textit{s.} -- poeira.

\textbf{robza'ra} \textit{vmp. pl.} = \textbf{robnomri} \textit{v. du.} -- amontoar, arrumar.

\textbf{robza'rã} \textit{s.} -- escuridão, madrugada.

\textbf{robza'rã} \textit{v.} -- escurecer, madrugar.

\textbf{robza'rãzé} \textit{s.} -- prateleira.

\textbf{robza'rata nhipese'wa} \textit{s.} -- computador.

\textbf{robza'rata nhipese'wa} \textit{} -- inventor, gênio.

\textbf{roza'ra wẽ} \textit{v./adv.} -- coordenar.

\textbf{roza'ra wẽ} \textit{s./adv.} -- coordenação.

\textbf{rozari} \textit{s.} -- animal doméstico.

\textbf{robza'ru bö} \textit{s.} -- qualquer lugar vazio, qualquer lugar desocupado.

\textbf{robza'ru'õ} \textit{s.} -- onde falta tudo.

\textbf{robzasõmrizé} \textit{s.} -- cabide.

\textbf{robze s.} -- alegria, felicidade, contentamento, agrado.

\textbf{robze} \textit{adj.} -- alegre, contente, feliz, agradável.

\textbf{robze} \textit{v.} -- contentar, agradar, consolar.

\textbf{robzé} \textit{s.} -- comoção.

\textbf{robzeire} \textit{s.} -- caramelo, bala.

\textbf{robzei'õ} \textit{adj. comp.} -- \textbf{ĩrobzei'õ} triste, infeliz.

\textbf{robzépada} \textit{v.} -- sofrer.

\textbf{robzépada} \textit{s.} -- paixão.

\textbf{robzépatazé} \textit{s.} -- paixão, sofrimento.

\textbf{robzö} \textit{s.} -- semente, bicho dentro da água, bicho dentro da terra.

\textbf{robzöri} \textit{v. sing.} = \textbf{romhöri} \textit{v. du.} -- defender.

\textbf{robzuri} \textit{v.} -- semear, plantar.

\textbf{robzuri'wa} \textit{s.} -- semeador.

\textbf{robna'rada} \textit{s.} -- começo de tudo, começo da criação.

\textbf{robnharĩ} \textit{v.} -- afirmar.

\textbf{robnhemezé} \textit{s.} -- armário.

\textbf{robnhi'ridi} \textit{s.} -- brecha.

\textbf{rob nhisi} \textit{s.} -- nome do lugar.

\textbf{rob nho'u} \textit{s.} -- multidão de lugares.

\textbf{ro bö} \textit{s./posp.} -- em toda a parte.

\textbf{robödö} \textit{v.} -- caminhar arrastando os pés, rastejar.

\textbf{robösi} \textit{s.} = \textbf{dabõsi} -- mentira, calúnia, potoca.

\textbf{robösire} \textit{s.} -- boato, falar do outro.

\textbf{rob'ra} \textit{s.} -- barulho.

\textbf{robra} \textit{adj.} -- \textbf{ĩrobra} escuro.

\textbf{robra} \textit{v.} -- escurecer; \textbf{ma tirobra} escureceu; \textbf{ma tiwi tirobra} ele desmaiou.

\textbf{rob'rã} \textit{s.} -- cacho, fruto, fruta.

\textbf{rob'rãza'õno} \textit{v.} -- . . . .

\textbf{rob'rãza'õtõ 'ratare} \textit{s.} -- fábula.

\textbf{rob'rãzé} \textit{s.} -- novembro.

\textbf{rob'rãizöri} \textit{v. sing.} -- \textbf{rob'rãihöri} \textit{v. du.} -- ceifar, colher.

\textbf{rob'rãihö} \textit{s.} -- casca de fruta.

\textbf{rob'rãihö} \textit{s.} -- lugar elevado.

\textbf{rob'rãihöri} \textit{v. du./pl.} -- ceifar, colher.

\textbf{rob'rãihöri'wa} \textit{s.} -- ceifador, colhedor.

\textbf{rob'rãiwa'u} \textit{s.} -- vinho de frutas.

\textbf{rob'rãiwiwede} \textit{s.} -- coqueiro.

\textbf{rob'rãpreçé} \textit{s.} -- maçã.

\textbf{robrãrã} \textit{s.} -- barulho.

\textbf{robrã'ãze} \textit{s.} -- uva.

\textbf{rob'rã 'rãdö} \textit{s.} -- beringela.

\textbf{rob'rãsudu} \textit{s.} -- fim de tudo, fim do mundo.

\textbf{rob'rãsutu} \textit{v} -- \textbf{rob'rãsudu} terminar, acabar.

\textbf{rob're} \textit{v.} -- semear.

\textbf{rob're} \textit{s.} -- lugar espaçoso, mansão.

\textbf{rob'ré} \textit{s.} -- terra seca, lugar seco, seca, ilha, continente.

\textbf{rob'réza'idi} \textit{s.} -- ilha.

\textbf{rob'rezé} \textit{s.} -- semeadeira.

\textbf{rob'rézé} \textit{} -- lugar seco, secador.

\textbf{rob'rehö} \textit{s.} -- buraco profundo, abismo.

\textbf{robrésizé} \textit{s.} -- política.

\textbf{robrési'õ} \textit{s.} -- calma.

\textbf{robrési'wa} \textit{s.} -- político.

\textbf{rob're'wa} \textit{s.} -- semeador.

\textbf{rob'ro} \textit{s.} -- fedor, mau cheiro, carniça, lugar de mau cheiro.

\textbf{robro'o} \textit{s.} -- tempo de seca, agosto.

\textbf{rob'ru} \textit{v.} -- mandar, chutar; \textit{dama ĩrob'ru} -- banir.

\textbf{rob'rui'wa} \textit{s.} -- dono, senhor, comandante, governador.

\textbf{rob'ruiwapari} \textit{v.} -- querer algo.

\textbf{rob'ruiwapari} \textit{s.} -- desgosto, raiva de tudo.

\textbf{robrudu} \textit{s.} -- zoada.

\textbf{rob'ruru} \textit{s.} -- barulho.

\textbf{robta} \textit{s.} -- baque, choque.

\textbf{robta'a} \textit{s.} -- \textbf{robta} baque, choque.

\textbf{robtede} \textit{v.} -- firmar, fixar.

\textbf{robtédé} \textit{v.} -- agarrar.

\textbf{robti} \textit{v.} -- fixar, apertar, pressionar.

\textbf{robtö} \textit{s.} -- colina, monte, morro.

\textbf{robsãmri} \textit{v.} -- perceber, constatar.

\textbf{robsãmrinhorõ} \textit{s.} -- nervo.

\textbf{robsipra} \textit{s.} -- descida.

\textbf{robsi'utõrĩ} \textit{v.} -- acabar, terminar.

\textbf{rob u} \textit{s./posp.} -- para o lugar.

\textbf{rob'u} \textit{adv.} -- fora.

\textbf{rob'uze} \textit{v.} -- alegrar, louvar.

\textbf{rob'uzéizé} \textit{s.} -- gozo, alegria.

\textbf{rob'ui'éré} \textit{v.} -- escrever, carta.

\textbf{rob'uipra} \textit{s.} -- compra, comércio.

\textbf{rob'uipra} \textit{v.} -- comprar, comercializar; \textit{pi'uriwi rob'uipra 'manharĩ} fazer contrabando.

\textbf{rob'uiprazé} \textit{s.} -- dinheiro.

\textbf{rob'uipira na romhuri'wa} \textit{s./posp.} -- comerciante.

\textbf{rob'uiprarésĩ'wa} \textit{s.} -- mercador, comerciante.

\textbf{rob'uipra'ubumro} \textit{s.} -- comércio.

\textbf{rob'uipra'wa} \textit{s.} -- comprador.

\textbf{rob'uiwẽ} \textit{vmp.} -- brilhar, clarear, iluminar.

\textbf{rob'uiwẽ'wa} \textit{s.} -- quem faz luz, luzeiro.

\textbf{rob'umnhasi} \textit{v.} -- confiar.

\textbf{rob'upa} \textit{v.} -- afastar-se, errar.

\textbf{rob'upsõzé} \textit{s.} -- sabão.

\textbf{rob'upsõzé zadaze} \textit{s.} -- sabonete.

\textbf{rob'utãtã} \textit{s.} -- trovoada.

\textbf{rob'uwaimramizé} \textit{s.} -- bússola.

\textbf{rob'uwe} \textit{v.} -- encontrar, aproximar.

\textbf{rob'wasari} \textit{v.} -- carregar.

\textbf{rom-} \textit{pref.} = \textit{ro-} = \textit{rob-} -- sentido amplo, intensificado, indefinido.

\textbf{ro'madö'ö} \textit{v.} -- ver, mostrar.

\textbf{ro'mahörö} \textit{} -- chamar, avisar, comunicar.

\textbf{ro'manharĩ} \textit{v.} -- fazer, agir, cometer, amaldiçoar, arrancar, estragar, condenar.

\textbf{ro'manharĩzé} \textit{s.} = \textit{dama ro'manharĩzé} -- condenação.

\textbf{ro'manharĩ 'ru} \textit{} -- concretizar.

\textbf{ro'manharĩ'wa} \textit{s.} = \textbf{daro ro'manharĩ'wa} -- colono.

\textbf{ro'manharĩ waihu'uzé} \textit{v./s.} -- ciência.

\textbf{ro'manharĩ wẽ} \textit{v./adv.} -- beneficiar.

\textbf{ro'manharĩwẽ'wa} \textit{s.} -- benfeitor.

\textbf{romduri} \textit{s.} = \textbf{robduri} -- carro, veículo.

\textbf{romhai'utõ} \textit{s.} -- fim (lugar ou tempo).

\textbf{romha'ö} \textit{v.} -- avançar, conquistar.

\textbf{romharé} \textit{s.} -- paz, calma, tranqüilidade.

\textbf{romharé romri} \textit{s./v.} -- fazer as pazes.onciliar.

\textbf{romhö} \textit{s.} -- couro.

\textbf{romhö} \textit{v.} -- lançar.

\textbf{romhö} \textit{adj.} -- longe.

\textbf{romhõ} \textit{v.} -- criar, renovar.

\textbf{romhöbö} \textit{s.} -- terra plana, planície, bacia.

\textbf{romhöiba} \textit{v.} -- existir, viver.

\textbf{romhöibazé} \textit{s.} -- molécula.

\textbf{romhõibarĩ} \textit{s.} -- imagem.

\textbf{romhöimana} \textit{v.} -- viver, existir.

\textbf{romhöimo} \textit{s.} -- norte.

\textbf{romhöiwa'u} \textit{s.} -- leite.

\textbf{romhöiwa'u'ẽne} \textit{s.} -- queijo.

\textbf{romhöiwa'u'uwa} \textit{s.} -- manteiga.

\textbf{romhö'madö} \textit{s. abr.} -- TV.

\textbf{romhö'madö'özé} \textit{s.} -- televisor.

\textbf{romhö'madö'özuri} \textit{s.} -- televisão (emissora).

\textbf{romhö'madö'öpodo} \textit{s.} -- filme de TV.

\textbf{romhõ na dapré} \textit{adj./posp./v.} -- chicotear.

\textbf{romhõ na dapré} \textit{adj./posp.} -- chicotada.

\textbf{romhõ na ro'madö'özé} \textit{adj./posp.} -- binóculo.

\textbf{romhöbö} \textit{v} = \textbf{romhöbö} -- aplainar terra.

\textbf{romhöpöwawẽ} \textit{s. aum.} -- chapadão.

\textbf{romhöri} \textit{v. du./pl.} -- defender, aplacar, pacificar, fazer as pazes.

\textbf{romhöri'wa} \textit{s.} -- defensor, pacificador.

\textbf{romhörö} \textit{s.} -- barulho, som.

\textbf{romhörözazé} \textit{s.} -- gravador.

\textbf{romhöröze} \textit{s.} -- música.

\textbf{romhörö mramizé} \textit{s.} -- antena de rádio.

\textbf{romhörö'rãihözé} \textit{s.} -- alto-falante.

\textbf{romhõsi} \textit{v} -- \textbf{romhõ} renovar.

\textbf{romhõsi'wa} \textit{s.} -- renovador das coisas.riador.

\textbf{romhu} \textit{s.} -- alegria.

\textbf{romhuri} \textit{v.} -- trabalhar.

\textbf{romhudu} \textit{adj.} -- perto.

\textbf{romhuri} \textit{s.} -- trabalho; \textbf{dasiré romhuri} cooperação.

\textbf{romhuri} \textit{v.} -- trabalhar.

\textbf{romhuri amo} \textit{s.} -- outra semana.

\textbf{romhurizé} \textit{s.} -- lugar de trabalho, oficina, ferramenta; \textbf{dasiré romhurizé} cooperativa.

\textbf{romhurimreme} \textit{s.} -- rádio.

\textbf{romhuri na dasi'wapé} \textit{s./posp.} -- concorrência.

\textbf{romhurina'rada} \textit{s.} -- segunda-feira.

\textbf{romhurina'ratanhana'remhã} \textit{s.} -- terça-feira.

\textbf{romhurinhihödö} \textit{s.} -- carta, escrito, escritura.

\textbf{romhurinhihödö} \textit{vmp.} -- escrever.

\textbf{romhurinhihötö 'maprabazé} \textit{s.} -- correio.

\textbf{romhurinhihötöre} \textit{s.} -- bilhete.

\textbf{romhurinhihötö'wa} \textit{s.} -- alumno, discípulo.

\textbf{romhurinhihötö 'wamarĩ} \textit{s.} -- burocracia.

\textbf{romhurinhipesezé} \textit{s.} -- computador.

\textbf{romhurinhipese'wa} \textit{s.} -- engenheiro.

\textbf{romhuripisutu'wa} \textit{s.} -- ministro.

\textbf{romhuri'rãsudu} \textit{s.} -- sábado.

\textbf{romhuri'rãsutuprã} \textit{s.} -- sexta-feira.

\textbf{romhuri 'ruzahi} \textit{s.} -- compromisso.

\textbf{romhuri 'ruzahi} \textit{s./v.} -- comprometer.

\textbf{romhuri si'uiwa ropisudu} \textit{s.} -- convênio.

\textbf{romhurisiwapto} \textit{s.} -- semana.

\textbf{romhuri ubumrozé} \textit{s.} -- companhia.

\textbf{romhuri'upa} \textit{s.} -- trabalho falho, trabalho malfeito.

\textbf{romhuri'upai'õzé} \textit{s.} -- aparelho de precisão.

\textbf{romhuri'wa} \textit{s.} -- trabalhador, criado, servo; \textbf{dasiré romhuri'wa} cooperador.

\textbf{romhuri wa'ö} \textit{s.} -- valor do trabalho, salário, ordenado.

\textbf{romhuri'wapesere} \textit{s.} -- gilette.

\textbf{romhuriwa'wa} \textit{s.} -- quarta-feira.

\textbf{romhuriduri} \textit{s.} -- quinta-feira.

\textbf{romhutu} \textit{adj. c.} -- \textbf{romhudu} perto.

\textbf{romhuture} \textit{adj. dim.} -- pertinho.

\textbf{romhu'u} \textit{v} -- \textbf{romhu} alegrar-se.

\textbf{romhu'uto} \textit{s.} -- carnaval.

\textbf{romnare} \textit{s.} -- caboclo.

\textbf{rom na rob'rui'wa} \textit{s./posp.} -- chanceler.

\textbf{romnhama} \textit{s.} -- \textbf{robzö} semente.

\textbf{romnhamare} \textit{s.} -- semente pequena; sementinha.

\textbf{romnhi} \textit{s.} -- carne.

\textbf{romnhi zadaze} \textit{s.} -- cheiro de carne.

\textbf{romnhizaro} \textit{s.} -- churrasco.

\textbf{romnhihöpöre} \textit{s.} -- bife.

\textbf{romnhirarã} \textit{s.} = \textbf{ĩsirãrã} -- flor.

\textbf{romnhirãrãzé} \textit{s.} -- maio.

\textbf{romnhitobzé} \textit{s.} -- barreira.

\textbf{romnhisi} \textit{s.} -- palavra, termo, vocábulo, nome, conceito.

\textbf{romnhisi'ubumro} \textit{s.} -- dicionário, vocábulo.

\textbf{romnhoré} \textit{v.} -- ensinar, estudar.

\textbf{romnhoré} \textit{s.} -- estudo, ensino, aula, ciência.

\textbf{romnhorézé} \textit{s.} -- escola.

\textbf{romnhoré höimana} \textit{s.} -- currículo.

\textbf{romnhoré na dasi'wapé} \textit{s./posp.} -- concurso.

\textbf{romnhoré nhib'ri} \textit{s.} -- colégio, escola.

\textbf{romnhoré nhihödö} \textit{s.} -- livro, caderno.

\textbf{romnhoré nhihödö ĩpe} \textit{s.} -- bíblia.

\textbf{romnhorénhipese'wa} \textit{s.} -- cientista.

\textbf{romnhoré nhihötö ubumrozé} \textit{s./s.} -- biblioteca.

\textbf{romnhorési'rẽ} \textit{adj.} -- científico.

\textbf{romnhoré'wa} \textit{s.} -- professor.

\textbf{romnhoré wa'ö} \textit{s.} -- nota de escola.

\textbf{romnhoré wa'öbözẽ} \textit{s.} -- boletim.

\textbf{romnhoré wa'õno} \textit{s.} -- classe, aula, série.

\textbf{romnomri} \textit{v.} -- adivinhar.

\textbf{ronhimihöze} \textit{s.} -- far-oeste.

\textbf{ronhimimnha} \textit{s.} -- perigo.

\textbf{rõno} \textit{v.} -- subir, pular.

\textbf{ronomro} \textit{v.} -- conjecturar.

\textbf{ronomro} \textit{s.} -- demora.

\textbf{ron'runho'uma} \textit{s.} -- governo.

\textbf{ro'o} \textit{v} -- \textbf{ĩro} acender luz, acender fogo, iluminar, esquentar, aquecer.

\textbf{ro'o} \textit{s.} -- \textbf{ĩro} luz, fogo; \textbf{ĩro'o da si'uwazi} -- fio elétrico.

\textbf{ro'o} \textit{s.} -- macaco.

\textbf{ro'o'ahö} \textit{s.} -- volt.

\textbf{ro'oza'ẽne} \textit{s.} -- clarão.

\textbf{ro'ozaparizé} \textit{s.} -- \textbf{ĩro'ozaparizé} transformador.

\textbf{ro'ozé} \textit{s.} -- material para fogo, lenha, fogão.

\textbf{ro'o'manharĩzé} \textit{s.} -- \textbf{ĩro'o 'manharĩzé} gerador.

\textbf{ro'o na robduri} \textit{s./posp.} -- bonde.

\textbf{ro'o nhiptede} \textit{s.} -- \textbf{ĩro'o nhiptede} energia elétrica, amperagem.

\textbf{ro'o nhotopa} \textit{s.} -- \textbf{ĩro'o nhotopa} incêndio.

\textbf{ro'opibuzé} \textit{s.} -- \textbf{ĩro'opibuzé} medidor de eletricidade.

\textbf{ro'opodo} \textit{s.} -- \textbf{ĩro'opodo} projetor, cinema.

\textbf{ro'opotozandizé} \textit{s.} -- \textbf{ĩro'opotozandizé} projetor.

\textbf{'ro'ora} \textit{s.} -- bugio.

\textbf{'ro'orawa'a} \textit{s.} -- preguiça (animal).

\textbf{'ro'orawawẽ} \textit{s. aum.} -- gorila.

\textbf{'ro'ore} \textit{s.} -- macaco.

\textbf{ro'ore} \textit{s.} -- \textbf{ĩro'ore} chama, luz lâmpada, fogo.

\textbf{ro'oreza'ozé} \textit{s.} -- \textbf{ĩro'oreza'ozé} soquete, candelabro.

\textbf{ro'ore pãrĩzé} \textit{s.} -- interruptor.

\textbf{ro'osipãrĩ} \textit{s.} -- macaco perigoso, macaco assassino.

\textbf{ro'o'uprosizé} \textit{s.} -- \textbf{ĩro'o uprosizé} watt.

\textbf{ro'o uwaimramizé} \textit{s.} -- \textbf{ĩro'o uwaimramizé} regulador de voltagem.

\textbf{ropãrĩ} \textit{v. du.} -- desmaiar.

\textbf{rope} \textit{s.} -- lugar limpo, lugar preparado.

\textbf{rope} \textit{v.} -- limpar, preparar; \textbf{dasima rope} confessar-se.

\textbf{ropé} \textit{v.} -- espalhar, andar em todo o lugar.

\textbf{ropese} \textit{v} -- \textbf{rope} limpar, preparar.

\textbf{ropese'wa} \textit{s.} -- preparador.

\textbf{ropĩ} \textit{s.} -- mel.

\textbf{ropibu} \textit{v.} -- zelar, guardar, cuidar; \textbf{ãma ropibu} conduzir.

\textbf{ropizari} \textit{v. du.} -- voltar, retomar.

\textbf{ropĩni} \textit{s.} -- \textbf{ropĩ} mel.

\textbf{ropipa} \textit{s.} -- perigo.

\textbf{ropipa pu} \textit{s.} -- catástrofe.

\textbf{ropire} \textit{s.} -- dificuldade.

\textbf{ropi'reba} \textit{s.} -- sul.

\textbf{ropisudu} \textit{s.} -- promessa, compromisso; \textbf{asa ropisudu} condição.

\textbf{ropisudu} \textit{v.} -- prometer, comprometer.

\textbf{ropisutu zahi} \textit{s.} -- contrato.

\textbf{ropisutuzé} \textit{s.} -- promessa, promessa bilateral escrita.

\textbf{ropisutu'ru} \textit{s.} -- tratado, voto.

\textbf{ropisutu si'uiwa nhihödö} \textit{s.} -- convênio, contrato.

\textbf{ropö} \textit{adv.} -- em todo o lugar.

\textbf{ropo} \textit{v.} -- espalhar, difundir.

\textbf{ropö dasisamro} \textit{adv./s.} -- circulação.

\textbf{ropodo} \textit{s.} -- criatura.

\textbf{ropo'o} \textit{v} -- \textbf{ropo} espalhar, difundir.

\textbf{ropöré} \textit{advmp.} -- em todo o lugar.

\textbf{ropöre} \textit{adv. dim.} -- comum.

\textbf{ropotonebre} \textit{s.} -- bicho, animal, animal que anda.

\textbf{ropoto'wa} \textit{s.} -- criador.

\textbf{ropru} \textit{s.} -- lixo.

\textbf{ropta} \textit{v.} -- bater, martelar.

\textbf{ropta'a} \textit{v} -- \textbf{ropta} bater, martelar.

\textbf{roptata} \textit{v.} -- dançar, bater palmas, batida frequente.

\textbf{roptata'a} \textit{v} -- \textbf{roptata} dançar, bater palmas, batida frequente.

\textbf{ropté} \textit{s.} -- terra nova, lugar novo.

\textbf{roptede'wa} \textit{s.} -- governo.

\textbf{ropti} \textit{v.} -- recomendar, insistir, pedir; \textbf{ãma ropti} comprimir.

\textbf{roptö} \textit{v.} -- aproximar-se.

\textbf{roptö} \textit{s.} -- lama no chão.

\textbf{ropsãmri} \textit{v.} -- ver.

\textbf{ropsãmri} \textit{v.} -- acertar com bola, pedra, flecha.

\textbf{ropupu} \textit{s.} -- barulho forte, estrondo.

\textbf{ropupuzé} \textit{s.} -- bateria (banda).

\textbf{ro'razöri} \textit{v. sing.} = \textbf{ro'rahöri} \textit{v. du.} passar por, atravessar.

\textbf{ro'rahöri} \textit{v. du./pl.} -- passar por, atravessar.

\textbf{ro'ro} \textit{v.} -- \textbf{ĩro'ro} chiar, crepitar.

\textbf{rotédé} \textit{v.} -- aproximar.

\textbf{rotété} \textit{v} -- \textbf{rotédé} aproximar.

\textbf{roti} \textit{v.} -- ensinar, dar conselho.

\textbf{rotizé} \textit{s.} -- lugar do conselho, conselho.

\textbf{rotinhipai'wa} \textit{s.} -- bispo.

\textbf{rotinhipai'wa ĩ'uzapré} \textit{s.omp.} -- cardeal.

\textbf{roti'upai'õ'wa} \textit{s.} -- papa.

\textbf{roti'uwaimrami'wa} \textit{s.} -- juiz.

\textbf{roti'wa} -- conselheiro, chefe; \textbf{dama roti'wa} conselheiro.

\textbf{roti'wa ĩhi} \textit{s.} -- senador.

\textbf{rotiwẽza'u'wa} \textit{s.} -- discípulo, apóstolo.

\textbf{rõtõ} \textit{v} -- \textbf{rõno} subir, pular.

\textbf{rosahudu} \textit{v.} -- aproximar.

\textbf{rosahutu} \textit{v} -- \textbf{rosahudu} aproximar.

\textbf{rosa'rada} \textit{v.} -- pensar, cogitar, contemplar.

\textbf{rosa'rada} \textit{s.} -- pensamento, cogitação, contemplação.

\textbf{rosa'rata} \textit{v} -- pensar, cogitar, contemplar.

\textbf{rosa'tatazé} \textit{s.} -- cérebro.

\textbf{rosa'rata pisudu} \textit{s.} -- capricho.

\textbf{rosa're} \textit{v.} -- saber, conhecer.

\textbf{rosa'rese} \textit{v} -- \textbf{rosa're} saber, conhecer.

\textbf{rosawẽrẽ} \textit{v.} -- sonhar.

\textbf{rosawẽrẽ} \textit{s.} -- sonho, visão.

\textbf{rosawẽrẽ'wa} \textit{s.} -- sonhador.

\textbf{rosawi} \textit{v.} -- proibir, afastar, negar, rejeitar; \textbf{dawi rosawi} proibir a alguém.

\textbf{rosõ'õ} \textit{adj.} -- primogênito, geral.

\textbf{rosõwada} \textit{vmp.} -- gerar.

\textbf{rosõwada'rada} \textit{s.} -- primogênito.

\textbf{rowa} \textit{s.} -- gordura, banha.

\textbf{ro'wa} \textit{s.} -- espinho.

\textbf{rowa'a} \textit{s.} -- dia, clareza, clarão, claridade.

\textbf{rowa'a} \textit{v.} -- clarear, amanhecer; \textbf{ma tô tirowa'a} amanheceu, clareou.

\textbf{rowazé} \textit{s.} -- respeito, vergonha.

\textbf{rowazé} \textit{adj.} -- respeitoso, vergonhoso; \textbf{rowazéb di é} respeitoso.

\textbf{rowahö} \textit{s.} -- tempo frio, friagem.

\textbf{rowahudu} \textit{v.} -- ensinar, conduzir, cultuar.

\textbf{rowahudu} \textit{s.} -- ensino, culto.

\textbf{rowahutu} \textit{v} -- \textbf{rowahudu} ensinar, conduzir, cultuar.

\textbf{rowahutuzé} \textit{s.} -- aula, culto, igreja, centro da aldeia.

\textbf{rowahutu'wa} \textit{s.} -- professor, ministro do culto.

\textbf{rowaihu} \textit{v.} -- aprender, compreender.

\textbf{rowaihu'u} \textit{v} -- \textbf{rowaihu} aprender, compreender.

\textbf{rowaihu'u na dasi'wapé} \textit{s./posp.} -- certame.

\textbf{rowaihu'upe} \textit{s.} -- conhecimento.

\textbf{rowaihu'upe} \textit{s.} -- conhecer a fundo.

\textbf{rowaihu'upese'wa} \textit{s.} -- conhecedor, doutor.

\textbf{rowaihu'u'wa} \textit{s.} -- discípulo.

\textbf{rowaihu'u wa'ö} \textit{s.} -- nota escolar.

\textbf{rowai'o} \textit{s.} -- lama, barro mole.

\textbf{ro'wa ĩ'ré} \textit{s.} -- espinho seco.

\textbf{rowairébé} \textit{v. pl.} -- coordenar, pensar, cogitar, combinar.

\textbf{rowairébé} \textit{s.} -- coordenação, pensamento, cogitação, combinação.

\textbf{rowãi wa} \textit{v.} -- no meio, no centro, em.

\textbf{ro'wamarĩ 'madö'öze} \textit{s.} -- belas artes.

\textbf{rowamirĩzé} \textit{s.} -- peneira.

\textbf{rowamnarĩ} \textit{} -- destruir, estragar.

\textbf{rowamnarĩzei'wa} \textit{s.} -- bandido.

\textbf{rowamreme} \textit{s.} -- barulho, vozerio.

\textbf{rowamri} \textit{s.} -- nome do lugar.

\textbf{ro'wanhipré} \textit{s.} -- broto, planta, árvore.

\textbf{rowanhirĩtĩ} \textit{s.} -- vão, brecha, picada.

\textbf{rowanhirĩtĩ} \textit{v.} -- abrir caminho.

\textbf{ro'wapé} \textit{v. du.} -- carregar.

\textbf{ro'wapézé} \textit{s.} -- bolsa, veículo.

\textbf{ro'wapéi'wa} \textit{s.} -- carregador da tora de buriti, carregador.

\textbf{rowapõrĩ} \textit{v.} -- soprar forte.

\textbf{rowaptẽrẽ} \textit{v.} -- pedir, exigir.

\textbf{rowaptẽrẽzé} \textit{s.} -- pedido, exigência.

\textbf{rowaptö} \textit{v.} -- demorar, não sair briga, não cumprir promessa.

\textbf{rowaptö'ö} \textit{v} -- \textbf{rowaptö} demorar, não sair briga, não cumprir promessa.

\textbf{rowa'ré} \textit{v.} -- atrapalhar, descontentar, complicar.

\textbf{rowa'ré} \textit{} -- desordem, complicação, controvérsia.

\textbf{ro'wa'ré} \textit{s.} -- espinho seco.

\textbf{ro'wa'ri} \textit{v.} -- suspirar.

\textbf{rowarĩrĩ} \textit{s.} -- clareira.

\textbf{rowa'ro} \textit{s.} -- calor.

\textbf{rowa'rudu} \textit{v.} -- incomodar, bagunçar, atrapalhar.

\textbf{rowa'rudu} \textit{s.} -- incômodo, bagunça.

\textbf{rowa'rutu} \textit{v} -- \textbf{rowa'rudu} incomodar, bagunçar, atrapalhar.

\textbf{ro'wasa} \textit{v. sing.} -- carregar.

\textbf{ro'wasari} \textit{v. pl.} -- carregar, levar, transportar.

\textbf{rowasédé} \textit{s.} -- briga, mal, ruimdade, violência, confusão; \textbf{dasi'ãma rowasédé} conflito.

\textbf{rowasétézépa} \textit{s.} -- longa violência; Aldeia N. S. Aparecida (R. I. Couto Magalhães).

\textbf{rowasu} \textit{s.} -- comunicação.

\textbf{rowasudu} \textit{v.} -- molestar, bagunçar, perseguir.

\textbf{rowasudu} \textit{s.} -- bagunça, amolação, perseguição.

\textbf{rowasutu} \textit{v} -- \textbf{rowasudu} molestar, bagunçar, perseguir.

\textbf{rowasu'u} \textit{v.} -- contar, conversar; \textbf{dasima rowasu'u} conversar.

\textbf{rowasu'u} \textit{s.} -- conversa, conto, história, bate-papo; \textbf{dasima rowasu'u} conversa.

\textbf{rowasu'u ĩsihötö na} \textit{s./posp.} -- correspondência, carta.

\textbf{rowasu'u nhihödö} \textit{s.} -- jornal, revista, boletim.

\textbf{rowasu'u nhihötö'wa} \textit{s.} -- escritor.

\textbf{rowasu'u nhoré} \textit{s.} -- revista.

\textbf{rowasu'u'wa} \textit{s.} -- pessoa que conta, profeta, fuxiqueiro, chateador.

\textbf{rowasu'u'wai'rada} \textit{s.} -- profeta.

\textbf{rowa'u} \textit{s.} -- vento, sopro.

\textbf{rowa'u'uhöire} \textit{s.} -- brisa.

\textbf{rowa'u'u pibuzé} \textit{s.} -- catavento.

\textbf{rowa'u'u pire pibuzé} \textit{s.} -- barômetro.

\textbf{rowa'u'u wahö} \textit{s.} -- vento frio.

\textbf{rowa'u'uwawẽ} \textit{s. aum.} -- temporal, tempestade.

\textbf{rowẽ} \textit{s.} -- paz, bonança, alegria, lugar bom, céu; \textbf{apö rowẽ} convalescer, convalescência.

\textbf{rowẽ} \textit{adj.} -- bom.

\textbf{rowẽ} \textit{v.} = \textbf{sima rowẽ} -- alegrar-se, gostar.

\textbf{rowede} \textit{s.} -- comércio.

\textbf{rowede} \textit{s.} -- adubo, fertilizante.

\textbf{rowede'wa} \textit{s.} -- vendedor, comerciante, mercador.

\textbf{rowede'wa} \textit{s.} -- aplicador de adubo.

\textbf{rowẽ za'ẽne} \textit{adj./adj.} -- grande paz, grande felicidade, grande alegria.

\textbf{rowẽza'ẽne} \textit{s.} -- muito bom, felicidade, glória.

\textbf{rowẽzé} \textit{s.} -- bondade.

\textbf{rowẽ na aimorĩ} \textit{adj./posp./v.} -- vai em paz.

\textbf{rowi} \textit{adv.} -- fora.

\textbf{'rowi} \textit{posp.} -- debaixo de; \textbf{'ri'rowi} debaixo da casa.

\textbf{'röwi} \textit{adv.} -- por sua vez, perto de.

\textbf{'ru} \textit{s.} -- rato.

\textbf{'ru} \textit{s.} -- \textbf{ĩ'ru} raiva, ordem, mando, ódio, desprezo.

\textbf{'ru} \textit{adj.} -- \textbf{ĩ'ru} zangado, irado.

\textbf{'ru} \textit{s.} -- lugar, túmulo; \textbf{adö'ö'ru} cemitério.

\textbf{'ru} \textit{v.} -- irar-se, zangar-se, desprezar, mandar; \textbf{õ hã, ma ti'ru} ele mandou; \textbf{dama ĩ'ru} obrigação.

\textbf{'ru'a} \textit{s.} -- cinza.

\textbf{'rubö} \textit{s.} -- rato.

\textbf{'rubu} \textit{s.} -- \textbf{ĩ'rubu} sede.

\textbf{'rubu} \textit{adj.} -- \textbf{ĩ'rubu} sedento, com sede.

\textbf{rudu} \textit{adj.} -- \textbf{ĩrudu} crespo, áspero, irregular.

\textbf{'rudu} \textit{adj.} -- \textbf{ĩ'rudu} curto, breve.

\textbf{'ru zahi} \textit{s.} -- \textbf{ĩ'ru zahi} ordem severa, dever, obrigação.

\textbf{'ruzahi} \textit{s.} -- \textbf{ĩ'ruzahi} mandamento, lei, constituição.

\textbf{'ruzé} \textit{s.} -- \textbf{ĩ'ruzé} ordem doída.

\textbf{'rui'manharĩ} \textit{vmp.} -- odiar.

\textbf{'ruiwa'öbö} \textit{v.} -- exigir algo por ordem, prestar contas.obrar.

\textbf{'ruiwapari} \textit{vmp.} -- odiar, hostilizar; \textbf{dasi'ruiwapari} arrepender-se.

\textbf{'runomri'wa} \textit{s.} -- \textbf{ĩ'runomri'wa} deputado.

\textbf{'ru'õ} \textit{adj. comp.} -- \textbf{ĩ'ru'õ} sem obrigação.

\textbf{'rupo're} \textit{s.} -- lebre.

\textbf{'rupo'rere} \textit{s.} -- coelho.

\textbf{'rupré} \textit{s.} -- camundongo.

\textbf{'ru'rã} \textit{s.} -- uva.

\textbf{'ru'raĩre ro} \textit{s.} -- vinha.

\textbf{'ru'rãire ro 'madö'ö'wa} \textit{s./s.} -- vinhateiro.

\textbf{'rure} \textit{s.} -- rato.

\textbf{ruru} \textit{s.} -- ruído do motor, partida do motor.

\textbf{ruru'u} \textit{v} -- \textbf{ruru} dar partida no motor.

\textbf{'ru-te} \textit{posp.} -- por causa de (algo negativo); \textbf{taha 'ru-te} por isso.

\textbf{'rutõ} \textit{adv.} -- não obrigado.

\textbf{'rutömhö'ã} \textit{s.} -- pequeno roedor.

\textbf{'rutu} \textit{adj. c.} -- \textbf{ĩ'rudu} curto.

\textbf{'rutu} \textit{v} -- \textbf{ĩ'rudu} encurtar.

\textbf{'ruture} \textit{adj. dim.} -- \textbf{ĩ'ruture} curtinho, pertinho.

\textbf{'ru'wa} \textit{s.} -- flecheiro, inimigo do bicho.

\textbf{'ry'ry} \textit{v. pl.} -- chorar; \textbf{õ hã, te 'ry'ry} ele chora.


%####################################################

\section*{S}

\textbf{sa} \textit{v. sing.} -- \textbf{ĩza} = \textbf{dasima'wara} \textit{v. du.} ficar de pé; \textbf{asa na} fique de pé.

\textbf{sa} \textit{v. sing.} = \textbf{sari} -- morder, picar; \textbf{ma tisa} ele mordeu.

\textbf{sa} \textit{v.} -- comer.

\textbf{sa} \textit{s.} -- comida.

\textbf{sa'a} \textit{s.} -- morro, colina.

\textbf{sa'a} \textit{v. sing.} -- ficar de pé, elevar, suspender, erguer, permanecer.

\textbf{sa'ã} \textit{v.} -- levantar, carregar (terra, areia).

\textbf{sababa} \textit{posp.} -- ao lado de.

\textbf{sababa} \textit{s.} -- \textbf{ĩsababa} banda, lado.

\textbf{sabzé} \textit{s.} -- lugar de ficar de pé.

\textbf{sabödö} \textit{s.} -- \textbf{ĩsabödö} vizinho, junto, debaixo.

\textbf{sabödö} \textit{v.} -- estar perto, compadecer-se, acompanhar, fazer por gosto.

\textbf{sabu} \textit{v.} -- ver.

\textbf{sada} \textit{posp.} -- \textbf{ĩsada} para, contra (finalidade).

\textbf{sada} \textit{s.} -- queimada, sapecada.

\textbf{sada} \textit{v.} -- cremar, queimar, sapecar.

\textbf{sadab're} \textit{s.} -- covinha na face, rugas de tristeza, desprezo.

\textbf{sadaze} \textit{s.} -- cheiro.

\textbf{sadaze} \textit{v.} -- \textbf{ĩsadaze} cheiroso.

\textbf{sadazöri} \textit{s.} -- \textbf{wede sadazöri} caixa, cofre, baú.

\textbf{sadazu} \textit{s.} -- . . . .

\textbf{sadazute} \textit{s.} -- parte superior da perna, coxa.

\textbf{sadazutenhi} \textit{s.} -- músculo da parte superior da perna, da coxa.

\textbf{sada'é} \textit{v.} -- cumprir palavra.

\textbf{sadahi'rãpo} \textit{s.} -- jaú.

\textbf{sadaihu} \textit{v.} -- compreender, conversar, combinar, entender-se.

\textbf{sadaihu'u} \textit{v} -- entender, compreender, conversar, combinar.

\textbf{sadaipro} \textit{s.} -- saliva, escarro, cuspe.

\textbf{sadai'ré} \textit{s.} -- garganta, boca (interior).

\textbf{sadai'ré} \textit{v.} -- falar.

\textbf{sadamnhasi} \textit{s.} -- medo de palavra pesada.

\textbf{sadamri} \textit{v. sing.} -- \textit{v. du./pl.} cheirar.

\textbf{sadamrimi} \textit{s. du. pl.} -- cheirar.

\textbf{sadanha} \textit{v. sing.} -- perguntar, pedir.

\textbf{sadanharĩ} \textit{v.} -- perguntar, pedir.

\textbf{sada'ö} \textit{v. sing.} = \textbf{dazada'ö} -- responder.

\textbf{sada'öbö} \textit{s. du. pl.} -- responder, retrucar, atender.

\textbf{sada'õno} \textit{s.} -- coxa.

\textbf{sadapada} \textit{s.} -- costas, omoplata.

\textbf{sadapada} \textit{s.} -- rosto, bochechas.

\textbf{sadapada wairĩ} \textit{s.} -- torcer as bochechas para castigo.

\textbf{sadapri'ri} \textit{s.} -- . . . \textbf{dazadapri'ri} . . . .

\textbf{sadapri'rine} \textit{s.} -- arbusto de que se tira fibras para fazer cordas para o pescoço.

\textbf{sadapsy} \textit{s.} -- estalo da língua como sinal de admiração ou aprovação.

\textbf{sadarã} \textit{s.} -- brejo, pântano, vargem.

\textbf{sada'rã} \textit{s.} -- pintura preto dos lábios.

\textbf{sadarada} \textit{s.} -- lado debaixo dos braços, banda, quadrado, tórax.

\textbf{sada'ratare} \textit{s.} -- xadrez.

\textbf{sada'ré} \textit{s.} -- bolo de milho.

\textbf{sada'ré} \textit{s.} -- \textbf{ĩsada'ré} beira, limite, beirada; \textbf{ö zada'ré} beira-d'água.

\textbf{sada'réze} \textit{s.} -- bolo doce.

\textbf{sadari} \textit{v.} -- gritar, clamar, proclamar, bradar, louvar.

\textbf{sadari} \textit{s.} -- grito, clamor, brado, louvor.

\textbf{sada'ro} \textit{s.} -- sol; grupo etário.

\textbf{sada'ro} \textit{s.} -- hálito.

\textbf{sada'rohidi} \textit{s.} -- abelha.

\textbf{sada'u} \textit{s.} -- saliva.

\textbf{sadawa} \textit{s.} -- boca, palavra.

\textbf{sadawa'a'a} \textit{s.} -- gritaria, aplauso, louvor.

\textbf{sadawa'a'a} \textit{v.} -- gritar, aplaudir, louvar, cantar.

\textbf{sadawa'ahu'õ} \textit{adj. comp.} -- incômodo, "chato".

\textbf{sadawaza'a} \textit{vmp.} -- bocejar.

\textbf{sadawa zaro} \textit{s./v. sing.} -- levantar a voz, contestar, reagir, não obedecer.

\textbf{sadawa zarono} \textit{s./v. du./pl.} -- levantar a voz, contestar, reagir, não obedecer.

\textbf{sadawahatu} \textit{v.} -- conversar barulhentamente.

\textbf{sadawanhipe} \textit{s.} -- mentira, mentiroso.

\textbf{sadawanhipe} \textit{vmp.} -- mentir.

\textbf{sadawapara} \textit{s.} -- representante, em nome de, delegado, autoridade, substituto.

\textbf{sadawa situri} \textit{s.} -- alegria, louvor.

\textbf{sadawa situri} \textit{s./v.} -- conversar barulhentamente, vociferar.

\textbf{sadö} \textit{s.} -- coberta; cobrir; \textbf{bazadö} cobertor das costas do animal.

\textbf{sadutudu} \textit{s.} -- amarração dos cabelos como rabo de galo.

\textbf{saze} \textit{v.} -- crer, acreditar, concordar, consentir.

\textbf{sazé} \textit{s.} -- cadeia, prisão.

\textbf{sazé} \textit{s.} -- comida.

\textbf{sazéb 'madö'ö'wa} \textit{s.} -- carcereiro.

\textbf{sazezé} \textit{s.} -- consentimento, fé.

\textbf{sazei'wa} \textit{s.} -- crente, fiel, confiante.

\textbf{sazöri} \textit{v. sing.} -- \textit{v. du./pl.} parar, interromper, terminar, partir para.

\textbf{sazu} \textit{adj.} -- \textbf{ĩsazu} cinzento.

\textbf{sazu} \textit{s.} -- comida torrada.

\textbf{sazuzu} \textit{s.} -- farinha torrada, farinha de milho.

\textbf{sa'é} \textit{} -- . . . .

\textbf{sa'ézé} \textit{s.} -- . . . .

\textbf{sa'ẽne} \textit{adj.} -- \textbf{isa'ẽne} grande, bastante.

\textbf{sa'ẽtẽ} \textit{adv.} -- depressa, rapidamente, bruscamente, em voz alta.

\textbf{sa'ẽtẽ} \textit{adj. c.} -- \textbf{ĩsa'ẽnẽ} grande, bastante.

\textbf{sahi} \textit{adj.} -- \textbf{ĩsahi} valente, corajoso, bravio, bravo, selvagem.

\textbf{sahi} \textit{v.} -- zangar-se, ficar valente, corajoso, criar coragem; \textbf{õ hã, ma tizahi} ele ficou zangado.

\textbf{sahi'ö} \textit{adj. comp.} -- \textbf{ĩsahi'õ} calmo, manso, tranquilo, covarde.

\textbf{sahö} \textit{v. sing.} -- \textbf{sahöri} \textit{v. du.} parar, cessar, cortar, interromper, partir para.

\textbf{sahödö} \textit{v.} -- \textbf{ĩsabödö} amontoar, ajuntar.

\textbf{sahöpö} \textit{posp.} -- \textbf{ĩsahöpö} embaixo de, debaixo de.

\textbf{sahöri} \textit{v. du./pl.} -- \textbf{ĩsahöri} parar, cessar.

\textbf{sahörizé} \textit{s.} -- \textbf{ĩsahörizé} parada, pausa.

\textbf{sahöri'wa si'rézé} \textit{s.} -- campanha.

\textbf{sahu} \textit{v.} -- repetir, renovar, atacar, agredir.

\textbf{sahu} \textit{adj.} -- \textbf{ĩsahu} outra vez, repetindo, bis.

\textbf{sahui'wa} \textit{s.} -- agressor, atacante.

\textbf{sahuré} -- dois, ambos.

\textbf{sai'a'a} \textit{s.} -- resto de comida, migalhas.

\textbf{sa'idi} \textit{s.} -- capoeira, matinha no campo.

\textbf{sa'idi} \textit{posp.} -- no meio de.

\textbf{saizu} \textit{s.} -- digestão.

\textbf{saizupe} \textit{s.} -- digestão boa.

\textbf{saizuzépu} \textit{s.} -- congestão.

\textbf{saihö} \textit{s.} -- boca, lábio, beirada.

\textbf{saihö} \textit{s.} -- que come sem parar, que é demorado em comer.

\textbf{saihö 'rudu} \textit{s.} -- lábio curto.

\textbf{saihösu} \textit{s.} -- bigode.

\textbf{saihuri} \textit{v.} = \textbf{si} \textit{v. du. comp. pl.} -- comer.

\textbf{saipãrĩ} \textit{s.} -- casamento.

\textbf{saipãrĩzé} \textit{} -- caça de casamento.

\textbf{saiprã} \textit{s.} -- \textbf{ĩsaiprã} que come qualquer coisa.

\textbf{sai'rada} \textit{s.} -- estrume, esterco, fezes.

\textbf{sai'rata 'warizé} \textit{s.} -- ânus.

\textbf{sa'idi} \textit{posp.} -- \textbf{ĩsa'idi} no meio de.

\textbf{sa'idi} \textit{s.} -- matinha no campo, capoeira.

\textbf{sa'iti prédu} \textit{} -- tempo, hora certa.

\textbf{sai'u} \textit{v. sing.} -- subir, sobressair.

\textbf{sai'uzé} \textit{s.} -- verdura.

\textbf{sai'uri} \textit{s. du. pl.} -- subir, sobressair.

\textbf{sai'urizé} \textit{s.} -- escada, escada de madeira, escadaria.

\textbf{saiwaihãzé} \textit{s.} -- cebola.

\textbf{saiwai'o} \textit{s.} -- caldo, sopa.

\textbf{saiwapu} \textit{s.} -- biscoito.

\textbf{saiwarĩrĩzé} \textit{s.} -- sal.

\textbf{sa'mahöpãrĩ} \textit{s.} -- gula.

\textbf{samarĩ} \textit{v.} -- seguir atrás, imitar, acompanhar; \textbf{ĩsihötö zamarĩ} cópia.

\textbf{samarĩ'wa} \textit{s.} -- seguidor, discípulo, companheiro.

\textbf{samini} \textit{v.} -- enfeitar com penas brancas.

\textbf{samo} \textit{s.} -- \textbf{ĩsamo} cauda de palha de buriti.

\textbf{sãmra} \textit{v. pl.} = \textbf{wabzuri} \textit{v. du.} -- jogar fora, despejar, lançar, destruir.

\textbf{sãmri} \textit{v.} -- ver, enxergar.

\textbf{sãna} \textit{s.} -- intestino, fezes.

\textbf{sãna za'ẽne} \textit{s.} -- intestino grosso.

\textbf{sãnahi} \textit{s.} -- umbigo.

\textbf{sãnahözé} \textit{s.} -- indigesto.

\textbf{sãnahöri'wa} \textit{s.} -- mulher que corta o cordão umbilical.

\textbf{sãnanhihörizé} \textit{s.} -- taquara para cortar cordão umbilical.

\textbf{sãnapré} \textit{s.} -- pintura na barriga.

\textbf{sãna'rada} \textit{v.} -- começar, iniciar.

\textbf{sãna'rãpré} \textit{s.} -- verme intestinal.

\textbf{sana'rata} \textit{v} -- \textbf{sãna'rada} começar, iniciar.

\textbf{sãna'rãsuture} \textit{s.} -- apêndice (do intestino).

\textbf{sãna're} \textit{num.} -- \textbf{ĩsãna're} segundo (2º), em seguida.

\textbf{sãna'ré} \textit{s.} -- prisão de ventre.

\textbf{sãna'rézé} \textit{s.} -- dificuldade de defecar, dispepsia.

\textbf{sãna'remhã} \textit{num.} -- \textbf{ĩsãna'remhã} segundo (2º), atrás de alguém.

\textbf{sãna'u} \textit{s.} -- diarréia.

\textbf{sãnawai'u} \textit{s.} -- verme intestinal.

\textbf{sanhamri} \textit{v.} -- trançar, tecer.

\textbf{sanho} \textit{v.} -- ensinar, ensaiar, dar exemplo.

\textbf{sani} \textit{v.} -- tirar, livrar, retirar, afastar, afugentar.

\textbf{sani'wa} \textit{s.} -- \textbf{ĩsani'wa} cunhada da mulher, nora.

\textbf{sano} \textit{} -- . . . .

\textbf{sanozé} \textit{s.} -- cinza quente.

\textbf{sa'o} \textit{v. du.} -- pender, suspender, suspender tampa, abrir, elevar.

\textbf{sa'odo} \textit{s.} -- \textbf{ĩsa'odo} contorno.

\textbf{sa'odo} \textit{v.} -- \textbf{ĩsa'odo} contornar.

\textbf{sa'ozé} \textit{s.} -- lugar de estar pendurado, lugar de suspender.

\textbf{sa'õmo} \textit{s.} -- cunhado, genro.

\textbf{sa'o'o} \textit{v. du./pl. c.} = \textbf{daza'o} -- pender.

\textbf{sa'oro} \textit{v. sing. c.} = \textbf{daza'o} -- pender.

\textbf{sa'oto} \textit{v.} -- dar volta.

\textbf{sapa'a} \textit{v.} -- tomar conta, participar, interessar-se, respeitar, cuidar.

\textbf{sapa'abö} \textit{s.} -- ajuda esperada, devia interessar-se.

\textbf{sapa'ahö} \textit{vmp.} -- tomar conta.

\textbf{sapa'a'õ} \textit{vmp.} -- desrespeitar, descuidar.

\textbf{sapada} \textit{v.} -- sentar de perna cruzada.

\textbf{sapari} \textit{v.} -- apoiar, segurar, esperar, vigiar, aguardar.

\textbf{saparizé} \textit{s.} -- apoio, lugar de espera.

\textbf{sapari'wa} \textit{s.} -- apoio, segurança, vigia, guarda, incentivador.

\textbf{sapo} \textit{adj.} -- \textbf{ĩsapo} baixo, pequeno.

\textbf{sapodo} \textit{adj.} -- \textbf{ĩsapodo} redondo.

\textbf{sapodo} \textit{s.} -- círculo.

\textbf{sapo'o} \textit{v.} -- cavar, fazer buraco.

\textbf{sapõri} \textit{v.} -- soprar, aspirar, tirar doença, ventilar.

\textbf{sapoto 'manharĩzé} \textit{s.} -- compasso.

\textbf{sapotore} \textit{s.} -- \textbf{ĩsapotore} pequeno redondo, moeda.

\textbf{sa pre} \textit{interj.} -- \textbf{za pre} excl. de espanto.

\textbf{saprĩ} \textit{v.} -- trocar, passar para o outro lado, atravessar, traduzir; \textbf{ãma saprĩ} cambiar; \textbf{dahõimana na ãma ĩsaprĩ} converter-se, conversão (da vida).

\textbf{saprõni} \textit{v. du./pl.} -- levar, carregar, dirigir, conduzir.

\textbf{sapru} \textit{s.} -- poeira.

\textbf{sapu} \textit{s.} -- \textbf{ĩsapu} cavidade.

\textbf{sapudu} \textit{s.} -- cabeceira.

\textbf{sapu'u} \textit{v.} -- furar.

\textbf{sara} \textit{s.} -- \textbf{ĩsara} colmeia.

\textbf{sa'ra} \textit{v. pl.} = \textbf{nomri} \textit{v. du.} -- pôr, agir, colocar, botar, deixar, estender, ficar, recolher, amontoar, reunir, arrumar; \textbf{höimo ĩsara} levantar, erigir.

\textbf{sa'rã} \textit{s.} -- sombra, assombração.

\textbf{sa'rada} \textit{v.} -- contar, enumerar, coordenar, recordar, lembrar, meditar, celebrar missa.

\textbf{sa'rai'wa} \textit{s.} -- quem coloca, suspendedor, guincho.

\textbf{sa'rata} \textit{v} -- \textbf{sa'rada} contar, enumerar, celebrar, recordar, lembrar, meditar, pensar em.

\textbf{sa'rata'ri} \textit{s.} -- \textbf{ĩsa'rata'ri} igreja.

\textbf{sa'rata'ri 'rare} \textit{s.} -- \textbf{ĩsa'rata'ri 'rare} capela.

\textbf{sa'rata'wa} \textit{s.} -- \textbf{ĩsa'rata'wa} sacerdote, padre.

\textbf{sa'ra'wa} \textit{s.} -- juiz, condenador.

\textbf{saré} \textit{s. fem.} -- caçula.

\textbf{sa're} \textit{v.} -- conhecer, saber, manifestar, dar a conhecer.

\textbf{sa're} \textit{v.} -- passar a frente.

\textbf{sa'ré} \textit{v.} -- ceder, afastar, conceder, trair; \textbf{dama ĩsa'ré} conceder, concessão; \textbf{dawasété zarina daza'ré} -- condenar, executar.

\textbf{sa'reba} \textit{s.} -- \textbf{ĩsa'reba} penas pequenas de arara.

\textbf{sarébé} \textit{s.} -- \textbf{ĩsarébé} cauda.

\textbf{sa'réi'õ} \textit{s.} -- \textbf{ĩsaréi'õ} que não cede, cabeçudo.

\textbf{sa'rese} \textit{v} = \textbf{sa're} -- conhecer, saber, manifestar, dar a conhecer.

\textbf{sa'rese'õ} \textit{adj. comp.} -- \textbf{ĩsa'rese'õ} que não dá para conhecer, incognoscível, ignorante.

\textbf{sa'rese'õwẽ} \textit{s.} -- \textbf{ĩsa'rese'õwẽ} que não pode ser conhecido, milagre.

\textbf{sa're'wa} \textit{s.} -- o que passa de alguém, que ultrapassa.

\textbf{sa'ré'wa} \textit{s.} -- traidor.

\textbf{sari} \textit{v.} -- arrastar, puxar arrastando, morder, picar.

\textbf{sãrĩ} \textit{v.} -- \textbf{ĩsãrĩ} pôr em pé.

\textbf{saribi} \textit{s.} -- \textbf{ĩsaribi} penas das asas.

\textbf{sarina} \textit{posp.} -- de acordo com, coerente, seguindo atrás de.

\textbf{sari'wa} \textit{s.} -- seguidor.

\textbf{saro} \textit{v.} -- \textbf{ĩsaro} assar.

\textbf{saro} \textit{v. sing.} -- \textbf{sarõno} levantar; \textbf{sadawa zaro} desobedecer.

\textbf{sarob'a} \textit{s.} -- \textbf{ĩsarob'a} brilhante, claro, vidro, transparente.

\textbf{sarõno} \textit{v.} -- levantar, elevar, erguer-se, pular.

\textbf{sarõtõ} \textit{s.} = \textbf{sarõno} -- levantar, elevar.

\textbf{sarõtõpe} \textit{s.} -- \textbf{ĩsarõtõpe} bola.

\textbf{sa'ru} \textit{s.} -- \textbf{ĩsa'ru} lugar de origem, terra nativa, pátria, residência, moradia, lugar limitado.

\textbf{sa'ru} \textit{s.} -- \textbf{ĩsa'ru} clarão, brinco.

\textbf{sata} \textit{v} -- \textbf{sada} queimar, sapecar, assar, cremar.

\textbf{sate} \textit{v.} -- expulsar, afastar, convidar.

\textbf{satede} \textit{s.} -- casca de árvore para "wapte".

\textbf{sati} \textit{s.} -- limite, confim, último.

\textbf{sati} \textit{v.} -- confinar.

\textbf{satõrĩ} \textit{v.} -- mandar embora, dispensar.

\textbf{satõrĩ we} \textit{v./adv.} -- mandar para cá, mandar de volta.

\textbf{sãsã} \textit{v.} -- \textbf{ĩsãsã} chocalhar.

\textbf{sasi} \textit{s.} -- ninho.

\textbf{sasi} \textit{v.} -- fazer ninho, aninhar.

\textbf{sasi} \textit{adj.} -- \textbf{ĩsasi} egoísta, orgulhoso.

\textbf{sasi ãna} \textit{adj./posp.} -- humilde, generoso.

\textbf{sasizé} \textit{s.} -- suporte, altura, lugar para ninho.

\textbf{sasimri} \textit{v.} -- alcançar.

\textbf{sasi'õ} \textit{adj. comp.} -- \textbf{ĩsasi'õ} doce, generoso.

\textbf{sasisi} \textit{s.} -- ondas (do mar).

\textbf{sasõ} \textit{v. sing.} = \textbf{sasõmri} -- colocar, pregar, pendurar, ser entregue, levantar, dar fé.

\textbf{sasõmri} \textit{v. du./pl.} -- colocar, pregar, pendurar, ser entregue.

\textbf{sasu} \textit{s.} -- \textbf{ĩsasu} penas, pelos.

\textbf{sasu} \textit{s.} -- \textbf{ĩsasu} que come depressa.

\textbf{sasuru} \textit{v. sing.} -- defecar, cagar.

\textbf{sasu'u} \textit{v. du./pl.} -- defecar, cagar.

\textbf{sa'u} \textit{s. neo.} -- saco.

\textbf{sa'u} \textit{v. sing.} -- \textbf{sa'uri} soprar.

\textbf{sa'u} \textit{adj. (adv. )} -- repetindo, outra vez, atrás de, depois de.

\textbf{sa'u} \textit{posp.} -- \textbf{ĩsa'u} depois dele.

\textbf{sa'uire} \textit{s. neo. dim.} -- sacola, embornal.

\textbf{sa'uri} \textit{v.} -- soprar.

\textbf{sa'uri'wa} \textit{s.} -- campeão, vencedor, corrida de iniciação.

\textbf{sa'u'u's} \textit{s.} -- outono.

\textbf{sa'u'wa} \textit{s.} -- último.

\textbf{sawazari} \textit{v.} -- contar, enumerar.

\textbf{sa'wari} \textit{v.} -- jogar fora, derramar, separar, espalhar, despejar.

\textbf{sa'wari} \textit{v. du.} -- deixar-se.

\textbf{sa'wari} \textit{s.} -- colocação.

\textbf{sa'warizé} \textit{s.} -- cama, maca, objeto para despejar.

\textbf{sa'wari höbö} \textit{s.} -- \textbf{ĩsa'wari höbö} camada.

\textbf{sa'wari'wa} \textit{s.} -- dispensador, despenseiro.

\textbf{sawẽrẽ} \textit{v.} -- olhar, meditar, sonhar.

\textbf{sawi} \textit{v.} -- amar, gostar; \textbf{marĩ zawi} conter.

\textbf{sawi} \textit{v.} = \textbf{dawi sawi} -- recusar, afastar, proibir.

\textbf{sawizé} \textit{s.} \textbf{dazawizé} -- amor, amizade.

\textbf{sawire} \textit{s.} -- coitado, carinho.

\textbf{sawi'wa} \textit{s.} -- amigo, camarada.

\textbf{sawörö} \textit{v.} -- explorar.

\textbf{sawörö} \textit{s.} -- canto dos exploradores.

\textbf{sawörö'wa} \textit{s.} -- explorador.

\textbf{se} \textit{adj.} -- \textbf{ĩse} gostoso, agradável, feliz; \textbf{se ti} é gostoso.

\textbf{sẽ} \textit{adj.} -- \textbf{ĩsẽ} doído, doloroso; \textbf{seé di} é doído.

\textbf{sẽ} \textit{v. sing.} -- \textbf{sẽme} meter, colocar; \textbf{õ hã, ma tô tisẽ} ele o colocou.

\textbf{seé} \textit{s.} -- urina.

\textbf{sé} \textit{s.} -- \textbf{ĩsé} gasolina, diesel; \textbf{ĩsé uptabi} diesel; \textbf{ĩsé ro'ope} gasolina.

\textbf{sé} \textit{s.} -- macaúba; \textbf{séwaipo} \textit{s.} broto de macaúba.

\textbf{sebre} \textit{v.} -- \textbf{ĩsebre} cozinhar, assar.

\textbf{sébré} \textit{v. sing.} -- = \textbf{dazasi} \textit{v. du.} entrar.

\textbf{sebrezé} \textit{s.} -- \textbf{ĩsebrezé} cozinha.

\textbf{sébézé} \textit{s.} -- entrada.

\textbf{séi'rã} \textit{s.} -- \textbf{ĩséi'rã} betume.

\textbf{séiro'ope} \textit{s.} -- \textbf{ĩséiro'ope} gasolina.

\textbf{séi'uptabi} \textit{s.} -- \textbf{ĩséi'uptabi} diesel, óleo diesel.

\textbf{sei wapu} \textit{adj./adj.} -- que gostoso!.

\textbf{séiwari} \textit{s.} -- \textbf{ĩséiwari} graxa.

\textbf{sẽme} \textit{v. pl.} = \textbf{ĩza} \textit{v. du.} -- meter, colocar.

\textbf{sẽmere} \textit{s.} -- \textbf{ĩsẽmere} cabo (para enxada etc. ).

\textbf{sẽme're} \textit{s.} -- \textbf{ĩsẽme're} que se mete dentro, bainha.

\textbf{sẽme'wa} \textit{s.} -- colocador, metedor.

\textbf{sena} \textit{s.} -- clareza, verdade.

\textbf{sena ĩpu} \textit{v./adj.} -- concretizar, acontecer.

\textbf{sena ĩsaze} \textit{s./v.} -- comprovar, acreditar mesmo.

\textbf{sena ĩwasu'u} \textit{s./v.} -- comprovar.

\textbf{sena'wa} \textit{s.} -- vereador.

\textbf{sénho} \textit{s.} -- palmito de macaúba.

\textbf{sépada} \textit{s.} -- \textbf{ĩsépada} comoção, sofrimento.

\textbf{sépada} \textit{v.} -- sofrer, comover-se.

\textbf{sépata da ĩrob'ru} \textit{v./conj./s.} -- condenação.

\textbf{sépatazé} \textit{s.} -- sofrimento.

\textbf{sépata'wa} \textit{s.} -- \textbf{ĩsépata'wa} sofredor, vítima.

\textbf{séptö} \textit{s.} -- curar, convalescer.

\textbf{séptö'ö} \textit{v. du./pl.} -- curar, sarar.

\textbf{séptö'özé} \textit{s.} -- cura, recuperação.

\textbf{séptörö} \textit{v. sing.} -- curar, sarar.

\textbf{sépu} \textit{s.} -- doente, doença.

\textbf{sépu'u} \textit{v} -- adoecer, ficar doente.

\textbf{sépu'u'wa} \textit{s.} -- \textbf{ĩsépu'u'wa} vírus, bacilo, provocador de doença, causador de doença.

\textbf{sẽrẽ} \textit{v. sing.} -- = \textbf{ĩza} \textit{v. du.} meter, colocar.

\textbf{séré} \textit{s.} -- cabelo.

\textbf{sé're} \textit{s.} -- bexiga.

\textbf{sẽrẽzé} \textit{s.} -- lugar de repouso, túmulo.

\textbf{sẽrẽ ẽtẽ're} \textit{v./s.} -- colocar no túmulo, sepultar.

\textbf{séré'manharĩ} \textit{s.} -- \textbf{ĩsé'ré'manharĩ} brocha.

\textbf{séré'õ} \textit{adj. comp.} -- \textbf{ĩséré'õ} calvo, careca.

\textbf{séré 'rã'ru} \textit{s.} -- cabelo ondulado, pixaim.

\textbf{séro'ope nhisédézé} \textit{s.} -- \textbf{ĩséro'ope nhisédézé} carburador.

\textbf{séto} \textit{s.} -- \textbf{ĩséto} óleo lubrificante.

\textbf{sé si'uwazi} \textit{s.} -- \textbf{ĩsé si'uwazi} cabo de aço.

\textbf{sé'wa} \textit{adj.} -- cruel.

\textbf{séwaipo} \textit{s.} -- broto de macaúba.

\textbf{séwaipo nhorõ} \textit{s.} -- casa de broto de macaúba.

\textbf{séwawẽ} \textit{s. aum.} -- \textbf{ĩséwaẽ} soro.

\textbf{si-} \textit{inf.} reflexivo; -- a si mesmo.

\textbf{si} \textit{s.} -- ave.

\textbf{si} \textit{adv.} -- só, somente; \textbf{misi si} apenas um.

\textbf{si} \textit{v. du.} -- comer; \textbf{si wa'aba} comei vocês dois duas coisas.

\textbf{si'a} \textit{s.} -- galinha.

\textbf{si'ã} \textit{v.} = \textbf{si'ãsi} \textit{v} -- andar torcendo, andar desviando, driblar.

\textbf{si'aba'ré} \textit{v. pl.} -- = \textbf{dane} \textit{v. du.} ir, andar, caminhar.

\textbf{si'azöri} \textit{v. sing.} = \textbf{dasi'ahöri} \textit{v. du.} -- lutar, brigar.

\textbf{si'ahöri} \textit{s.} -- -- luta, briga, batalha, contenda.

\textbf{si'ahöri} \textit{v. du./pl.} -- -- lutar, brigar, batalhar.

\textbf{si'ahörizéb da} \textit{s./posp.} -- \textbf{ĩsi'ahörizéb de} bélico.

\textbf{si'ahöri'ézé} \textit{s.} -- espada.

\textbf{si'ahöri'wa} \textit{s.} -- soldado, guerreiro, policial, combatente.

\textbf{si'ahöri'wa nhorõwa} \textit{s.} -- caserna, quartel.

\textbf{si'ahöröre} \textit{vmp. dim.} -- cacarejar.

\textbf{si'a hörö wi} \textit{s./posp.} -- ao cantar do galo, ao amanhecer, de madrugada.

\textbf{si'anhi wa'u} \textit{s.} -- caldo de galinha, canja.

\textbf{si'are} \textit{s.} -- ovo de galinha.

\textbf{si'atarẽme'õ} \textit{vmp.} -- \textbf{ĩsi'atarẽme'õ} não sobrar, caber.

\textbf{si'ãsi} \textit{v} = \textbf{si'ã} -- andar em curva, andar desviando, driblar, coroar.

\textbf{si'a'wa} \textit{s.} -- que fazem os anciãos.

\textbf{si'awawẽ} \textit{s. aum.} -- peru.

\textbf{sib'a} \textit{s.} -- \textbf{ĩsib'a} constipação, gripe.

\textbf{siba} \textit{v.} -- compadecer-se, ter amor.

\textbf{siba} \textit{s.} -- \textbf{ãma dasiba} -- compaixão.

\textbf{siba s.} -- garça.

\textbf{siba'aro} \textit{s.} -- lugar das garças; Aldeia Garças.

\textbf{siba'awawẽ} \textit{s. aum.} -- cegonha.

\textbf{sibabarã} \textit{posp. refl.} -- atrás de si.

\textbf{sibaze} \textit{vmp.} -- ficar com pena, ter piedade.

\textbf{sibaze} \textit{s.} -- piedade, compaixão, dó.

\textbf{sib'apẽ'ẽhö} \textit{s.} -- asma.

\textbf{sib'a'uwẽ} \textit{s.} -- homem dele, súbdito, seguidor, discípulo, adepto, povo, clã, grupo.

\textbf{sibzari} \textit{s.} -- oferta, bondade, presente, dom.

\textbf{sibzari wẽ} \textit{s.} -- bondade para dar.

\textbf{sibzeru} \textit{s. neo.} -- dinheiro.

\textbf{sibzibi} \textit{s.} -- gengibre.

\textbf{sib'ézé} \textit{s.} -- faca.

\textbf{sib'ézébre} \textit{s.} -- faquinha.

\textbf{sib'ézé 'wapa} \textit{s.} -- espada, facão.

\textbf{sib'ĩhö'a} \textit{s.} -- chefe, cacique, liderança, dirigente.

\textbf{sibi} \textit{s.} -- aranha.

\textbf{sibi nhi'manharĩ} \textit{s.} -- teia de aranha.

\textbf{sibö} \textit{s.} -- rabada de pássaro.

\textbf{sib'ö} \textit{s.} -- água da pessoa.

\textbf{sib'rã} \textit{s.} -- cordão umbilical.

\textbf{sib'rada} \textit{s.} -- mão.

\textbf{sib'ratahi} \textit{s.} -- osso do dedo.

\textbf{sib'ratapo} \textit{s.} -- palma da mão.

\textbf{sib'rata prabazé} \textit{s.} -- corrimão.

\textbf{sib'ri} \textit{s.} -- casa própria, casa que ele fez.

\textbf{sibro} \textit{s.} -- bens, lugar dele, pertences, propriedade.

\textbf{sibro'ahö} \textit{adj. comp.} -- rico.

\textbf{sibrob'õ} \textit{s.} -- cigano, vagabundo, pobre, pobre sem nada.

\textbf{sibrope} \textit{adj. comp.} -- rico.

\textbf{sibroire} \textit{s.} -- bugiganga.

\textbf{sib'ru} \textit{s.} -- ira, raiva, rancor, coragem.

\textbf{sib'ru} \textit{adj.} -- irado, raivoso, corajoso, colérico.

\textbf{sibu} \textit{adj.} -- lotado.

\textbf{sib'uwa} \textit{s.} -- fraqueza.

\textbf{sib'uware} \textit{adj. dim.} -- fraco.

\textbf{sidaimama} \textit{s.} -- cunhado da mulher.

\textbf{sidana} \textit{s.} -- cunhada do homem.

\textbf{sidi} \textit{v.} -- fiar, brilhar.

\textbf{sidö} \textit{v.} -- fechar, tampar.

\textbf{sidözé} \textit{s.} -- \textbf{ĩsidözé} tampa, fechadura.

\textbf{sidöpösi} \textit{adv.} -- sempre, toda vez.

\textbf{sidotodo} \textit{v.} -- \textbf{ĩsidotodo} ferver, borbulhar, jorrar.

\textbf{sidototo} \textit{v} -- \textbf{ĩsidotodo} ferver, borbulhar, jorrar.

\textbf{sidupu} \textit{v.} -- apito de duas taquaras.

\textbf{sidurudu} \textit{s.} -- \textbf{ĩsidurudu} bolha.

\textbf{sizahi} \textit{s.} -- ave de rapina, pássaro de rapina.

\textbf{sizaribi} \textit{s.} -- penacho de pássaro.

\textbf{sizé} \textit{s.} -- \textbf{ĩsizé} fumaça, neblina; \textbf{unhama nhizé} fumaça.

\textbf{sizö} \textit{s.} -- cascavel.

\textbf{sizö} \textit{v. sing.} -- \textbf{sizöri} = \textbf{sihöri} \textit{v. du./pl.} cortar.

\textbf{sizöri} \textit{v. sing.} -- \textbf{sihöri} \textit{v. du./pl.} cortar.

\textbf{si'éi'rata'wa} \textit{s.} -- primeiro guia.

\textbf{si'éré} \textit{s.} -- trançado de flechas.

\textbf{sihi} \textit{v.} -- cambalear.

\textbf{sihö} \textit{v.} -- rir, sorrir.

\textbf{sihöbö} \textit{v.} -- encurvar-se.

\textbf{sihödö} \textit{s.} -- riscado, escrito, carta, escrita.

\textbf{sihödö} \textit{v.} -- riscar, escrever.

\textbf{sihö za'ra} \textit{v./des.} -- combater, combate.

\textbf{sihözé} \textit{s.} -- coisa que combate, antídoto; \textbf{dawa're nhihözé} remédio.

\textbf{sihöibato} \textit{vmp.} -- juntar em casamento.

\textbf{sihöi'ré} \textit{v.} -- aparecer, comparecer.

\textbf{sihöiwada} \textit{s.} -- prepúcio.

\textbf{sihöiwawẽ} \textit{s. aum.} -- concha.

\textbf{sihöiwi} \textit{s.} -- pinguim.

\textbf{sihöpö} \textit{v} -- encurvar-se.

\textbf{sihöri} \textit{v. du./pl.} -- cortar.

\textbf{sihörirã} \textit{s.} -- nome xavante.

\textbf{sihöri'wa} \textit{s.} -- cortador.

\textbf{sihötö} \textit{v} -- riscar, escrever.

\textbf{sihötö'ahörizé} \textit{s.} -- máquina de escrever.

\textbf{sihötö zamarĩ} \textit{s.} -- cópia.

\textbf{sihötözamarĩ} \textit{s.} -- fotocópia, xerox.

\textbf{sihötözurizé} \textit{s.} -- editora.

\textbf{sihötö'maprabazé} \textit{s.} -- correio.

\textbf{sihötönhoré} \textit{s.} -- livro.

\textbf{sihötörobtizé} \textit{s.} -- carimbo.

\textbf{sihötö'ru} \textit{s.} -- tarefa.

\textbf{sihötö'ru ubumrozé} \textit{s.} -- \textbf{ĩsihötö'ru ubumrozé} cartório.

\textbf{sihötötébrézé} \textit{s.} -- copiadora.

\textbf{sihötösisamro} \textit{s.} -- telex.

\textbf{sihötö upto} \textit{s.} -- borrão.

\textbf{sihötö'wa} \textit{s.} -- escritor, escrivão.

\textbf{sihötöwa'õno} \textit{s.} -- \textbf{ĩsihötöwa'õno} artigo, capítulo.

\textbf{sihötöwati'izé} \textit{s.} -- carimbo, tipografia.

\textbf{sihudu} \textit{s.} -- neto.

\textbf{sihutu} \textit{v. pl.} -- chegar = \textbf{dasimasisi} \textit{v. du.} alcançar, conseguir, vir.

\textbf{sihutuzé} \textit{s.} -- chegada.

\textbf{si'i} \textit{s.} -- \textbf{ĩsi'i} falha de grãos na espiga de milho.

\textbf{si'ĩhi} \textit{v.} -- envelhecer.

\textbf{si'ĩpe} \textit{v.} -- aperfeiçoar-se, tornar-se santo.

\textbf{si'ĩpese} \textit{v} -- \textbf{si'ĩpe} aperfeiçoar-se.

\textbf{sima} -- para si, consigo.

\textbf{si'madö'özé} \textit{s.} -- espelho.

\textbf{simahudu} \textit{v.} -- atrasar, demorar.

\textbf{simahudu'õ} \textit{adj. comp.} -- não demorado, não atrasado, pontual.

\textbf{sima ĩ'ru} \textit{pron./s.} -- capricho, empenho.

\textbf{simanapité} \textit{s.} -- xexeu (\textit{Cacicus cela}).

\textbf{simani} \textit{v.} -- fugir, perder-se em algum lugar.

\textbf{simanizé} \textit{s.} -- \textbf{ĩsimanizé} fuga.

\textbf{sima rowasu'u} \textit{pron./s.} -- colóquio, conversa.

\textbf{simasa} \textit{v. pl.} -- ficar de pé, encontrar-se.

\textbf{simasabzé} \textit{s.} -- \textbf{ĩsimasabzé} estábulo, curral, lugar de reunião, praça, páteo.

\textbf{simasi} \textit{v. sing.} -- \textit{v. du.} ficar, estar, comparecer, chegar, sentar-se.

\textbf{simasisi} \textit{v. du.} -- ficar, estar, comparecer, chegar, sentar-se.

\textbf{simasisizé} \textit{s.} -- cadeira, assento.

\textbf{simasisizé pa} \textit{s.} -- banco (de se sentar).

\textbf{simasisizé waipo} \textit{s.} -- trono.

\textbf{simasisi u'ösi'wa} \textit{s.} -- \textbf{ĩsimasisi u'ösi'wa} cliente.

\textbf{sima'uri} \textit{v.} -- esconder-se; \textbf{dazô dasima'uri} cilada, esconder em busca de.

\textbf{sima'uri} \textit{v.} -- \textbf{ĩsima'uri} esconder algo.

\textbf{sima'uri} \textit{adj.} -- escondido.

\textbf{simawẽ} \textit{vmp.} -- querer, gostar, preferir.

\textbf{simawẽ upa} \textit{vmp./s.} -- cobiça.

\textbf{sime} \textit{v. sing.} -- = \textbf{dasiwabzuri} \textit{v. du.} jogar-se, meter-se, deitar-se.

\textbf{sime} \textit{} -- -- junto de; \textbf{aime na} ajude!.

\textbf{simhi} \textit{s.} -- garrafa, casulo de inseto.

\textbf{simhörö'é'é} \textit{s.} -- gaita.

\textbf{simhu} \textit{s.} -- vara, chicote; \textbf{ĩhö'a nhimhu} cetro.

\textbf{simizadaihu} \textit{v.} -- entender, atender por palavras.

\textbf{simizadaihu'u} \textit{v} -- entender, atender por palavras.

\textbf{simizaze} \textit{v.} -- crer, acreditar.

\textbf{simizaze} \textit{s.} -- fé, crença, convicção, confiança.

\textbf{simizazezé} \textit{s.} -- credo, fé.

\textbf{simizaze'õ} \textit{adj. comp.} -- \textbf{ĩsimizaze'õ} infiel.

\textbf{simizazöri} \textit{v. sing.} -- \textit{v. du.} parar, brecar.

\textbf{simizahöri} \textit{v. du./pl.} -- parar, brecar.

\textbf{simizahörizé} \textit{s.} -- parada, lugar de pousar.

\textbf{simizama} \textit{s.} -- animal doméstico, submisso, escravo.

\textbf{simiza'õno} \textit{s.} -- cotovelo.

\textbf{simiza're} \textit{s.} -- razão, sabedoria, cautela.

\textbf{simiza're} \textit{adj.} -- esperto.

\textbf{simiza'rese} \textit{s.} -- razão, sabedoria.

\textbf{simizawi} \textit{s.} -- benevolência, caridade, amor.

\textbf{simizawipe} \textit{s.} -- clemência, cordialidade, piedade.

\textbf{simizawipe} \textit{adj. comp.} -- clemente, benigno, manso.

\textbf{simizu} \textit{s.} -- pulso da mão.

\textbf{simizu} \textit{s.} -- \textbf{ĩsimizu} cacho, penca.

\textbf{simizu 'rare} \textit{s.} -- cacho pequeno.

\textbf{simizutoto} \textit{s.} -- pulso (batida).

\textbf{simizusi} \textit{v. du.} -- colocar de pé.

\textbf{simizu sihöri} \textit{s.} -- mão destroncada.

\textbf{simi'ẽ} \textit{adj.} -- aplicado, diligente.

\textbf{simi'e} \textit{adj.} -- a mão esquerda, a esquerda.

\textbf{simihö} \textit{s.} -- discórdia, briga, rixa.

\textbf{simihö} \textit{adj.} -- \textbf{ĩsimihö} briguendo.

\textbf{simihöze} \textit{s.} -- briga, gosto pela briga.

\textbf{simihöpãrĩ} \textit{s.} -- espírito mau.

\textbf{simihö're} \textit{s.} -- câncer.

\textbf{simimnha} \textit{adj.} -- perigoso.

\textbf{siminhasi} \textit{adj.} -- perigoso.

\textbf{siminhihörö} \textit{s.} -- \textbf{ĩsiminhihörö} trombeta.

\textbf{siminho're} \textit{s.} -- glândula.

\textbf{siminho'redupu} \textit{s.} -- cachumba.

\textbf{siminho'ru} \textit{s.} -- inveja, ciúme.

\textbf{simi'ö} \textit{s.omp} -- veneno.

\textbf{simi'ötede'wa} \textit{s.} -- bruxa, feiticeiro.

\textbf{simipari} \textit{v.} -- esperar.

\textbf{simipari} \textit{s.} -- esperança.

\textbf{simiparizé} \textit{s.} -- esperança, lugar de esperar.

\textbf{simi'rãmi} \textit{s.} -- arrepio, calafrio; \textbf{dahöinhimi'rãmi} \textit{s.} alergia, calafrio.

\textbf{simire} \textit{v.} -- estar disposto, a mão direita, a direita.

\textbf{simiré'é} \textit{v.} -- contar exemplos.ontar histórias.

\textbf{simirẽme} \textit{v.} -- deixar, abandonar.

\textbf{simirẽme'õ} \textit{adj. comp.} -- fiel, que não abandona.

\textbf{simirésizé} \textit{s.} -- máquina.

\textbf{simirowa'a} \textit{s.} -- luz da pessoa.

\textbf{simi'ru} \textit{v.} -- instar, insistir, mandar.

\textbf{simi'ru} \textit{s.} -- inimizade, ódio, ordem.

\textbf{simisaré} \textit{adv.} -- juntamente, prontamente, repentino, imediato.

\textbf{simisõ} \textit{s.} -- lavação, ablução das mãos.

\textbf{simisutu} \textit{v.} -- reunir.

\textbf{simisutu} \textit{adj.} -- juntos, todos.

\textbf{simi'ubumro} \textit{s.} -- discípulo.

\textbf{simi'uri} \textit{v.} -- \textbf{ĩsimi'uri} esconder, camuflar, extraviar.

\textbf{simi'wa} \textit{s.} -- casco, unha de animal.

\textbf{simiwazé} \textit{s.} -- respeito.

\textbf{simiwapto} \textit{s.} -- cama, acampamento.

\textbf{simiwãrã} \textit{s.} -- dores de parto.

\textbf{simi'wara} \textit{v.} -- parar, pousar, estar de pé.

\textbf{simiwẽ} \textit{s.} -- benevolência.

\textbf{simiwẽ} \textit{v.} -- gostar com preferência, pedir esmola, otimista.

\textbf{simnhana} \textit{v. pl.} -- = \textbf{daza'o} \textit{v. du.} pendurar.

\textbf{simnhatã} \textit{v. pl. c.} -- = \textbf{daza'o} \textit{v. du.} pendurar.

\textbf{simnhasi} \textit{v.} -- pedir a quem tem, querer, confiar, desconfiar.

\textbf{simnhi'ã} \textit{s.} -- arco, arma.

\textbf{simnhihörö} \textit{s.} -- melodia, concerto.

\textbf{simnhihörözé} \textit{s.} -- instrumento de banda.

\textbf{simnhihörö'ubumro} \textit{s.} -- banda musical.

\textbf{simnhihörö'wa} \textit{s.} -- músico.

\textbf{simnhihörö wapri} \textit{s.} -- \textbf{ĩsimnhihörö wapri} corneta.

\textbf{simnhohu} \textit{s.} -- padrinho, madrinha.

\textbf{si'maropari} \textit{v.} -- tropeçar.

\textbf{simrami} \textit{v. du.} -- abraçar para a luta; \textbf{dapo're} esquecer.

\textbf{simri} \textit{v.} -- \textbf{ĩsimri} jogar pedra, balear, atirar, flechar.

\textbf{simro} \textit{v. pl.} -- levar, guiar, trair, conduzir, matar.

\textbf{simro} \textit{v. pl.} -- = \textbf{ubumro} = \textbf{dasimasisi} \textit{v. du.} reunir, sentar, ficar.

\textbf{sina} \textit{v.} -- continuar, juntar.

\textbf{sina} \textit{adv.} de gerúndio; fazer continuamente, continuar a fazer, costumar de fazer.

\textbf{si na} \textit{v. du.} -- comam! = \textbf{si} \textit{v. du.} comer.

\textbf{sina} \textit{s.} -- promessa; \textbf{dasina damrozé} casamento, bodas; \textbf{dasina ropisudu} promessa pública; \textbf{dasina damro} casamento, bodas.

\textbf{sinari} \textit{v.} -- arrancar, rasgar, despedacar, tirar.

\textbf{sine} \textit{adj.} -- semelhante a ele mesmo, parecido, como.

\textbf{sinha} \textit{s.} -- lugar a parte.

\textbf{sinhama} \textit{s.} = \textbf{sizö} -- cascavel.

\textbf{sinho'ra} \textit{s.} -- grupo etário mais novo.

\textbf{sinho'reza'a} \textit{s.} -- raposa, espécie de pássaro.

\textbf{siniwi} \textit{posp.} -- do lado de.

\textbf{si'odo} \textit{adj.} -- \textbf{ĩsi'odo} encurvado, torto.

\textbf{sinhosé'éré} \textit{s.} -- quero-quero.

\textbf{si'õmo} \textit{s.} = \textbf{si'u} -- gavião.

\textbf{si'õmozé} \textit{s.} -- alça.

\textbf{si'õno} \textit{s.} -- cesto xavante, baquité.

\textbf{si'oto} \textit{adj. c.} -- \textbf{ĩsi'odo} torto, encurvado.

\textbf{si'õtõ} \textit{s.} -- \textbf{si'õno} -- cesto xavante, baquité.

\textbf{si'õtõ höpö} \textit{s.} -- cesto grande, baquité grande.

\textbf{sipa} \textit{adj.} -- \textbf{ĩsipa} acima de, por cima de, superior, vencedor; \textbf{ĩsipai u} acima de, mais que.

\textbf{sipa} \textit{s.} -- \textbf{ĩsipa} = \textbf{upa} erro, engano; \textbf{ĩsipai ãna} sem engano, sem sobressair.

\textbf{sipahudu} \textit{s.} -- urubu.

\textbf{sipahuture} \textit{s.} -- corvo.

\textbf{sipahutuwawẽ} \textit{s. aum.} -- condor.

\textbf{sipapara} \textit{posp.} -- debaixo de si mesmo.

\textbf{siparabu} \textit{s.} -- bate-pé no "wai'a".

\textbf{sipãrĩze} \textit{s.} -- \textbf{ĩsipãrĩze} bandido, assassino.

\textbf{sipa'uwati} \textit{v.} = \textbf{dapa'uwati} dissuadir, desaconselhar.

\textbf{sipe} \textit{adj.} -- engenhoso.

\textbf{sipé} \textit{v.} -- dividir, espalhar, dissipar.

\textbf{sipi} \textit{v.} -- cozinhar.

\textbf{sipizari} \textit{v. du.} -- virar, girar, rolar.

\textbf{sipi'ra} \textit{v. sing.} -- = \textbf{dasipizari} \textit{v. du.} -- virar-se, voltar-se.

\textbf{sipisu} \textit{s.} -- que faz coisas depressa.

\textbf{sipo} \textit{s.} -- unha.

\textbf{sipo} \textit{v.} -- \textbf{ĩsipo} rachar, madurar.

\textbf{sipodo} \textit{s.} -- \textbf{ĩsipodo} célula.

\textbf{sipo'o} \textit{v} -- \textbf{ĩsipo} rachar, madurar, dividir.

\textbf{sipra} \textit{v.} -- depender, dependurar.

\textbf{sipra} \textit{v.} -- \textbf{ĩsipra} aderir, reforçar; \textbf{bötö sipra wi} de tarde.

\textbf{siprã} \textit{s.} -- que sabe fazer pouco.

\textbf{sip'ra} \textit{v.} -- \textbf{ĩsip'ra} feito pela própria mãe.

\textbf{sipré} \textit{s.} -- pintura nas têmporas.

\textbf{siprẽrẽ} \textit{s.} -- marimbondo amarelo.

\textbf{siptede} \textit{adj.} -- \textbf{ĩsiptede} forte.

\textbf{siptede} \textit{s.} -- força.

\textbf{siptete} \textit{adj. c.} -- \textbf{ĩsiptede} forte.

\textbf{siptetezé} \textit{s.} -- força, fortificante.

\textbf{siptete'wa} \textit{s.} -- quem apoia, quem dá força.

\textbf{sipti} \textit{adj.} -- \textbf{ĩsipti} escolha própria, amais.

\textbf{sipti} \textit{v.} -- acrescentar.

\textbf{sipti} \textit{s.} -- córrego, cabeceira, ramal; \textbf{wedepanhipti} galho.

\textbf{siptire} \textit{s.} -- córrego.

\textbf{sipto} \textit{s.} -- dedo.

\textbf{siptõmo} \textit{s.} -- dedo.

\textbf{siptõmohi} \textit{s.} -- osso do dedo; \textbf{danhiptõ'mohi wasisizé} anel.

\textbf{siptoro} \textit{s.} -- flecha.

\textbf{sipsaihuri} \textit{v.} -- roubar.

\textbf{sipsaihuri} \textit{s.} -- objeto roubado, roubo.

\textbf{sipsaihuri} \textit{s.} -- ladrão, assaltante.

\textbf{sipsi} \textit{s.} -- pulseira de cordinha.

\textbf{sipsisizé} \textit{s.} -- amarra do pulso, bracelete.

\textbf{sipsisipré} \textit{s.} -- bracelete vermelho.

\textbf{sipu'u} \textit{s.} -- qualquer um, todo o mundo, variado, misturado.

\textbf{sirã} \textit{s.} -- \textbf{ĩsirã} rasto, pegada.

\textbf{sirã} \textit{s.} -- céu da boca.

\textbf{si'rã} \textit{s.} -- cesto para armazenar, pássaro preto.

\textbf{si'ra} \textit{v.} -- \textbf{ĩsi'ra} descer, abaixar.

\textbf{si'rãza'rõno} \textit{vmp.} -- \textbf{ĩsi'rãza'rõno} vingar-se.

\textbf{si'razöri} \textit{v. sing.} -- \textbf{ĩsi'rahöri} \textit{v. du./pl.} cruzar.

\textbf{si'rahöri} \textit{v. du./pl.} -- \textbf{ĩsi'rahöri} cruzar.

\textbf{si'rãi'ré} \textit{s.} -- tuiuiú (\textit{Jabiru mycteria}).

\textbf{si'rãi'ré amo} \textit{s.} -- cegonha.

\textbf{si'rãmi} \textit{v.} -- levar susto, comover-se, espantar-se, cismar.

\textbf{si'rãmi} \textit{s.} -- espanto, susto, choque.

\textbf{si'rãmi} \textit{adj.} -- \textbf{ĩsi'rãmi} espantoso, chocante.

\textbf{si'rã'õno} \textit{v.} -- reunir-se.

\textbf{si'rã'õno} \textit{s.} -- reunião, comício.

\textbf{si'rã'õtõzé} \textit{s.} -- lugar de reunião, lugar da assembleia, templo.

\textbf{si'rã pari} \textit{s./v.} -- \textbf{ĩsi'rã pari} cortar cabelo.

\textbf{sirãrã} \textit{s.} -- \textbf{ĩsirãrã} flor.

\textbf{sirãrã'ã} \textit{v.} -- florescer.

\textbf{sirãrã'rã} \textit{v.} -- florescer.

\textbf{si'rata} \textit{posp.} -- perto de si mesmo.

\textbf{sire} \textit{s.} -- pássaro, passarinho.

\textbf{siré} \textit{posp.} com, junto com; \textbf{dasiré} junto com a gente; \textbf{ĩsiré} comigo; \textbf{asiré} contigo; \textbf{siré} consigo.

\textbf{siré} \textit{s.} -- colega, companheiro, camarada; \textbf{dasiré romhuri} cooperar, cooperação.

\textbf{si'ré} \textit{v.} -- = \textbf{ai'ré} ir, afastar-se, separar, crescer, avançar, progredir.

\textbf{si'ré} \textit{posp.} -- \textbf{ĩsi'ré} separado, dividido.

\textbf{sirẽ} \textit{s.} -- tonsura, coroa.

\textbf{siré dahöimana} \textit{s.} -- coabitar, comunidade.

\textbf{siré daro} \textit{posp.} -- conterrâneo, terra comum.

\textbf{siré darob 'remhã} \textit{posp./posp.} -- compatriota.

\textbf{sirẽme} \textit{v.} -- separar, abandonar.

\textbf{si'rẽne} \textit{v. sing.} -- = \textbf{dasiwapsisi} \textit{v. du.} bailar, dançar.

\textbf{si'rẽne} \textit{s.} -- baile, dança.

\textbf{si'ré'õ} \textit{adj. comp.} -- \textbf{ĩsi'ré'õ} não separado, não dividido, cheio.

\textbf{siré romhuri'wa} \textit{posp.} -- \textbf{ĩsiré romhuri'wa} colaborador.

\textbf{si'ré'wa} \textit{s.} -- do outro clã.

\textbf{siri} \textit{s.} -- coração.

\textbf{siri na hã} \textit{s./posp./enf.} -- cardíaco.

\textbf{sirinhimizahöri} \textit{s.} -- enfarte.

\textbf{sirisi'rãmi} \textit{s.} -- ataque cardíaco.

\textbf{siro} \textit{adv.} particípio do passado; já realizado, pronto, sempre existente.

\textbf{sirobo} \textit{s.} -- constipação.

\textbf{sirobo} \textit{s.} -- \textbf{ĩsirobo} espuma; \textbf{sinhirobo} penugem.

\textbf{siroma} \textit{s.} -- o branco (civilizado).

\textbf{siromo} \textit{adv} = \textbf{siro} -- particípio passado; já realizado, pronto, sempre existente.

\textbf{sirõno} \textit{s.} -- \textbf{ĩsirõno} fumaça, neblina; \textbf{höiwa nhirõno} nuvem.

\textbf{sirõrĩ} \textit{v.} -- aparar penas da flecha.

\textbf{sirõtõ} \textit{s.} -- \textbf{ĩsirõno} fumaça, neblina.

\textbf{si'ru} \textit{v.} -- \textbf{ĩ'ru} \textbf{ti'ru} pedir, solicitar, oferecer-se.

\textbf{si'ruiwa'öbö} \textit{v.} -- vingar.

\textbf{si'ruiwapari} \textit{v.} -- arrepender-se.

\textbf{si'ruiwapari} \textit{s.} -- arrependimento, contrição.

\textbf{si'ruiwaparizé} \textit{s.} -- arrependimento.

\textbf{sisai're} \textit{} -- . . . .

\textbf{sité} \textit{v.} -- segurar-se, unir-se, renovar, repetir; \textbf{õ norĩ hã, te sitéb za'ra} eles se unem, renovam.

\textbf{sitébré} \textit{v.} -- aumentar, procriar, proliferar.

\textbf{sitébrézé} \textit{s.} -- genitais, sexo.

\textbf{sitéhã} \textit{s.} -- bem-te-vi.

\textbf{siti} \textit{posp.} -- longe de, afastado de.

\textbf{siti'a} \textit{v.} -- tornar-se terra.

\textbf{siti'ru} \textit{v.} -- irar-se, zangar-se.

\textbf{siti'ru} \textit{s.} -- ira, rancor.

\textbf{siti'ru} \textit{adj.} -- irado, zangado, mal-humorado.

\textbf{sito} \textit{v.} -- \textbf{ĩsito} fechar, cerrar.

\textbf{sitobzé} \textit{s.} -- \textbf{ĩsitobzé} fechadura.

\textbf{sitobzé} \textit{s.} -- cárcere, prisão, cativeiro.

\textbf{sitob'ézé} \textit{s.} -- \textbf{ĩsitob'ézé} chave.

\textbf{sitob're} \textit{s.} -- banheiro, privada, WC.

\textbf{sitob'ru} \textit{s.} -- inimigo.

\textbf{sitobtetezé} \textit{s.} -- \textbf{ĩsitobtetezé} cofre.

\textbf{sitobwati} \textit{s.} -- \textbf{ĩsitobwati} cadeado.

\textbf{sitowa} \textit{v.} -- \textbf{ĩsitowa} abrir.

\textbf{sisa} \textit{s.} -- comida de pássaro.

\textbf{sisada} \textit{s.} -- \textbf{ĩsisada} carga para gente.

\textbf{sisada'öbö} \textit{v.} -- discutir, disputar, responder.

\textbf{sisahu} \textit{s.} -- batalha.

\textbf{sisãmri} \textit{v.} -- pensar em si mesmo, examinar-se.

\textbf{sisãmri} \textit{s.} -- contato, consciência.

\textbf{sisãmrizé} \textit{s.} -- consciência.

\textbf{sisamro} \textit{v. du.} -- correr; \textbf{dapẽ'ẽ sisamro} entristecer-se, assustar-se.

\textbf{sisamro} \textit{s.} -- corrida, viagem.

\textbf{sisamrozé} \textit{s.} -- condução.

\textbf{sisãnawã} \textit{s.} -- irmão, irmã, parente.

\textbf{sisanho} \textit{v.} -- dar exemplo, ensinar.

\textbf{sisanhozé} \textit{s.} -- exemplo, ensino.

\textbf{sisaprĩ} \textit{v.} -- transformar.

\textbf{sisa're} \textit{v. pl.} -- = \textbf{dasisamro} \textit{v. du.} correr.

\textbf{sisaré} \textit{v.} -- trair.

\textbf{sisa're'ra} \textit{v.} -- fazer na ausência, fazer por pirraça.

\textbf{sisa'ridi} \textit{s.} -- forquilha.

\textbf{sisarõno} \textit{v.} -- reclamar, elevar-se, elogiar-se, gabar-se.

\textbf{sisarõno} \textit{s.} -- dança, pulo, brio, soberba.

\textbf{sisasisi} \textit{s.} -- \textbf{ĩsisasisi} barulho, confusão.

\textbf{sisa'u} \textit{posp.} -- \textbf{ĩsisa'u} atrás de si mesmo.

\textbf{sisawi} \textit{v.} -- recusar-se, negar-se.

\textbf{sisé} \textit{s.} -- vergonha, pudor, respeito.

\textbf{sisé} \textit{adj.} -- vergonhoso, respeitoso; \textbf{siséb di} é respeitoso.

\textbf{siséb'õ} \textit{adj. comp.} -- \textbf{ĩsiséb'õ} desrespeitoso.

\textbf{sisédé} \textit{s.} -- \textbf{ĩsisé} fumaça.

\textbf{sisédé pusizé} \textit{s.} -- \textbf{ĩsisédé pusizé} chaminé.

\textbf{sisi} \textit{v.} -- encher; \textbf{'masisi} encher uma só vez.

\textbf{sisi} \textit{s.} -- nome.

\textbf{sisi} \textit{v.} -- chamar-se.

\textbf{sisi} \textit{posp.} -- \textbf{ĩsisi} por cima de, encima de.

\textbf{sisi'ã} \textit{s.} -- curva, coroa, colar de algodão.

\textbf{sisizé} \textit{s.} -- \textbf{ĩsisizé} nomeação.

\textbf{sisihunhiptede} \textit{s.} -- elefante.

\textbf{sisimini} \textit{v.} -- perder-se, perverter-se.

\textbf{sisimri} \textit{v.} -- lançar, atirar, jogar (=contra si mesmo).

\textbf{sisi're} \textit{s.} -- nariz.

\textbf{sisiti} \textit{posp.} -- \textbf{ĩsisiti} longe de si mesmo.

\textbf{sisisi} \textit{v. pl.} -- = \textbf{dazasi} \textit{v. du.} entrar.

\textbf{sisisizé} \textit{s.} -- porta, entrada.

\textbf{sisi nhipti} \textit{s.} -- sobrenome.

\textbf{sisi u} \textit{posp.} -- no alto de, encima de.

\textbf{sisiwi} \textit{posp.} -- encima de, por cima de, acima de.

\textbf{sisõ} \textit{v.} -- banhar-se.

\textbf{sisõ'a} \textit{adj.} -- \textbf{ĩsisõ'a} entregue.

\textbf{sisodadu} \textit{s. neo.} -- soldado, militar.

\textbf{sisõmri} \textit{v.} -- entregar-se.

\textbf{sisõpãrĩ} \textit{v. du.} -- caçar (animal que mata).

\textbf{sisõpẽne} \textit{s.} -- contacto, encontro.

\textbf{sisõpẽtẽ} \textit{s.} -- contacto, encontro.

\textbf{sisõwa} \textit{posp.} -- antes de, adiante de.

\textbf{sisu} \textit{s.} -- folha de buriti.

\textbf{sisu} \textit{v.} -- colocar penas em flecha.

\textbf{sisuwa} \textit{v.} -- cobrir-se, encobrir.

\textbf{sisuwazé} \textit{s.} -- cobertor, esconderijo, "pistolão".

\textbf{situri} \textit{v.} = \textbf{sadawa situri} falando muita gente, barulho de conversa.

\textbf{si'u} \textit{s.} -- caldo, suco; \textbf{arãrã nhi'u} néctar.

\textbf{si'u} \textit{s.} -- gavião, carcará (\textit{Carcara plancus}).

\textbf{si'u} \textit{v.} -- juntar, fixar; \textbf{õ hã, ma si'u uba 're} amarrou a canoa.

\textbf{si'u'a} \textit{v.} -- bater-se, machucar-se ferindo.

\textbf{si'ubu} \textit{v.} -- esconder, camuflar, cobrir-se.

\textbf{si'ubuzi} \textit{adj.} -- precioso, brilhante.

\textbf{si'ubura} \textit{s.} -- redemoinho.

\textbf{si'udö} \textit{v.} -- enganar-se, iludir-se, mentir.

\textbf{si'udu} \textit{v.} -- contestar, retrucar.

\textbf{si'uza} \textit{v.} -- vestir-se.

\textbf{si'uzazé} \textit{s.} -- vestiário, placenta.

\textbf{si'uzé'a} \textit{s.} -- canário.

\textbf{si'u'ẽne} \textit{s.} -- convulsão.

\textbf{si'uhi} \textit{s.} -- corredor; \textbf{danebzé si'uhi} corredor.

\textbf{si'uihöna} \textit{adj.} -- sozinho.

\textbf{si'uiwa} \textit{adv.} -- juntamente, repetindo; \textbf{romhuri si'uiwa ropisudu} convênio.

\textbf{si'uiwana} \textit{adj.} -- lado ao lado, paralelo, repetindo; \textbf{maparanesi'uiwana} quatro.

\textbf{si'umdatõ} \textit{num.} -- três.

\textbf{si'umdatõnhito} \textit{s.} -- tria (jogo).

\textbf{si'u'õ} \textit{vmp.} -- enxugar-se.

\textbf{si'u'õzé} \textit{s.} -- \textbf{ĩsi'u'õzé} toalha.

\textbf{si'upari} \textit{v.} -- encostar-se, descansar, fazer merenda.

\textbf{si'upi} \textit{s.} -- pega-pega (jogo), convívio matrimonial, coito (=tocar-se).

\textbf{si'upte} \textit{s.} -- banho.

\textbf{si'upte} \textit{v.} -- tomar banho.

\textbf{si'uptezé} \textit{s.} -- chuveiro, lugar de banho, banheiro.

\textbf{si'upsibizé} \textit{s.} -- manto.

\textbf{si'upsõ} \textit{v.} -- lavar-se.

\textbf{si'upsõzé} \textit{s.} -- sabão, algo para lavar-se.

\textbf{si'upsõzé ĩsadaze} \textit{s.} -- \textbf{ĩsi'upsõzé ĩsadaze} sabonete.

\textbf{si'utõrĩ} \textit{v.} -- morrer, acabar.

\textbf{si'utõri'õ} \textit{adj.} -- \textbf{ĩsi'utõrĩ'õ} que não acaba, eterno.

\textbf{si'usu} \textit{s.} -- do mesmo grupo etário.

\textbf{si'uwa} \textit{adj.} -- leve, suave, tenro, macio.

\textbf{si'uwazi} \textit{s.} -- \textbf{ĩsi'uwazi} arame, fio, anel, cerca.

\textbf{si'uwazinhoré} \textit{s.} -- \textbf{ĩsi'uwazinhoré} cerca.

\textbf{si'uwazinhorõ} \textit{s.} -- \textbf{ĩsi'uwazinhorõ} cabo de aço.

\textbf{si'uwazi tob'ãi'ãsitede} \textit{s.} -- \textbf{ĩsi'uwazi tob'ãi'ãsitede} corrente.

\textbf{si'uwazi sisãmri} \textit{s.} -- \textbf{ĩsi'uwazi sisãmri} circuito.

\textbf{si'uwazi sitede} \textit{s.} -- \textbf{ĩsi'uwazi sitede} corrente.

\textbf{si'uwaziwamreme} \textit{s.} -- \textbf{ĩsi'uwaziwamreme} fio de linha telegráfica, cítara.

\textbf{si'uware} \textit{s.} -- anu branco.

\textbf{si'wa} \textit{s.} -- ferramenta de corte e ponta.

\textbf{siwa'aba} \textit{s. ad.} -- clareira, onde fica a anta.

\textbf{siwa'aba} \textit{v.} -- condenar, executar.

\textbf{siwabzuzé} \textit{s.} -- sexo, genitais, pênis, vulva, vagina.

\textbf{siwabzuri} \textit{s. du.} -- convívio matrimonial, coito.

\textbf{siwada} \textit{s.} -- bico de passarinho.

\textbf{siwada'uri} \textit{v.} -- exortar, animar, convidar.

\textbf{siwada'urizé} \textit{s.} -- exortação, convite.

\textbf{siwadi} \textit{v.} -- reunir-se, aproximar-se.

\textbf{siwadi} \textit{s.} -- \textbf{ĩsiwadi} conhecido, primo, parente.

\textbf{siwazari} \textit{v.} -- misturar-se; \textbf{a'uwẽ siwazari} caboclo.

\textbf{siwazusizé} \textit{s.} -- bengala.

\textbf{siwaihu'u} \textit{v.} -- conhecer-se, conscientizar-se.

\textbf{siwai'õ} \textit{s.} -- canto da festa de imposição de nome às mulheres.

\textbf{siwarĩ} \textit{v.} -- hesitar, duvidar, demorar, mudar, torcer, trocar.

\textbf{siwairõri} \textit{v.} -- fumar.

\textbf{siwairõrĩzé} \textit{s.} -- \textbf{ĩsiwairõrĩzé} cigarro, fumo.

\textbf{siwaiwẽrẽ} \textit{s.} -- canto para afastar doença.

\textbf{si'wamarĩzé} \textit{s.} -- brilhantina, coisa para enfeitar.

\textbf{siwamnarĩ} \textit{s.} -- estrago, quebra, ruína, loucura.

\textbf{siwamnarĩ} \textit{v.} -- estragar, quebrar, arruinar, enlouquecer.

\textbf{siwamnharĩ} \textit{s.} -- consternação.

\textbf{siwamreme} \textit{s.} -- avião.

\textbf{siwanhizari} \textit{v.} -- separar, afastar, apartar-se.

\textbf{siwa'õ} \textit{s.} -- luto, tristeza, separação.

\textbf{siwa'õno} \textit{s.} -- choro, luto, tristeza, separação.

\textbf{siwa'õno v.} -- chorar.

\textbf{si'wapé} \textit{v. du.} -- lutar, guerrear, concorrer, combater, competir.

\textbf{si'wapé} \textit{s.} -- luta, guerra, competição, combate, contenda, copa (de futebol); \textbf{romhuri na dasi'wapé} concorrência.

\textbf{siwaprosi} \textit{adj.} -- só, sozinho.

\textbf{siwapto} \textit{adj.} -- muitos, bastante, alguns.

\textbf{siwapsari} \textit{v.} -- queixar-se, murmurar, criticar, reclamar, resmungar; \textbf{dasiwapsari wasézé} blasfemar.

\textbf{siwapsi} \textit{v.} -- juntar, reunir, dançar sozinho.

\textbf{siwapsisi} \textit{v. du./pl.} -- dançar.

\textbf{siwa'rãmi} \textit{v.} -- contradizer-se.

\textbf{siwari} \textit{v.} -- pedir coisas.

\textbf{siwãri} \textit{v.} -- ajudar.

\textbf{si'wa'ri} \textit{s.} -- cocar.

\textbf{si'wa'rizé} \textit{s.} -- \textbf{ĩsi'wa'rizé} bucha, coisa para coçar-se.

\textbf{siwãrĩ'wa} \textit{s.} -- \textbf{ĩsiwãrĩ'wa} ajudante, auxiliar.

\textbf{siwa'ro} \textit{s.} -- \textbf{ĩsiwa'ro} calor, ardor.

\textbf{siwa'ro} \textit{v.} -- \textbf{ĩsiwa'ro} esquentar-se, aquecer-se.

\textbf{siwa'ru} \textit{adj.} -- comum, qualquer um.

\textbf{si'wa'rutu} \textit{v. du.} -- cansar-se, afatigar-se, muito ocupado.

\textbf{siwatasu pari} \textit{s./v.} -- barbear-se.

\textbf{si'waté} \textit{v.} -- purificar-se pela cerimônia de bate-água.

\textbf{siwatébré} \textit{s.} -- \textbf{ĩsiwatébré} filhote.

\textbf{siwatébrérã} \textit{s.} -- filho.

\textbf{siwati} \textit{v.} -- ficar apertado, vencer-se a si mesmo, dominar-se.

\textbf{siwati'i} \textit{v} -- \textbf{siwati} ficar apertado, vencer-se a si mesmo, dominar-se.

\textbf{siwasi} \textit{v.} -- livrar, misturar-se.

\textbf{siwasutu} \textit{v.} -- cansar-se, afatigar-se.

\textbf{siwasu'u} \textit{v.} -- falar de si mesmo, confessar, confessar-se, professar.

\textbf{si'wa'uburé} \textit{s.} -- festa, cerimônia (no culto xavante).

\textbf{si'wa'uburé} \textit{v.} -- celebrar (no culto xavante).

\textbf{si'wa'uburé nhopa} \textit{s.} -- dança do "wai'a" (culto xavante; tabu!).

\textbf{si'wa upsõzé} \textit{s.} -- escova de dente.

\textbf{si'wa'wa} \textit{s.} -- abertura no meio de.

\textbf{si'wawẽ} \textit{s. aum.} -- águia.

\textbf{siwẽ} \textit{s.} -- namorada.

\textbf{siwẽ'ẽ} \textit{v.} -- inclinar-se, disparar em corrida.

\textbf{sô} \textit{posp.} -- para, a, para ele (3ªpessoa ) em busca de.

\textbf{sõ} \textit{s.} -- nome de pássaro.

\textbf{sõ} \textit{v. sing.} = \textbf{sõmri} -- dar, doar, engolir, contribuir; \textbf{õ hã, ma tisõ} ele deu.

\textbf{sõ'a} \textit{v.} -- apresentar, elevar, revelar.

\textbf{sõ'a} \textit{s.} -- subida.

\textbf{sõ'a} \textit{posp.} -- em presença de.

\textbf{sõ'awi} \textit{adv.} -- abertamente.

\textbf{sõzaré} \textit{s.} -- \textbf{ĩsõzaré} caçula.

\textbf{sõzarési} \textit{s.} -- \textbf{ĩsõzaré} caçula.

\textbf{sõzi} \textit{v.} -- abrir, estender.

\textbf{sõ'ẽ} \textit{v.} -- desmatar.

\textbf{sõ'ẽ'wa} \textit{s.} -- trator de esteira.

\textbf{sõhu} \textit{v.} -- ser padrinho, apadrinhar.

\textbf{sõhui'wa} \textit{s.} -- padrinho.

\textbf{sõiba} \textit{s.} -- noiva.

\textbf{sõmo'a} \textit{adj.} -- aberto, afunilado, alargado.

\textbf{sõmri} \textit{v.} -- dar, doar, engolir, contribuir.

\textbf{sõmri} \textit{s.} -- contribuição.

\textbf{sõmrizé} \textit{s.} -- \textbf{ĩsõmrizé} dom, dádiva.

\textbf{sõmri mono} \textit{v./adv.} -- distribuir.

\textbf{sõnhi'ã} \textit{s.} -- \textbf{ĩsõnhi'ã} canga (brincadeira).

\textbf{sõnhi'ã} \textit{s.} -- clavícula.

\textbf{sõnhi'ã} \textit{s.} -- colar de algodão de "wapté".

\textbf{sõnhihöbö} \textit{s.} -- lado do peito.

\textbf{sõ'õmo} \textit{s.} -- \textbf{ĩsõ'õmo} esteio, pilar; \textbf{'ri nho'õmo} esteio da casa.

\textbf{sõ'õmo} \textit{posp.} -- \textbf{ĩsõ'õmo} em direção de, rumo a.

\textbf{sõ'õno} \textit{v.} -- enfrentar.

\textbf{sõ'õno} \textit{} -- . . . \textbf{dasina ĩsõ'õno} confortar, ter coragem.

\textbf{sõ'o'o} \textit{v. du./pl.} -- vomitar.

\textbf{sõ'oro} \textit{v. sing.} -- vomitar.

\textbf{sõ'õto} \textit{v} -- \textbf{sõ'õno} carregar.

\textbf{sõpãrĩ} \textit{v. du.} -- caçar.

\textbf{sõpẽne} \textit{s.} -- encontro.

\textbf{sõpẽne} \textit{v.} -- encontrar, achar.

\textbf{sõpẽtẽ} \textit{v} -- encontrar, achar.

\textbf{sõpre} \textit{s.} -- coroa.

\textbf{sõpre} \textit{v.} -- coroar.

\textbf{sõpré} \textit{v. pl.} -- procurar, refletir, olhar, ver.

\textbf{sõpru} \textit{adj.} -- generoso.

\textbf{sõprubzé} \textit{s.} -- generosidade.

\textbf{sõ'rada} \textit{s.} -- resto de comida, migalhas.

\textbf{so'rara} \textit{s.} -- araracanga.

\textbf{sõ'ra're're} \textit{s.} -- \textbf{ĩsõ'ra're're} devedor.

\textbf{sõré} \textit{v.} -- ler, enumerar, contar, enfileirar.

\textbf{sõré} \textit{s.} -- \textbf{ĩsoré} leitura, fila.

\textbf{sõ're} \textit{s.} -- canto, oração, garganta.

\textbf{sõ're} \textit{v.} -- cantar, orar, rezar.

\textbf{sõ're} \textit{posp.} -- em frente de.

\textbf{sõ're'a} \textit{s.} -- rouquidão.

\textbf{sõrézé} \textit{s.} -- fila, proclamação.

\textbf{so're hipãrĩ} \textit{s./v.} -- \textbf{ĩso're hipãrĩ} sangrar.

\textbf{sõrénhoré} \textit{s.} -- livro.

\textbf{sõ'reptu} \textit{v.} -- salvar, resgatar.

\textbf{sõ'reptuzé} \textit{s.} -- salvação.

\textbf{sõ'reptu'wa} \textit{s.} -- salvador.

\textbf{sõ're'ré} \textit{s.} -- garganta, gola.

\textbf{sõ'reti} \textit{s.} -- estrangulamento.

\textbf{sõ'reti} \textit{v.} -- estrangular, enforcar.

\textbf{sõ're'wa} \textit{s.} -- puxador de canto.

\textbf{sõrõ} \textit{s.} -- corda.

\textbf{sõrõ} \textit{v.} -- torcer corda.

\textbf{sõrõhöpré} \textit{s.} --  casca vermelha para arco.

\textbf{sõrõna'ratapré} \textit{s.} -- raiz vermelha.

\textbf{sõrõrã} \textit{s.} -- embira branca.

\textbf{sõrõ'rã} \textit{s.} -- embira preta.

\textbf{sõrõwa} \textit{s.} -- casa.

\textbf{sõ'ru} \textit{s.} -- inveja, ira, ciúme.

\textbf{sõté} \textit{s.} -- arara amarela, arara canindé (\textit{Ara ararauna}).

\textbf{sõti} \textit{adj.} -- mesquinho, avarento.

\textbf{sõtõ} \textit{v} -- dormir.

\textbf{sõtõzé} \textit{s.} -- lugar de dormir, cama, dormitório, cobertor.

\textbf{sõsi} \textit{s.} -- irmão segundo-nascido.

\textbf{sõ'u} \textit{s.} -- \textbf{ĩsõo'u} córrego, líquido, varjão.

\textbf{sõ'u} \textit{s.} -- \textbf{ĩsõ'u} . . . ; \textbf{bödödinho'u} avenida.

\textbf{sõ'udu} \textit{s.} -- peito, coração.

\textbf{sõ'uma} \textit{pron. ind.} -- todos.

\textbf{sõ'umimi} \textit{s.} -- penugem branca na barriga da ave.

\textbf{sõ'utu} \textit{s.} -- peito, coração.

\textbf{sõ'utu'ru} \textit{s.} -- coração.

\textbf{sõ'utu'u} \textit{s.} -- \textbf{ĩsõ'utu'u} capa, túnica.

\textbf{sõwa} \textit{posp.} -- em frente de, antes de, diante de.

\textbf{sõwa} \textit{s.} -- frente, barriga, abdômen.

\textbf{sõwada} \textit{s.} -- \textbf{ĩsõwada} cerco.

\textbf{su} \textit{v.} -- socar.

\textbf{su} \textit{s.} -- \textbf{ĩsu} suor, folha, página.

\textbf{subzazöri} \textit{v. sing.} = \textbf{subdazhöri} \textit{v. du./pl.} -- parar de limpar, parar de socar.

\textbf{subzahöri} \textit{v. du./pl.} -- parar de limpar, parar de socar.

\textbf{subzapodo} \textit{s.} -- \textbf{ĩsubzapodo} folha redonda.

\textbf{subzari} \textit{v.} -- carregar, acariciar.

\textbf{subzé} \textit{s.} -- \textbf{ĩsubzé} pilão, máquina de socar, máquina de limpar arroz, moinho.

\textbf{subrã} \textit{s.} -- \textbf{ĩsubrã} farinha branca de raiz.

\textbf{subre} \textit{v.} -- \textbf{zubre} socar.

\textbf{suzöri} \textit{v. sing.} = \textbf{suhöri} \textit{v. du.} -- cortar cacho.

\textbf{suhöri} \textit{v. du./pl.} -- cortar cacho.

\textbf{suire} \textit{s.} -- \textbf{ĩsuire} folha, página.

\textbf{supara} \textit{s.} -- areia.

\textbf{suparahi} \textit{s.} -- \textbf{ĩsuparahi} folha em forma de dedo de pé.

\textbf{supararo} \textit{s.} -- deserto.

\textbf{supréze} \textit{s.} -- \textbf{ĩsupréze} cacau.

\textbf{supréze'u'ẽne} \textit{s.} -- \textbf{ĩsupréze'u'ẽne} chocolate.

\textbf{su'rãzahi'ri} \textit{s.} -- \textbf{ĩsu'rãzahi'ri} raiz para feitiço.

\textbf{su'rã're} \textit{adj.} -- seco.

\textbf{su'rã're} \textit{v.} -- secar.

\textbf{suri} \textit{v.} -- \textbf{ĩzuri} brotar, germinar.

\textbf{surizé} \textit{s.} -- lugar de aparecer, nascer, sair, brotar.

\textbf{susu} \textit{s.} -- tipo de palmeira.

\textbf{sutu'rã} \textit{s.} -- glande.

\textbf{su'u} \textit{v.} -- alisar.

\textbf{suwa} \textit{v. pl.} = \textbf{'waza} \textit{v. du.} -- pôr na cinza, assar na cinza.

\textbf{suwaipo} \textit{s.} -- broto de buriti.

\textbf{syry} \textit{adj.} -- pequeno, breve, brando; \textbf{syry na} um pouquinho, um bocado.

\textbf{syryre} \textit{adj.} --  pequeninho.


%########################################################
\section*{T}



\textbf{ta} \textit{pref. dem.} -- esse, essa; \textbf{tanorĩ hã} esses, essas; \textbf{tahã} \textit{pron. dem.} esse, essa.

\textbf{ta . . . taha ta} \textit{} -- está aí; \textbf{e mahã ta} onde está?.

\textbf{ta} \textit{v. sing.} -- \textbf{tari} = \textbf{rĩ} \textit{v. du.} -- pegar, tirar, colher; \textbf{õ hã, te tita} ele pega, ele tira, ele colhe.

\textbf{ta} \textit{v.} = \textbf{robta} -- bater, tocar, pingar.

\textbf{ta} \textit{s.} = \textbf{robta} -- batida, choque, toque.

\textbf{ta'a} \textit{v} -- \textbf{ta} bater, tocar, pingar, cair; \textbf{õ hã, te tita'a} ele cai.

\textbf{tã'a'ane} \textit{s.} -- chuveiro.

\textbf{ta'are} \textit{s.} -- \textbf{ĩta'are} batidinha, sino, campainha.

\textbf{taza} \textit{conj.} -- contudo, todavia.

\textbf{taza hã} \textit{conj./enf.} -- contudo, todavia, mas, por isso.

\textbf{tahã} \textit{pron. dem.} -- esse, essa; \textbf{tanorĩ hã} esses, essas.

\textbf{taha} \textit{pron. dem. c.} -- \textbf{tahã} esse, essa.

\textbf{taha na} \textit{pron. dem. c./posp.} -- por aí, com isso.

\textbf{taha pari} \textit{pron. dem. c./posp.} -- depois disso.

\textbf{taha ta} \textit{pron. dem. c./. . .} -- está aí.

\textbf{taha wa} \textit{pron. dem. c./posp.} -- por isso.

\textbf{taha wi} \textit{pron. dem. c./posp.} -- daí, a partir daí.

\textbf{tãi'a} \textit{s.} -- \textbf{tã} chuva.

\textbf{tãi'are} \textit{s.} -- chuvisco.

\textbf{tãirãrã} \textit{s.} -- trovão.

\textbf{tãirãrã} \textit{vmp.} -- trovoar.

\textbf{tãiwaipo} \textit{s.} -- arco-íris.

\textbf{tãiwapsa} \textit{s.} -- raio.

\textbf{tãiwapsa'amramizé} \textit{s.} -- pára-raios.

\textbf{tãi wi dasisuwazé} \textit{s./posp.} -- capa de chuva.

\textbf{tãma} \textit{pron./posp.ontr.} -- para ele, a ele, para ela, a ela.

\textbf{tama'a} \textit{s.} -- espátula.

\textbf{tame} \textit{pron./posp.ontr.} -- aí, ali.

\textbf{tã na'rada} \textit{s.} -- começo da chuva.

\textbf{tãna'rada} \textit{s.} -- setembro.

\textbf{tane} \textit{pron./posp.ontr.} -- semelhante a esse, assim [aí] (resposta de aprovação).

\textbf{tãpĩ} \textit{s.} -- água de chuva.

\textbf{tãpĩni} \textit{s.} -- água de chuva.

\textbf{tãpĩniza'ézé} \textit{s.} -- abril.

\textbf{taré} \textit{adv.} -- à toa, ao acaso, nada.

\textbf{taré dawasu'u} \textit{adv./s.} -- calúnia.

\textbf{tarére} \textit{s.} -- bobagem, brincadeira.

\textbf{taré rowasu'u} \textit{adv./s.} -- charada.

\textbf{tari} \textit{v. sing.} = \textbf{rĩ} \textit{v. du.} -- colher, pegar, tirar, quebrar, arrancar, cortar, rasgar; \textbf{õ hã, ma tita} ele tira.

\textbf{tã, te ti'a'a} \textit{s./pron./v.} -- chove.

\textbf{tata} \textit{v.} -- bater frequentemente.

\textbf{tata'a} \textit{v.} -- \textbf{tata} bater frequentemente.

\textbf{tãsi'ubuzi} \textit{s.} -- relâmpago.

\textbf{tawa} \textit{conj.} -- então, pois, por isso.

\textbf{tawamhã} \textit{conj.} -- então, estando aí.

\textbf{te} \textit{adj. poss.} -- pertencente a, de propriedade de; \textbf{aibö te} pertencente ao homem.

\textbf{te} \textit{pron. rel. 2ªpessoa} -- tu, você, vós, vocês; \textbf{a hã, te ĩ'manha} você faz.

\textbf{te} \textit{pron. rel. 3ªpessoa} -- ele, ela; \textbf{õ hã, te mo} ele vai.

\textbf{te} \textit{pron. obl. 1ª e 3ªpessoa} -- a ele, a ela, o, a, os, as; eles, elas; \textbf{te te waihu'u da, aibö de we mo} o homem vem para sabê-lo.

\textbf{te} \textit{posp.} -- por causa de, por motivo de; \textbf{wẽ-te} porque é bom.

\textbf{te} \textit{s.} -- \textbf{ĩte} cacho.

\textbf{te} \textit{s.} -- canela, perna.

\textbf{té} \textit{interj. excl.} -- de desgosto e admiração.

\textbf{té} \textit{adj.} -- \textbf{ĩté} novo, cru, não maduro.

\textbf{tẽ} \textit{adj.} -- . . . ; \textbf{'rãtẽ} manco, coxo, paralítico; \textbf{'rãtẽ} mancar.

\textbf{tebe} \textit{s.} -- peixe.

\textbf{tebe} \textit{s.} -- \textbf{ĩtebe} coquinho maduro = \textbf{tiritebe}.

\textbf{tebe} \textit{s.} -- \textbf{ĩtebe} tia.

\textbf{tébé} \textit{s.} -- venerador da lua, adorador da lua.

\textbf{tére} \textit{adj. dim.} -- \textbf{ĩtébre} novinho, cru.

\textbf{tébré} \textit{v.} -- aumentar, criar, procriar.

\textbf{tede} \textit{adj.} -- \textbf{ĩtede} duro, consistente.

\textbf{tede} \textit{s.} -- \textbf{ĩtede} consistência.

\textbf{tédé} \textit{adj.} -- \textbf{ĩtédé} firme, seguro, gravado.

\textbf{tédé} \textit{v.} -- \textbf{ĩtédé} conter.

\textbf{tédé} \textit{s.} -- cativeiro, abraço.

\textbf{tede'wa} \textit{s.} -- \textbf{ĩtede'wa} dono, proprietário.

\textbf{tehöri} \textit{s. du. pl.} -- ceifar, colher.

\textbf{tehudu} \textit{s.} -- \textbf{ĩtehudu} cerrado ralo, campo.

\textbf{tehihi} \textit{s.} -- cacho, espiga; \textbf{asaro teihi} espiga de arroz.

\textbf{téiwa} \textit{v.} -- provar, experimentar.

\textbf{tẽme} \textit{posp.} -- \textbf{ĩtẽme} até, para perto de.

\textbf{tepãiwasi} \textit{s.} -- \textbf{ĩtepãiwasi} caderno.

\textbf{tepãiwasisire} \textit{s.} -- \textbf{ĩtepãiwasisire} caderneta.

\textbf{tepe} \textit{s.} = \textbf{tebe} -- peixe.

\textbf{tepebö} \textit{s.} -- arraia, raia.

\textbf{tepeza'ru} \textit{s.} -- barbatana.

\textbf{tepezé} \textit{s.} -- pesca, pescaria.

\textbf{tépé manadö} \textit{s.} -- comida de "tébé".

\textbf{tepe mramizé} \textit{s.} -- anzol.

\textbf{tepenhimizawi} \textit{s.} -- golfinho.

\textbf{tepepe} \textit{s.} -- peixe preto.

\textbf{tepetoro} \textit{s.} -- tubarão.

\textbf{tepe'ubumro} \textit{s.} -- cardume.

\textbf{tépé wapsu} \textit{s.} -- enfeite de pena de arara para "tébé".

\textbf{te'rãwabza} \textit{s.} -- riscos na canela, na perna.

\textbf{tete} \textit{adj. c.} = \textbf{ĩtede} -- duro, consistente.

\textbf{tété} \textit{adj. c.} = \textbf{ĩtédé} -- firme, seguro.

\textbf{tété} \textit{v} -- \textbf{ĩtédé} segurar, firmar.

\textbf{tetezé} \textit{s.} -- \textbf{ĩtetezé} força, segurança.

\textbf{ti} \textit{pref. pessoa 3ª pessoa} -- \textbf{õ hã, te ti'ö} ele pega; objeto, quantidade definidos.omo pref. pessoa em substantivos só usado quando acompanhado de objeto ou conectivo.

\textbf{ti} \textit{s.} -- flecha, taquara de bambu.

\textbf{ti} \textit{c.} = \textbf{di} -- ser, ter, haver, estar; \textbf{se ti} é gostoso.

\textbf{ti} \textit{adj.} -- escolhido.

\textbf{ti} \textit{v.} -- escolher, estabelecer, tocar; \textbf{õ hã, ma ti ni} ele ficou encarregado.

\textbf{ti'a} \textit{s.} -- chão, terra.

\textbf{ti'a} \textit{s.} (longo i) -- carapato.

\textbf{ti'a} \textit{ v.} = \textbf{a} -- dar; \textbf{õ hã, te ti'a} ele dá; \textbf{a na} dé!.

\textbf{ti'a} \textit{ v.} -- enterrar.

\textbf{ti'aza'ãzé} \textit{s.} -- carregadeira.

\textbf{ti'azada'ré} \textit{s.} -- fronteira, limite.

\textbf{ti'azawizé} \textit{s.} -- bandeira.

\textbf{ti'azubzé} \textit{s.} -- grade.

\textbf{ti'ahöbö} \textit{s.} -- carapato grande.

\textbf{ti'ai baba} \textit{s.} -- terra desocupada, baldio, no chão (correndo ou arrastando).

\textbf{ti'aizé} \textit{s.} -- petróleo.

\textbf{ti'ai'madö'ö'wa} \textit{s.} -- governador.

\textbf{ti'ai'ré} \textit{s.} -- terra seca.

\textbf{ti'ai'ré} \textit{s.} -- deserto, terra seca.

\textbf{ti'aiwabu} \textit{s.} -- cabo.

\textbf{ti'aiwa'u} \textit{s.} -- petróleo.

\textbf{ti'ai wi} \textit{s./posp.} -- afastado da terra, longe da terra, da terra.

\textbf{ti'atede'wa} \textit{s.} -- dono da terra, presidente da nação.

\textbf{ti'a'u'ẽne} \textit{s.} -- tijolo.

\textbf{ti'a'u'ẽne} \textit{s.} -- fazer tijolo.

\textbf{ti'a'wa'a} \textit{s.} -- terra cozida, tijolo.

\textbf{ti'awa'izé} \textit{s.} -- arado.

\textbf{ti'awedezé} \textit{s.} -- adubo químico, fertilizante.

\textbf{ti'awi} \textit{adv.} -- baixo, de baixo, raso, encostado na terra.

\textbf{ti'awire} \textit{adv. dim.} -- em voz baixa.

\textbf{tiha} \textit{pron. ind. fem.} -- o que.

\textbf{tiha} \textit{s. fem.} -- coisa.

\textbf{tiha} \textit{pron. interr. fem.} = \textbf{e tiha} -- como?.

\textbf{ti'i} \textit{s.} -- ariranha.

\textbf{ti'i} \textit{s.} = \textbf{ti} -- flecha, taquara de bambu.

\textbf{ti'ibu} \textit{s.} -- taquarinha.

\textbf{ti'ibu'wa} \textit{s.} -- lasca de taquarinha para riscar a pele.

\textbf{ti'ihöbö} \textit{s.} -- lontra.

\textbf{ti'ihöpöne} \textit{s.} -- castor.

\textbf{ti'i nhihö} \textit{s.} -- funda, flecha.

\textbf{ti'ipe} \textit{s.} -- flecha do bom espírito.

\textbf{tinha} \textit{v. sing.} -- ele tece = \textbf{nhamri} tecer esteira.

\textbf{tinha} \textit{v. sing.} -- ele fala = \textbf{nharĩ} falar.

\textbf{ti'õ} \textit{adj. comp.} -- \textbf{ĩti'õ} intocável, não indicado.

\textbf{tipe} \textit{interj.} -- pequeninho (exclamação).

\textbf{tiri} \textit{s.} -- semente de acuri.

\textbf{tirinho} \textit{s.} -- palmito de acuri.

\textbf{tiri'rã} \textit{s.} -- coco preto.

\textbf{tirire} \textit{s.} -- coco de acuri.

\textbf{tiritebe} \textit{s.} -- coco de acuri, coco de babaçu (quando é bom para comer).

\textbf{tiritopsu waipo} \textit{s.} -- folha de palmeira.

\textbf{tiriwede} \textit{s.} -- acuri do campo, tronco de acuri.

\textbf{tirowa} \textit{s.} -- carapato; grupo etário.

\textbf{tite} \textit{pron. poss.} = \textbf{te} -- dele, dela; pron. pessoa ele, ela.

\textbf{titopré} \textit{s.} = \textbf{ĩsirẽ} -- coroa-de-frade (fruta).

\textbf{titu} \textit{v.} -- parar de repente, parar devagar; \textbf{ma titu} ele parou.

\textbf{to} \textit{interj.} -- vamos!.

\textbf{to} \textit{s.} -- olho.

\textbf{to} \textit{s.} -- brincadeira, jogo, alegria.

\textbf{to} \textit{adj.} -- \textbf{ĩto} feliz, alegre, contente.

\textbf{to} \textit{adj.} -- \textbf{ĩsito} fechado; \textbf{ĩsitobzé} coisa fechada, fechadura.

\textbf{to} \textit{v.} -- brincar; \textbf{õ norĩ hã, te tito za'ra} eles brincam.

\textbf{tõ} \textit{adv.} -- não; \textbf{romhuri tõ} não trabalhe!.

\textbf{tô} \textit{adv.} -- também, realmente.

\textbf{tô} \textit{adv. des.} -- do passado; \textbf{wa hã, wa tô 'madö} eu vi.

\textbf{tö} \textit{s.} -- saliência; \textbf{dawatatö} bochecha.

\textbf{tob a} \textit{s.} -- \textbf{ĩtob a} olho branco.

\textbf{tob'a} \textit{adj. comp.} -- cego.

\textbf{tob'a} \textit{v.} -- cegar.

\textbf{tob'ãi'ã} \textit{s.} -- círculo, zero.

\textbf{tob'ãi'ãsĩ} \textit{s.} -- \textbf{ĩtob'ãi'ã} círculo, redondo.

\textbf{tobda'é} \textit{s.} -- peteca.

\textbf{tobzé} \textit{s.} -- \textbf{ĩsitobzé} -- coisa que fecha, tampa.

\textbf{tobzeire} \textit{adj. comp. dim.} -- \textbf{ĩtobzeire} cômico.

\textbf{tob'rã} \textit{s.} -- \textbf{ĩtob'rã} breu, piche.

\textbf{tob'rada} \textit{s.} -- rosto, face, cara, vista, aspecto, perfil.

\textbf{tob'ratanhi} \textit{s.} -- bochecha.

\textbf{tob'ratato} \textit{s.} -- um clã, círculo da face.

\textbf{tobro} \textit{v. sing.} -- = \textbf{dapusi} \textit{v. du.} sair; \textbf{watobro} \textit{v. sing.} pular.

\textbf{tobto} \textit{v.} -- colar, grudar.

\textbf{tobtobzé} \textit{s.} -- tipo de resina.

\textbf{tobtõrĩ} \textit{v.} -- pingar.

\textbf{tob'uzu} \textit{v.} -- admirar, aplaudir, admirar-se, assustar-se.

\textbf{toza'rata'wa} \textit{s.} -- organizador de festas.

\textbf{toza're} \textit{s.} -- dama.

\textbf{tozé} \textit{s.} -- brinquedo.

\textbf{töibö} \textit{interj.} -- fim!, término!.

\textbf{toire} \textit{s.} -- brincadeira.

\textbf{tomhö} \textit{s.} -- \textbf{ĩtomhö} casquinha de arroz.

\textbf{tomhö} \textit{s.} -- pálpebra.

\textbf{tomhösu} \textit{s.} -- cabelos das pálpebras, pestanas.

\textbf{tõmo} \textit{s.} = \textbf{dato} -- olho; \textbf{ĩtõmowawẽ} olho grande.

\textbf{tõmozé} \textit{s.} -- dor de olho.

\textbf{tõmo nhib'uwa} \textit{s.} -- fraqueza da vista, vista fraca.

\textbf{tõmoti} \textit{s.} -- marmelada.

\textbf{tõmosu} \textit{s.} -- pestanas.

\textbf{tonhopa} \textit{s.} -- carnaval.

\textbf{topo'o} \textit{v. du./pl.} -- acordar.

\textbf{toporo} \textit{v. sing.} -- acordar.

\textbf{topré} \textit{s.} -- conjuntivite.

\textbf{topti} \textit{s.} -- terçol.

\textbf{to pu} \textit{s./v.} -- -- cegar.

\textbf{tõrĩ} \textit{v.} -- urinar.

\textbf{tõrĩzé} \textit{s.} -- lugar de urinar, banheiro, WC.

\textbf{toro} \textit{s.} -- gula pela carne.

\textbf{toto} \textit{v.} -- latejar, pulsar.

\textbf{tõ sena} \textit{adv.} -- realmente, na verdade.

\textbf{tu} \textit{interj.} -- ó, oh!.

\textbf{tu} \textit{adj.} -- abandonado; \textbf{bödödi tu} caminho abandonado.

\textbf{tutu'rãpré} \textit{s.} -- pica-pau.

\textbf{tututu'rãpré} \textit{s.} -- pica-pau.


%#####################################################
\section*{U}



\textbf{u} \textit{s.} -- água parada, água na cabaça, lago.

\textbf{u} \textit{s.} -- bacia da pessoa, anca.

\textbf{u} \textit{s.} -- \textbf{ĩ'u} chifre.

\textbf{u} \textit{posp.} -- para, a; \textbf{romhuri u} ao trabalho.

\textbf{u} \textit{interj.} -- falou, disse.

\textbf{u'a} \textit{v.} -- \textbf{ĩ'u'a} bater ferindo, machucar.

\textbf{u'ã} \textit{s.} -- jabuti.

\textbf{u'azé} \textit{s.} -- batida; \textbf{da'u'azé dawi ĩdupto} contusão.

\textbf{u'ãihöpö} \textit{s.} -- tartaruga.

\textbf{u'awi} \textit{s.} -- cuia.

\textbf{uba} \textit{s.} -- ponte.

\textbf{uba're} \textit{s.} -- barco, canoa.

\textbf{uba'rehöbö} \textit{s.} -- balsa.

\textbf{uba're napraba'wa} \textit{s.} -- barqueiro.

\textbf{uba're sari} \textit{s./v.} -- remar.

\textbf{uba're sarizé} \textit{s.} -- remo.

\textbf{uba'rewã} \textit{s.} -- casco.

\textbf{uba uparizé} \textit{s.} -- coluna de ponte, esteio de ponte.

\textbf{ubdi} \textit{s.} -- inhame.

\textbf{ubdö} \textit{s.} -- capivara.

\textbf{ub'ra} \textit{s.} -- borduna redonda, borduna cerimonial.

\textbf{ub'rã} \textit{s.} -- cacho.

\textbf{ub'rã} \textit{v.} -- \textbf{ĩ'ub'rã} cerrar.

\textbf{ub'rãna} \textit{s.} -- taquara, taboca.

\textbf{ub'rãtãhiwawẽ} \textit{s.} -- taquara de bambu.

\textbf{ub'rãtãwawẽ} \textit{s. aum.} -- taboca grande; Aldeia Namuncurá.

\textbf{ub'ré} \textit{s.} -- ramo, galho seco.

\textbf{ubu} \textit{s.} -- rosto, cara.

\textbf{ubu} \textit{s.} -- mosca.

\textbf{ubu} \textit{v.} -- \textbf{ĩ'ubu} cobrir, embrulhar.

\textbf{ubuzé} \textit{s.} -- \textbf{ĩ'ubuzé} cobertor.

\textbf{ubuzi} \textit{adj.} -- lustroso, brilhante.

\textbf{ubuzi} \textit{s.} -- choque elétrico.

\textbf{ubuzö} \textit{s.} -- espinha no rosto.

\textbf{ubumro} \textit{v. pl.} -- = \textbf{dasimasisi} \textit{v. du.} ficar, reunir, reunir-se, sentar-se, ajuntar.

\textbf{ubumro} \textit{s.} -- coleção, conjunto.

\textbf{ubumrozé} \textit{s.} -- ônibus.

\textbf{ubumroipe} \textit{vmp.} -- \textbf{ĩ'ubumroipe} concentrar, colecionar.

\textbf{ubumroi'wa} \textit{s.} -- \textbf{ĩ'ubumroi'wa} quem reúne, colecionador.

\textbf{ubuni} \textit{adj.} -- virgem.

\textbf{ubupré} \textit{adj. comp.} -- de rosto vermelho, bofetada.

\textbf{uburã} \textit{s.} -- rosto branco.

\textbf{uburé} \textit{adj.} -- tudo.

\textbf{uburõ} \textit{s.} -- tumor, abscesso.

\textbf{uburõi pré} \textit{s.} -- penas de arara vermelha.

\textbf{ubu'upsõmri} \textit{vmp.} -- beijar (lamber o rosto).

\textbf{udö} \textit{v.} -- mentir, negar, fingir.

\textbf{udu} \textit{v.} -- levantar-se.

\textbf{uza} \textit{s.} -- roupa, camisa, vestido, túnica.

\textbf{uza} \textit{v.} -- vestir-se.

\textbf{uza ãna} \textit{s./posp.} -- sem roupa, nu.

\textbf{uzadö} \textit{s.} -- calça.

\textbf{uzadö pa} \textit{s.} -- calça comprida.

\textbf{uzadö 'rudu} \textit{s.} -- calça curta, calção.

\textbf{uzazé} \textit{s.} -- roupa, roupagem, vestimenta.

\textbf{uza hönhi'u} \textit{s.} -- camiseta sem manga.

\textbf{uzapa} \textit{s.} -- camisola.

\textbf{uzapo} \textit{s.} -- morango.

\textbf{uzapodo} \textit{s.} -- cara grande.

\textbf{uzapu} \textit{s.} -- mau jeito de costas.

\textbf{uze} \textit{v.} -- louvar.

\textbf{uze} \textit{adj.} -- \textbf{ĩ'uzé} verde, azul.

\textbf{uzi} \textit{s.} -- lanterna.

\textbf{uziwaihõ} \textit{s.} -- pilha.

\textbf{uzö} \textit{s.} -- fogo.

\textbf{uzöne} \textit{s.} -- abóbora.

\textbf{uzönewede} \textit{s.} -- mamão.

\textbf{uzu} \textit{s.} -- coco de buriti.

\textbf{uzu} \textit{s.} -- \textbf{ãma ĩ'uzu} continuação.

\textbf{uzu} \textit{v.} -- cumprimentar, saudar.

\textbf{uzuri} \textit{v.} -- riscar.

\textbf{uzusi} \textit{v} -- cumprimentar, aproximar-se, ir ao encontro.

\textbf{uzusizé} \textit{s.} -- lugar de recepção, sala de recepção.

\textbf{uzusu} \textit{} -- . . . .

\textbf{u'ẽne} \textit{s.} -- \textbf{ĩ'u'ẽne} pão, bolo feito xavante; \textbf{ti'a'u'ẽne} tijolo.

\textbf{u'ẽne} \textit{v.} -- \textbf{ĩ'u'ẽne} fazer massa, fazer pão.

\textbf{u'ẽtẽ} \textit{s.} -- \textbf{ĩ'u'ẽne} massa dura; pão, bolo feito.

\textbf{u'ẽtẽ} \textit{v} -- \textbf{ĩ'u'ẽne} fazer massa, fazer pão.

\textbf{u'ẽtẽze} \textit{s.} -- \textbf{ĩ'u'ẽtẽze} bolo.

\textbf{u'ẽtẽzeire} \textit{s.} -- \textbf{ĩ'u'ẽtẽzeire} bolacha, biscoito.

\textbf{u'ẽtẽwapu} \textit{s.} -- \textbf{ĩ'u'ẽtẽwapu} pão.

\textbf{u'éwaru} \textit{s.} -- zebra.

\textbf{uhi} \textit{s.} -- feijão.

\textbf{uhi} \textit{} -- . . . ; \textbf{danebzé si'uhi} corredor.

\textbf{uhizanozé} \textit{s.} -- junho.

\textbf{uhö} \textit{s.} -- queixada.

\textbf{uhö} \textit{s.} -- pintura preta.

\textbf{uhöba} \textit{s.} -- \textbf{ĩ'uhöba} pintura das costas.

\textbf{uhö ba} \textit{s.} -- costas de queixada.

\textbf{uhöbö} \textit{s.} -- porco.

\textbf{uhödö} \textit{s.} -- anta.

\textbf{uhönhisi'reti} \textit{s.} -- hipopótamo.

\textbf{uhöre} \textit{s.} -- caititu.

\textbf{uhu} \textit{adj.} -- \textbf{ĩ'uhu} nebuloso, enfumaçado.

\textbf{uibro} \textit{s.} -- borduna, cacete.

\textbf{uibro'wa} \textit{s.} -- jararacuçu.

\textbf{ui'éré} \textit{v.} -- escrever.

\textbf{ui'éré} \textit{s.} -- escrito, carta.

\textbf{uihö} \textit{s.} -- fronte, testa.

\textbf{uihö'idi'idi} \textit{s.} -- vertigem.

\textbf{uihöiwadu} \textit{s.} -- sobrancelha.

\textbf{uimre} \textit{adv.} -- antes, primeiramente; 
\textbf{wa uimre} primeiro, pioneiro.

\textbf{uipĩ} \textit{s.} -- lágrima.

\textbf{uipini'a'a} \textit{adj. comp.} -- \textbf{ĩ'uipini'a'a} lacrimejando.

\textbf{uipra} \textit{v.} -- comprar.

\textbf{uirĩ} \textit{v.} -- \textbf{ĩ'uirĩ} rodear, contornar, dar a volta, envolver, circundar, consolar.

\textbf{uirĩ} \textit{s.} -- \textbf{ĩ'uirĩ} contorno, volta.

\textbf{uirĩzé} \textit{s.} -- \textbf{ĩ'uirĩzé} circunferência.

\textbf{u'iti'wa} \textit{s.} -- carrasco.

\textbf{ui'uzé} \textit{} -- bebida alcoólica, cachaça.

\textbf{ui'uzénhipãri} \textit{s.} -- bêbado.

\textbf{uiwada} \textit{adj.} -- par, ao lado de, paralelo.

\textbf{uiwada} \textit{s.} -- sobrancelha.

\textbf{uiwẽ} \textit{v.} -- brilhar, resplandecer.

\textbf{uiwede} \textit{s.} -- buriti, corrida de tora de buriti.

\textbf{uiwẽzé} \textit{s.} -- \textbf{ĩ'uiwẽzé} brilho.

\textbf{umhi} \textit{s.} -- coluna vertebral, espinha.

\textbf{umhöbö} \textit{s.} -- \textbf{powawẽ umhöbö} búfalo.

\textbf{umimi} \textit{s.} -- \textbf{ĩ'umimi} mofo.

\textbf{umimire} \textit{s.} -- formiga de asa, rainha de formiga que voa.

\textbf{umnha} \textit{s.} -- confiança, crédito.

\textbf{umnhasi} \textit{v} -- confiar.

\textbf{umnhasi} \textit{s.} -- crédito.

\textbf{umnhasizé} \textit{s.} -- confiança.

\textbf{umnhasi-te} \textit{s. posp.} -- confiando.

\textbf{umnhasi waihu'upe} \textit{s./vmp.} -- \textbf{ĩ'umnhasi waihu'upe} confidencial.

\textbf{umnhi'ã} \textit{s.} -- arco, arma.

\textbf{umnhi'ãsi} \textit{s.} -- \textbf{umnhi'ã} arco, arma.

\textbf{umnhi'ãsiza'ré} \textit{s.} -- coronha.

\textbf{umnhi'ãsiro} \textit{s.} -- espingarda, carabina.

\textbf{umnhi'ãsiro zapore} \textit{s.} -- revólver.

\textbf{umnhi'ãsiro'owawẽ} \textit{s. aum.} -- canhão.

\textbf{umnhi'ã'umro} \textit{s.} -- arco alisado.

\textbf{umo} \textit{v.} -- aplainar borduna, alisar borduna, alisar pauzinho.

\textbf{um'rãtede} \textit{s.} -- sapo grande.

\textbf{umrezeire} \textit{s.} -- melancia.

\textbf{umrenhiduruture} \textit{s.} -- apito de cabacinha.

\textbf{umro} \textit{adj.} -- \textbf{ĩ'umro} pouco, pouca gente.

\textbf{umrore} \textit{adj. dim.} -- \textbf{ĩ'umrore} pouco, pouquinho.

\textbf{umro wẽi'õ} \textit{adj./adj.} -- admiração de muita coisa.

\textbf{umro wẽ'õ} \textit{adj./adj.} -- abundante.

\textbf{una} \textit{s.} -- \textbf{ĩ'una} carreta, trole.

\textbf{uneire} \textit{s.} -- arbusto.

\textbf{unhama} \textit{s.} -- uzö chama.

\textbf{unhamare} \textit{s.} -- fósforo.

\textbf{unhamaro} \textit{s.} -- chama.

\textbf{unhamaro} \textit{vmp.} -- chamejar.

\textbf{unhinha} \textit{s.} -- feijão com farinha, tutu.

\textbf{unhoti} \textit{s.} -- espinha.

\textbf{u'ö} \textit{adv.} -- sempre, continuamente, constantemente; \textbf{ãne u'ö} constância, continuação.

\textbf{u'õ} \textit{adj. comp.} -- seco.

\textbf{u'öbö} \textit{posp.} -- por causa de; \textbf{õhõ u'öbö} por causa daquele.

\textbf{u'ösi} \textit{adv} -- \textbf{u'ö} sempre, constantemente, continuamente; \textbf{õne u'ösi} sempre, constantemente.

\textbf{upa} \textit{s.} -- mandioca.

\textbf{upa} \textit{v.} -- errar.

\textbf{upa} \textit{s.} -- erro.

\textbf{upada} \textit{s.} -- coxa.

\textbf{upa za'ẽne} \textit{s.} -- \textbf{ĩ'upa za'ẽne} erro grande, crime.

\textbf{upazé} \textit{s.} -- mandioca braba.

\textbf{upazu} \textit{s.} -- farinha de mandioca.

\textbf{upai'õ} \textit{adj. comp.} -- \textbf{ĩ'upai'õ} perfeito, sem defeito.

\textbf{upana} \textit{posp.} -- \textbf{ĩ'upana} substituto, alheio.

\textbf{upari} \textit{v.} -- ajudar, defender, apoiar, refazer, sustentar, promover; \textbf{sadawa da'upari} apoiar.

\textbf{uparizé} \textit{s.} -- \textbf{ĩ'uparizé} sustentáculo, suporte, base.

\textbf{upari'wa} \textit{s.} -- ajudante.

\textbf{upata} \textit{s.} -- coxa.

\textbf{upatazépu} \textit{s.} -- ciática.

\textbf{upawã} \textit{s.} -- berrante.

\textbf{upi} \textit{s.} -- choro, lágrima.

\textbf{upi} \textit{s.} -- peixe eléctrico.

\textbf{upi} \textit{v.} -- tocar.

\textbf{upini} \textit{s.} -- lágrima, ranger de dente.

\textbf{upire} \textit{s.} -- brincadeira na água.

\textbf{upõrĩ} \textit{v.} -- enxugar, soprar, secar.

\textbf{upro} \textit{v.} -- acabar com, terminar com, consumir, concluir, exterminar.

\textbf{uprosi} \textit{v} -- \textbf{ĩ'upro} acabar com, terminar com, consumir, concluir, exterminar.

\textbf{uprosi'wa} \textit{s.} -- consumidor.

\textbf{upta} \textit{adv.} -- verdadeiro, muito.

\textbf{uptabi} \textit{adv.} -- verdadeiro, muito; \textbf{ĩ'wẽ uptabi} bom mesmo.

\textbf{uptabi'õ} \textit{adj. comp.} -- \textbf{ĩ'uptabi'õ} casual.

\textbf{uptabi'õre} \textit{s.} -- \textbf{ĩ'uptabi'õre} bobagem.

\textbf{upte} \textit{v.} -- lavar, tomar banho.

\textbf{upté} \textit{s.} -- pintura vermelha.

\textbf{upti} \textit{} -- .. \textbf{dapara'upti} bicho do pé.

\textbf{upto} \textit{v.} -- sujar, borrar.

\textbf{upto} \textit{s.} -- \textbf{ĩ'upto} mancha, borrão.

\textbf{upsãna} \textit{s.} -- divisão, separação, partilha.

\textbf{upsãna} \textit{v.} -- dividir, separar, pôr em ordem, partilhar.

\textbf{upsãtã} \textit{v} -- \textbf{upsãna} dividir, separar, pôr em ordem.

\textbf{upsé} \textit{s.} -- cócega, coceira, sarna.

\textbf{upsi} \textit{v. sing.} -- cobrir, encobertar.

\textbf{upsibi} \textit{v.} -- lamber, chupar, beijar.

\textbf{upure} \textit{s.} -- mosquito.

\textbf{u'rã} \textit{v.} -- \textbf{ĩ'u'rã} chamuscar.

\textbf{u'rã} \textit{s.} -- \textbf{ĩ'u'rã} crepúsculo.

\textbf{u'rada} \textit{s.} -- têmpora.

\textbf{u'rãre} \textit{adj. comp. dim.} -- cedo.

\textbf{u'rata} \textit{v} -- afastar-se, virar as costas.

\textbf{u'ratazéré} \textit{s.} -- costeleta.

\textbf{u're} \textit{s.} -- água nascente, fonte.

\textbf{u'ré} \textit{v.} -- ungir.

\textbf{u'rébéro} \textit{s.} -- morro em chamas; Aldeia Meruri.

\textbf{u'rézé} \textit{s.} -- unguento, óleo de ungir.

\textbf{u'réi'wa} \textit{s.} -- aquele que unge.

\textbf{uridi} \textit{s.} -- casa de tijolo, rancho, pórtico.

\textbf{urĩrĩ} \textit{v.} -- mostrar, descobrir.

\textbf{uriti} \textit{s.} -- \textbf{uridi} casa de tijolo, pórtico.

\textbf{uriti} \textit{s.} -- \textbf{uridi} casa de tijolo, pórtico.

\textbf{uro} \textit{s.} -- cabelo das têmporas.osteleta.

\textbf{utãtã} \textit{s.} = \textbf{rob'utãtã} -- trovoada.

\textbf{uti} \textit{s.} -- sapo.

\textbf{uti'wa} \textit{s.} -- que lava a criança.

\textbf{utö} \textit{s.} -- \textbf{uhödö} anta.

\textbf{utönhisi're'u} \textit{s.} -- rinoceronte.

\textbf{utõrĩ} \textit{v.} -- envolver.

\textbf{utö'uwatine} \textit{s.} -- pá.

\textbf{usi} \textit{s.} -- cinto, furúnculo.

\textbf{usimri'wa} \textit{s.} -- parente do outro clã na caça.

\textbf{usisi} \textit{s.} -- cinto.

\textbf{usisitetezé} \textit{s.} -- cinto.

\textbf{usu} \textit{s.} -- chuva interminável.

\textbf{usu} \textit{s.} -- do mesmo clã.

\textbf{utu'u} \textit{s.} -- pomba.

\textbf{u'u} \textit{s.} = \textbf{u} -- lago.

\textbf{u'unawapsã} \textit{s.} -- foca.

\textbf{u'unho'a} \textit{s.} -- baía.

\textbf{u'utede'wa} \textit{s.} -- dono do lago.

\textbf{uwa} \textit{adj.} -- \textbf{ĩ'uwa} fraco, leve, bambo.

\textbf{uwa} \textit{s. neo.} -- uva.

\textbf{uwa} \textit{v.} -- \textbf{l'uwa} enfraquecer.

\textbf{uwa} \textit{s.} -- fronte.

\textbf{uwaibaba} \textit{posp.} -- \textbf{ĩ'uwaibaba} rumo a, por cima de, sempre que, de acordo com, em base de, no meio de.

\textbf{uwaibaba} \textit{pron. ind.} -- \textbf{ĩ'uwaibaba} qualquer coisa que, o que quer que for.

\textbf{uwaibaba} \textit{s.} -- ordem recebida, prescrição.

\textbf{uwaimrami} \textit{v.} -- conseguir, acertar, alcançar, realizar, cumprir, ordenar, regular.

\textbf{uwai're} \textit{s.} -- fruta do mato.

\textbf{uwape} \textit{adj. comp.} -- muito fraco, fraquíssimo.

\textbf{uwapese} \textit{adj./adv.} -- fraquíssimo.

\textbf{uware} \textit{adj. dim.} -- \textbf{ĩ'uware} fraco, mole.

\textbf{uwari} \textit{v.} -- tirar feijão do bago, bater feijão, descascar.

\textbf{uwa'ru} \textit{s. neo.} -- vinha.

\textbf{uwa'ru 'madö'ö'wa} \textit{s. neo./s.} -- vinhateiro.

\textbf{uwati} \textit{v.} -- censurar, apaziguar, negar.

\textbf{uwe} \textit{v.} -- alcançar, conseguir, chegar, alcançar.

\textbf{uwire} \textit{s.} -- canivete.

%#################################################

\section*{W}.




\textbf{'wa da} \textit{v./posp.} -- afiar, para afiar.

\textbf{'wa'rada} \textit{s.} -- mandíbula.

\textbf{'wa'rada} \textit{s.} -- pente.

\textbf{'wa'ratahi} \textit{s.} -- maxilar.

\textbf{'wa'rutu} \textit{adj.} -- \textbf{ĩwa'rutu} sujo.

\textbf{'wa'rutu} \textit{v.} -- sujar, estragar.

\textbf{'wa'wa} \textit{s.} -- piranha.

\textbf{'wa'ö} \textit{conj.} -- se (condicional).

\textbf{'wahire} \textit{adj. dim.} -- \textbf{ĩ'wahire} magro, magrinho.

\textbf{'wahi} \textit{adj.} -- \textbf{ĩ'wahi} magro.

\textbf{'wamarĩ 'manharĩ} \textit{s.} -- bênção.

\textbf{'wamarĩ höi'ré} \textit{s.} -- árvore.

\textbf{'wamarĩzubtede'wa} \textit{s.} -- sonhador, pacificador.

\textbf{'wamarĩzu} \textit{s.} -- pó de feitiço do pacificador.

\textbf{'wamarĩ} \textit{s.} -- tronco de árvore, cruz.

\textbf{'wamhã} \textit{conj.} -- quando; \textbf{niwamhã} certa vez, quando.

\textbf{'wamnarĩsipu} \textit{adj. comp.} -- colorido.

\textbf{'wamnarĩ} \textit{s.} -- enfeite, cor.

\textbf{'wamnarĩ} \textit{v.} -- enfeitar, corar.

\textbf{'wanherẽ} \textit{v. sing.} = \textbf{'waza} \textit{v. du.} -- meter na brasa, assar.

\textbf{'wanhihöiwa} \textit{s.} -- gavião.

\textbf{'wanhimo} \textit{s.} -- \textbf{ĩ'wanhimo} flecha comum.

\textbf{'wanhipré} \textit{s.} -- broto.

\textbf{'wanhipré} \textit{v.} -- brotar.

\textbf{'wanhirã} \textit{s.} -- gengiva.

\textbf{'wapa} \textit{} -- ..; \textbf{sib'ézé 'wapa} espada, facão.

\textbf{'wapéi'wa} \textit{s.} -- competidor.

\textbf{'wapézé} \textit{s.} -- \textbf{ĩwapézé} maca, lugar de competição, veículo.

\textbf{'wapé} \textit{v. du.} -- levar, carregar, lutar.

\textbf{'wari} \textit{v.} -- ..; \textbf{dasai 'warizé} \textit{s.} ânus.

\textbf{'wasari} \textit{v. pl.} = \textbf{'wapé} \textit{v. du.} -- carregar, distribuir, partir, levar.

\textbf{'wasa} \textit{s.} -- constelação.

\textbf{'wasa} \textit{v. sing.} -- \textbf{'wasari} = \textbf{'wapé} \textit{v. du.} carregar.

\textbf{'wasito} \textit{s.} -- \textbf{ĩ'wasito} alicate.

\textbf{'waté'wa} \textit{s.} -- batedor de água.

\textbf{'watébrémi} \textit{s.} -- menino.

\textbf{'wawi} \textit{posp.} -- afastado de, de, em defesa de, dentro de, em, em frente de; \textbf{'ri 'wawi} em frente da casa.

\textbf{'waza'a} \textit{s.} -- pneumonia.

\textbf{'waza} \textit{v. du.} -- cozinhar em brasas, assar em brasas.

\textbf{'wazé} \textit{s.} -- \textbf{ĩ'wazé} lima, grosa.

\textbf{'wazé} \textit{s.} -- dor de dente.

\textbf{'wa} \textit{des. du. 2ªpessoa} -- \textbf{a norĩ wa'wa hã, te ĩromhuri 'wa} vocês dois trabalham.

\textbf{'wa} \textit{pron. interr.} -- quem? \textbf{e 'wa hã} quem?.

\textbf{'wa} \textit{s.} -- dente.

\textbf{'wa} \textit{v. sing.} -- afiar; \textbf{wa hã, wa za 'wa} eu vou afiar.

\textbf{-'wa} \textit{suf.} -- agente da ação verbal, realizador da ação verbal.

\textbf{-wawẽ} \textit{suf.} -- aumentativo.

\textbf{wa'awede} \textit{s.} -- tucum.

\textbf{wa'azé} \textit{s.} -- preguiça.

\textbf{wa'a} \textit{adj.} -- \textbf{ĩwa'a} preguiçoso.

\textbf{wa'a} \textit{s.} -- preguiça.

\textbf{wa'a} \textit{v} -- {ser preguiçoso}

\textbf{wa'ẽ} \textit{v.} -- zombar, chingar; \textbf{ãma dawa'ẽ} cínico.

\textbf{wa'ire} \textit{s.} -- \textbf{ĩwai're} óvulo, subida no morro.

\textbf{wa'i} \textit{s.} -- terra arada para a luta, luta dos "wapté".

\textbf{wa'ĩni} \textit{s.} -- pilão.

\textbf{wa'rada} \textit{s.} -- nosso avô.

\textbf{wa'ra} \textit{s.} -- cerca de madeira, cerca, cercado, redil, páteo.

\textbf{wa're zubru} \textit{s.} -- abscesso.

\textbf{wa're'a} \textit{s.} -- cicatriz.

\textbf{wa'redö} \textit{s.} -- cicatriz.

\textbf{wa're} \textit{s. aa.} -- orelha.

\textbf{wa're} \textit{s.} -- ferida, chaga.

\textbf{wa'ridi} \textit{s.} -- vagina.

\textbf{wa'ritire} \textit{s.} -- seriema (\textit{Cariama cristata}).

\textbf{wa'ritire} \textit{s.} -- seriema.

\textbf{wa'ro} \textit{adj.} -- \textbf{ĩwaro} quente.

\textbf{wa'ro} \textit{s.} -- febre, calor.

\textbf{wa'ru wa'u} \textit{s.} -- sabugo de milho.

\textbf{wa'ru'a} \textit{s.} -- milho branco.

\textbf{wa'rudu} \textit{adj.} -- \textbf{ĩwa'rudu} bagunçado.

\textbf{wa'rudu} \textit{v.} -- confundir, misturar, fazer massa (bolo, tijolo).

\textbf{wa'rutu} \textit{s.} -- \textbf{wa'rudu} confusão, mistura.

\textbf{wa'rutu} \textit{v} -- \textbf{wa'rudu} confundir, misturar, fazer massa (bolo, tijolo).

\textbf{wa'rutu} \textit{v} -- \textbf{wa'rutu} encurtar.

\textbf{wa'ru} \textit{adj.} -- \textbf{ĩwa'ru} alto.

\textbf{wa'ru} \textit{s.} -- milho.

\textbf{wa'ru} \textit{} = \textbf{da'rãi wa'ru} -- confundir.

\textbf{wa'rãmi} \textit{s. du. pl.} -- responder, retrucar, contrariar, dar em troca, vingar-se, contestar.

\textbf{wa'rãmi} \textit{s.} -- \textbf{ãma ĩwa'rãmi} contrariedade.

\textbf{wa'rã} \textit{v. sing.} -- \textbf{wa'rãmi} = \textbf{dawa'rãmi} \textit{v. du./pl.} responder, retrucar.

\textbf{wa'rémapu} \textit{s.} -- \textbf{ĩwa'rémapu} esponja.

\textbf{wa'ré} \textit{s.} -- injeção.

\textbf{wa'ré} \textit{v. sing. du.} -- machucar, ferir com flecha, pedra, dar injeção.

\textbf{wa'u'u} \textit{s.} -- \textbf{wa'u} soprar.

\textbf{wa'u'u} \textit{v} -- \textbf{wa'u} sopro, vento.

\textbf{wa'uburé} \textit{s.} -- diabo, demônio.

\textbf{wa'udu} \textit{v.} -- descansar.

\textbf{wa'uihörö} \textit{vmp.} -- chupar, sorver.

\textbf{wa'uire} \textit{s.} -- \textbf{ĩwa'uire} caneta.

\textbf{wa'uiwamre} \textit{s.} -- óleo, unguento.

\textbf{wa'umhi} \textit{s.} -- coluna vertebral.

\textbf{wa'utuzé} \textit{s.} -- descanso, lugar de descanso, sábado.

\textbf{wa'utu} \textit{v} -- \textbf{wa'udu} descansar.

\textbf{wa'utu} \textit{v} -- {sair correndo}

\textbf{wa'uzadaze} \textit{s.} -- \textbf{ĩwa'uzadaze} perfume.

\textbf{wa'uze} \textit{s.} -- \textbf{ĩwa'uze} caldo.

\textbf{wa'uzé} \textit{s.} -- \textbf{ĩwa'uzé} caneta.

\textbf{wa'uzé} \textit{s.} = \textbf{rob'rãiwa'uzé} vinagre.

\textbf{wa'u} \textit{s.} -- líquido, caldo, palmito, suco, extrato de plantas, miolo, cerne.

\textbf{wa'u} \textit{s.} -- sêmen, esperma.

\textbf{wa'u} \textit{s.} = \textbf{rowa'u} -- sopro, vento.

\textbf{wa'u} \textit{v.} -- exprimir, tirar caldo.

\textbf{wa'wa} \textit{des. 2ªpessoa du. pl.} -- \textbf{a norĩ wa'wa hã, te ĩromhuri za'ra wa'wa} vocês todos trabalham.

\textbf{wa'wa} \textit{posp.} -- no meio de, central.

\textbf{wa'õno} \textit{s.} -- \textbf{ĩwa'õno} parte; pedaço de bolo de milho, metade.

\textbf{wa'õno} \textit{v.} -- \textbf{ĩwa'õno} cortar em pedaços, partir, dividir.

\textbf{wa'õtõtede} \textit{s.} -- cápsula.

\textbf{wa'õ} \textit{s.} -- quati.

\textbf{wa'öbö na ĩsõmri} \textit{s./posp./v.} -- vender.

\textbf{wa'öbö syry} \textit{s.} -- \textbf{ĩwa'öbö syry} preço baixo, barato.

\textbf{wa'öbö za'ẽne} \textit{s.} -- \textbf{ĩwa'öbö za'ẽne} caro, preço alto.

\textbf{wa'öbözé} \textit{s.} -- \textbf{ĩwa'öbözé} preço, pagamento.

\textbf{wa'öbö} \textit{v} -- \textbf{ĩwa'ö} cobrar, pagar.

\textbf{wa'ö} \textit{s.} -- \textbf{ĩwa'ö} preço.

\textbf{wa'ö} \textit{v.} -- \textbf{ĩwa'ö} cobrar, pagar; \textbf{dama ĩwa'ö} vender.

\textbf{wa-} \textit{pref. pessoa 1ªpessoa du. e pl.} -- \textbf{wadub'rada} nosso irmão mais velho.

\textbf{wab're} \textit{adj.} -- \textbf{ĩwab're} dentro, acertado, duro, sólido.

\textbf{wab'ru'rã} \textit{s.} -- batata.

\textbf{wab'ẽne} \textit{s.} -- gemido.

\textbf{wab'ẽne} \textit{v.} -- gemer.

\textbf{wab'ẽtẽ} \textit{v} -- gemer.

\textbf{wabu} \textit{s.} -- talo de folha de buriti.

\textbf{wabza} \textit{s.} -- risco no corpo.

\textbf{wabzu'wa} \textit{s.} -- quase do mesmo tamanho.

\textbf{wabzuri} \textit{v. du.} -- \textbf{ĩwabzuri} jogar, jogar fora, expulsar, afastar, chutar, lançar, criar.

\textbf{wabzu} \textit{s. sing.} -- \textbf{wabzuri} \textit{v. du.} achar, encontrar.

\textbf{wabzéréwede} \textit{s.} -- cacto.

\textbf{wabzéré} \textit{s.} -- espinho, cardo.

\textbf{wada'uri'wa} \textit{s.} -- chefe, animador, comandante, conselheiro.

\textbf{wada'urizé} \textit{s.} -- exortação, animação.

\textbf{wada'uri} \textit{s.} -- consolação, ânimo, exortação.

\textbf{wada'uri} \textit{v.} -- exortar, animar, confortar, consolar, mandar.

\textbf{wada} \textit{s.} -- \textbf{ĩwada} bico.

\textbf{wada} \textit{s.} -- queixo.

\textbf{wada} \textit{v.} = \textbf{ãma dawada} -- bater palmas, louvar, elogiar.

\textbf{wadi} \textit{adj.} -- \textbf{ĩwadi} vários.

\textbf{wahi nhisari} \textit{s.} -- mordida de cobra, picada de cobra.

\textbf{wahi'uptabi} \textit{s.} -- jararaca.

\textbf{wahi'wa wa'u uprosizé} \textit{s.} -- soro antiofídico.

\textbf{wahi'wa'u} \textit{s.} -- veneno de cobra (líquido de dente de cobra).

\textbf{wahi'wa} \textit{s.} -- grupo etário mais velho.

\textbf{wahihöira} \textit{s.} -- jibóia.

\textbf{wahiprére} \textit{s.} -- coral (cobra).

\textbf{wahiwawẽ} \textit{s. aum.} -- surucucu.

\textbf{wahi} \textit{s.} -- cobra, serpente.

\textbf{wahu pibuzé} \textit{s.} -- calendário.

\textbf{wahu'õ} \textit{adj. comp.} -- \textbf{ĩwahu'õ} consistente, não solta, não afrouxa.

\textbf{wahu'õ} \textit{s.} -- \textbf{ĩwahu'õ} consistência.

\textbf{wahub amo na} \textit{s./posp.} -- ano próximo, ano que vem.

\textbf{wahub tede'wa} \textit{s.} -- dono da seca.

\textbf{wahub'rãsudu} \textit{s.} -- fim do ano, dezembro.

\textbf{wahubna'rada} \textit{s.} -- começo de ano, janeiro.

\textbf{wahudu} \textit{v.} -- levantar-se.

\textbf{wahuré} \textit{pron. ind.} -- tudo, todo.

\textbf{wahu} \textit{s.} -- ano, seca.

\textbf{wahö mapu} \textit{s.} -- \textbf{ĩwahö mapu} neve.

\textbf{wahöbö} \textit{s.} -- descida; \textbf{ẽtẽ wahöbö} descida do monte.

\textbf{wahöpö} \textit{s.} -- \textbf{wahöbö} descida.

\textbf{wahözé} \textit{s.} -- \textbf{ĩwahözé} gelo, gelado, frio; \textbf{öwahö'u'ẽne} \textit{s.} gelo.

\textbf{wahözé} \textit{v.} -- \textbf{ĩwahözé} gelar, congelar, resfriar.

\textbf{wahö} \textit{adj.} -- frio.

\textbf{wahö} \textit{v.} -- perder calor, perder temperatura, abandonar, deixar, sair, resfriar, ficar arrepiado; \textbf{õ hã, ma tô tiwahö} ele perdeu a febre.

\textbf{wahö} \textit{v} -- {ser morno}

\textbf{wai'a sipi'õ} \textit{s.} -- mulher de "wai'a".

\textbf{wai'a'rada} \textit{s.} -- velho "wai'a" (homem).

\textbf{wai'a} \textit{s.} -- festa religiosa importante, culto xavante.

\textbf{wai'i} \textit{s.} -- gavião noturno, coruja.

\textbf{wai'i} \textit{s.} -- soluço.

\textbf{wai'o} \textit{s.} -- \textbf{ĩwai'o} caldo, lama, mass.om água, sopa.

\textbf{wai're ubumro} \textit{s.} -- \textbf{ĩwai're ubumro} ovário.

\textbf{wai'rẽne} \textit{s.} -- parecido a morrinho situado no alto; Aldeia São José (R.I. Couto Magalhães).

\textbf{wai'ré} \textit{s.} -- colo (=perna).

\textbf{wai'u} \textit{s.} -- berne.

\textbf{waibaba} \textit{posp.} -- no meio de.

\textbf{waihi nhisu} \textit{s.} -- flecha para criança.

\textbf{waihizati} \textit{s.} -- costela.

\textbf{waihi} \textit{adj.} -- \textbf{ĩwaihi} estreito, magro.

\textbf{waihi} \textit{s.} -- \textbf{ĩwaihi} costela, galho, membro, ramo, nervo da folha do buriti.

\textbf{waihi} \textit{v.} -- molhar, derramar, jogar água.

\textbf{waihu'upe} \textit{adj. comp.} = \textbf{dama ĩwaihu'upe} -- célebre.

\textbf{waihu'uzé} \textit{s.} -- sabedoria, conhecimento.

\textbf{waihu'u} \textit{v.} -- saber, conhecer; \textbf{wa za tãma waihu'u} eu vou comunicar-lhe.

\textbf{waihu} \textit{adj.} -- \textbf{ĩwaihu} ciente.

\textbf{waihã} \textit{} -- ..; \textbf{dasaiwaihãzé} \textit{s.} cebola.

\textbf{waihõsi'rãmi} \textit{s.} -- cólica.

\textbf{waihõ} \textit{s.} -- interior do homem, intestino; \textbf{uziwaihõ} pilha de lanterna.

\textbf{waihö'özé} \textit{s.} -- vingança, ciúme.

\textbf{waihö'ö} \textit{v} -- \textbf{waihö} enciumar-se, pirraçar, zangar, vingar-se.

\textbf{waihörö} \textit{s.} -- papagaio.

\textbf{waihö} \textit{v.} -- enciumar-se, pirraçar, zangar, vingar-se.

\textbf{waimri} \textit{adj.} -- com calma.

\textbf{waipare} \textit{s.} -- rego.

\textbf{waipozé} \textit{s.} -- brilho.

\textbf{waipo} \textit{adj.} -- \textbf{ĩwaipo} claro, brilhante.

\textbf{waipo} \textit{s.} -- broto de palmeira, broto de coqueiro.

\textbf{wairĩ} \textit{v.} -- segurar, virar, girar, torcer, rodar, apertar.

\textbf{wairowaipo za} \textit{s./v.} -- coroar.

\textbf{wairowaipo} \textit{s.} -- coroa.

\textbf{wairo} \textit{adj.} -- \textbf{ĩwairo} frouxo, bambo.

\textbf{wairo} \textit{s.} -- cocar.

\textbf{wairébé} \textit{v. pl.} = \textbf{dapusi} \textit{v. du.} -- sair.

\textbf{wairéré} \textit{v.} -- estender.

\textbf{wairé} \textit{v.} -- \textbf{wairébé} \textit{v. pl.} = \textbf{dapusi} \textit{v. du.} sair.

\textbf{wairõ} \textit{s.} -- \textbf{ĩwairõ} costas do animal.

\textbf{wajhizati} \textit{s} -- {costela}

\textbf{wajhi} \textit{v} -- {ser úmido}

\textbf{wajhu'u} \textit{aux} -- {epistêmico/factual (conhecer, poder)}

\textbf{wamhi} \textit{s.} -- .. \textbf{ĩsimnhihörö wamhi} -- corneta.

\textbf{wamhöri} \textit{v. du./pl.} -- lutar, combater.

\textbf{wamirĩ} \textit{v.} -- sacudir, coar, tirar, chacoalhar.

\textbf{wamnarĩzé} \textit{s.} -- corrupção.

\textbf{wamnarĩ} \textit{v.} -- arruinar, destruir.

\textbf{wamnhanare} \textit{s.} -- marimbondo.

\textbf{wamnharĩ} \textit{v.} -- enfeitar.

\textbf{wamo} \textit{s.} -- carícia.

\textbf{wamramizé} \textit{s.} -- dar nome a nós.

\textbf{wamrami} \textit{v.} -- ..

\textbf{wamreme} \textit{s.} = \textbf{rowamreme} -- barulho, rádio.

\textbf{wamri} \textit{s.} -- carícia.

\textbf{wamri} \textit{s.} -- nome próprio.

\textbf{wamri} \textit{v.} -- ser chamado.

\textbf{wamro'wa} \textit{s.} -- varredor.

\textbf{wamrozé} \textit{s.} -- vassoura.

\textbf{wamro} \textit{v.} -- varrer.

\textbf{wanarĩdobe} \textit{s.} -- canto de iniciação, dança.

\textbf{wana} \textit{posp.} -- \textbf{ĩwana} antes.

\textbf{wanha'u} \textit{s.} -- sucuri.

\textbf{wanhimizahöi'o} \textit{s.} -- camarão.

\textbf{wanhizari} \textit{v.} -- separar.

\textbf{wano'õsipo'o'õ} \textit{s.} -- \textbf{ĩwano'õsipo'o'õ} bomba atômica.

\textbf{wano'õsipo} \textit{s.} -- \textbf{ĩwano'õsipo} bomba.

\textbf{wano'õwabzuri} \textit{s.} -- bomba.

\textbf{wano'õ} \textit{s.} -- \textbf{ĩwano} tiro, estouro.

\textbf{wano'õ} \textit{v} -- afastar, expulsar, dispensar, estourar, disparar.

\textbf{wano} \textit{s.} -- \textbf{ĩwano} tiro, estouro.

\textbf{wano} \textit{v.} -- \textbf{ĩwano} crepitar (ruído do tiro); \textbf{ẽtẽwano} urânio.

\textbf{wapari'wa} \textit{s.} -- \textbf{ĩwapari'wa} ouvinte, ouvidor.

\textbf{wapari} \textit{v.} -- escutar, ouvir.

\textbf{wapre} \textit{v.} -- enrolar, prender, amarrar.

\textbf{waprosi} \textit{adj.} -- .. \textbf{siwaprosi} sozinho.

\textbf{waprosi} \textit{v.} -- ceder, explicar.

\textbf{wapru sisamro} \textit{s.} -- circulação de sangue.

\textbf{wapru siwamnarĩ} \textit{s.} -- intoxicação.

\textbf{wapru syry} \textit{s.} -- anemia.

\textbf{wapru'ub'rã} \textit{s.} -- coágulo.

\textbf{waprunhimzaihö} \textit{s.} -- sangria.

\textbf{waprunhorõdupu} \textit{s.} -- varizes.

\textbf{waprunhorõ} \textit{s.} -- veia, artéria.

\textbf{wapruwati} \textit{s.} -- pressão sanguínea.

\textbf{wapruza'wari} \textit{s.} -- derramamento de sangue de gente, chacina.

\textbf{wapru} \textit{s.} -- sangue.

\textbf{wapré} \textit{s.} -- ..; \textbf{zada'ré wapré} amigdalite.

\textbf{wapsari} \textit{v.} -- reclamar, resmungar, apelar.

\textbf{wapsisizé} \textit{s.} -- batida, contusão.

\textbf{wapsisi} \textit{v} -- \textbf{ĩwapsi} bater, socar.

\textbf{wapsi} \textit{conj.} -- só quando.

\textbf{wapsi} \textit{s.} -- baque.

\textbf{wapsi} \textit{v.} -- \textbf{ĩwapsi} bater.

\textbf{wapso} \textit{s.} -- ruído da água.

\textbf{wapsã pi'õ} \textit{s.} -- cadela.

\textbf{wapsã'uwa} \textit{s.} -- lobo.

\textbf{wapsã'u} \textit{s.} -- pulga.

\textbf{wapsãtoro} \textit{s.} -- besta, urso.

\textbf{wapsãwawẽ} \textit{s.} -- raposa (lit. cachorro grande).

\textbf{wapsã} \textit{s.} -- cão, cachorro.

\textbf{wapsõmrizé} \textit{s.} -- \textbf{ĩwapsõmrizé} chupeta.

\textbf{wapsõmri} \textit{v.} -- chupar, sorver.

\textbf{wapsẽrẽ} \textit{s.} -- sopa, mingau de milho.

\textbf{waptẽrẽ} \textit{v.} -- pedir.

\textbf{wapti'i} \textit{adj. c.} -- \textbf{wapti} molhado.

\textbf{wapti} \textit{adj.} -- molhado; \textbf{wapti'i di} é molhado.

\textbf{wapto} \textit{v.} -- \textbf{ãma ĩwapto} contribuir, contribuição, completar; \textbf{dawapto} que está junto de gente.

\textbf{wapto} \textit{v.} -- alegrar-se, acrescentar, provar, dar testemunho.

\textbf{waptu} \textit{s.} -- \textbf{ĩwaptu} mofo.

\textbf{waptu} \textit{s.} -- enfeite do "tébé".

\textbf{waptu} \textit{v.} -- apressar-se; \textbf{waptu na} depressa, rápido.

\textbf{waptã'ãzé} \textit{s.} -- nascimento, lugar de nascimento.

\textbf{waptã'ã} \textit{v. du./pl.} -- deixar cair, nascer, cair.

\textbf{waptãrã} \textit{v. sing.} -- nascer, cair.

\textbf{wapté} \textit{s.} -- jovem não iniciado à vida de adulto, menino.

\textbf{wapté} \textit{v. sing.} -- \textbf{waptẽrẽ} pedir.

\textbf{waptö'uwazé} \textit{s.} -- colchão.

\textbf{waptö'özé} \textit{s.} -- lençol.

\textbf{waptö'ö} \textit{v} -- \textbf{waptö} parar de chorar.

\textbf{waptözé} \textit{s.} -- lençol.

\textbf{waptö} \textit{s.} -- cama, esteira de dormir, lugar de dormir.

\textbf{waptö} \textit{v.} -- parar de chorar.

\textbf{wapu za'ru} \textit{s.} -- \textbf{ĩwapu za'ru} campo de jogo.

\textbf{wapure} \textit{adj. dim.} -- \textbf{ĩwapure} humilde, leve.

\textbf{wapure} \textit{s.} -- \textbf{ĩwapure} bolinha.

\textbf{wapuwabzuri'wa wa'õno} \textit{s.} -- \textbf{ĩwapuwabzuri'wa wa'õno} time de futebol.

\textbf{wapuwabzuri'wa} \textit{s.} -- \textbf{ĩwapuwabzuri'wa} jogador de futebol.

\textbf{wapuwaihu'u'wa} \textit{s.} -- técnico de futebol.

\textbf{wapu} \textit{adj.} -- \textbf{ĩwapu} leve; \textbf{ö wapu} água quente.

\textbf{wapu} \textit{s.} -- \textbf{ĩwapu} bola.

\textbf{wapu} \textit{} -- \textbf{ĩwapu na dato 'madö'ö'wa} presidente de futebol.

\textbf{wapu} \textit{} -- \textbf{ĩwapu na dato upari'wa} padrinho do jogo.

\textbf{wapõri} \textit{v.} -- soprar.

\textbf{wara'wa} \textit{s.} -- \textbf{iwara'wa} corredor (pessoa).

\textbf{warahuzé} \textit{} -- ..

\textbf{warahu} \textit{} -- ..

\textbf{warazu romnare} \textit{s.} -- caipira.

\textbf{warazunho'retizé} \textit{s.} -- estrangulamento do civilizado; Aldeia Cristo Rei.

\textbf{warazupotore} \textit{s.} -- boneca.

\textbf{warazu} \textit{s.} -- o branco, o "civilizado", o não-índio.

\textbf{warazu} \textit{s} -- {homem branco}

\textbf{wara} \textit{s.} -- = \textbf{dasisamro} corrida.

\textbf{wara} \textit{v. sing.} -- \textit{v. du.} correr.

\textbf{waripe} \textit{s.} -- gavião de fumaça.

\textbf{wari} \textit{v.} -- pedir.

\textbf{wari} \textit{v.} -- quebrar nozes, quebrar.

\textbf{wari} \textit{v} -- {ser elástico}

\textbf{warĩ'are} \textit{s.} -- cigarro.

\textbf{warĩ'a} \textit{s.} -- cigarro.

\textbf{warĩrĩzé} \textit{s.} -- sal.

\textbf{warĩrĩ} \textit{v.} -- salgar, condimentar, perfumar comida.

\textbf{warĩwedere} \textit{s.} -- cachimbo.

\textbf{warĩzeire} \textit{s.} -- chiclé.

\textbf{warĩ} \textit{s.} -- fumo, cigarro.

\textbf{warobo} \textit{s.} = \textbf{ĩhöiwarobo} papel; \textbf{dahöibarĩ warobo} filme.

\textbf{warob} \textit{s} -- {alimento}

\textbf{warurudupu} \textit{s.} -- amigdalite.

\textbf{waruru} \textit{s.} -- glândula.

\textbf{warã} \textit{s.} -- centro da aldeia, lugar de reunião.

\textbf{waréire} \textit{s.} -- arbusto com fruta.

\textbf{waré} \textit{adj.} -- \textbf{ĩwaré} fino e longo, solto, sem parar.

\textbf{waré} \textit{s} -- {mulher branca}

\textbf{warõno} \textit{s.} -- espinho.

\textbf{warõtõ'õ} \textit{v} -- \textbf{warõtõ} não tem capim na roça.

\textbf{warõ} \textit{adj.} -- \textbf{ĩwarõ} parado, calado, inanimado.

\textbf{warõ} \textit{s.} -- carrancudo.

\textbf{warõ} \textit{s.} -- comoção.

\textbf{wasi siwapto} \textit{s.} -- constelação.

\textbf{wasi ubumro} \textit{s.} -- constelação.

\textbf{wasi'usu} \textit{s.} -- noss.ompanheiro do mesmo grupo etário.

\textbf{wasihui'wa} \textit{s.} -- aquele que solta.

\textbf{wasihu} \textit{v.} -- soltar, desamarrar.

\textbf{wasihö} \textit{v} -- {brigar, lutar}

\textbf{wasisizé} \textit{s.} -- \textbf{ĩwasisizé} amarra, anel; \textbf{danhiptõmohi wasisizé} anel.

\textbf{wasisi} \textit{v} -- \textbf{wasi} meter, colocar, amarrar, amarrar em feixes.

\textbf{wasiwawẽ} \textit{s. aum.} -- estrela Vênus.

\textbf{wasi} \textit{s.} -- estrela.

\textbf{wasi} \textit{v.} -- amarrar.

\textbf{wasi} \textit{v.} -- mexer, misturar.

\textbf{wasu'u'wa} \textit{s.} -- sabedor, conhecedor, comentarista, acusador, crítico.

\textbf{wasu'uzé} \textit{s.} -- \textbf{ĩwasu'uzé} comentário, censura.

\textbf{wasu'u} \textit{s.} -- crítica, conto, comentário, censura; \textbf{rowasu'u} história.

\textbf{wasu'u} \textit{v.} -- contar, comentar, criticar, censurar.

\textbf{wasudu} \textit{adj.} -- \textbf{ĩwasudu} cansado.

\textbf{wasudu} \textit{s.} -- cansaço.

\textbf{wasudu} \textit{v.} -- cansar, lutar, insistir, pelejar, incomodar, chatear.

\textbf{wasutu} \textit{adj. c.} -- \textbf{ĩwasudu} cansado.

\textbf{wasutu} \textit{s.} -- cansaço.

\textbf{wasutu} \textit{v} -- cansar, lutar, pelejar, insistir.

\textbf{wasã ré} \textit{s./posp.} -- gravidez.

\textbf{wasãzé} \textit{s.} -- lugar do feto, útero.

\textbf{wasã} \textit{s.} -- feto, mulher grávida.

\textbf{wasã} \textit{v.} -- engravidar, conceber.

\textbf{wasédé za'ẽne} \textit{s.} -- \textbf{ĩwasété za'ẽne} falta enorme, crime.

\textbf{wasédé} \textit{adj.} -- \textbf{ĩwasédé} mau, faltoso.

\textbf{wasédé} \textit{s.} -- mal, pecado.

\textbf{wasété pari} \textit{s./v.} -- perdoar, clemência.

\textbf{wasété wa'öböze} \textit{s.} -- \textbf{ĩwasété wa'öböze} castigo.

\textbf{wasété wa'öbö} \textit{s./v.} -- \textbf{ĩwasété wa'öbö} castigar.

\textbf{wasété wẽ'õ} \textit{adj./adj.} -- excelente, bacana.

\textbf{wasété'rata} \textit{s.} -- pecado original.

\textbf{wasétézé} \textit{s.} -- maldade.

\textbf{watasu nhinarĩ} \textit{s./v.} -- arrancar a barba.

\textbf{watasu pari} \textit{s./v.} -- barbear.

\textbf{watasu} \textit{s.} -- barba.

\textbf{watasé} \textit{s.} -- pernilongo.

\textbf{watatö} \textit{s.} -- bochecha, sinal de raiva, de pirraça.

\textbf{watawasisi} \textit{vmp.} -- amarrar saco.

\textbf{watawãhö höiba'amo} \textit{s.omp.} -- caramujo.

\textbf{wataza'ru} \textit{s.} -- \textbf{ĩwataza'ru} hélice.

\textbf{watazé} \textit{s.} -- louvor, congratulação.

\textbf{wata} \textit{v} -- louvar, elogiar.

\textbf{wati pe tõmo zusizé} \textit{} -- brilho.

\textbf{wati'i} \textit{v} -- \textbf{ĩwati} exprimir, calcar, apretar, comprimir, pisar.

\textbf{watinhito} \textit{s.} -- cadeado.

\textbf{watipe} \textit{s.} -- plástico.

\textbf{watizu} \textit{s.} -- pó resultante de exprimir; \textbf{a'öiwatizu} incenso, resina.

\textbf{wati} \textit{v.} -- \textbf{ĩwati} exprimir, calcar, apertar, comprimir, pisar.

\textbf{watobrozé} \textit{s. sing.} = \textbf{dapusizé} \textit{s. du.} -- saída.

\textbf{watobro} \textit{v. sing.} = \textbf{dapusi} \textit{v. du.} -- sair.

\textbf{watu} \textit{s.} -- ..

\textbf{watözé} \textit{s.} -- dor de rins.

\textbf{watö} \textit{adj.} -- \textbf{ĩwatö} fuso.

\textbf{watö} \textit{s.} -- rim.

\textbf{wawa} \textit{s.} -- choro.

\textbf{wawa} \textit{s.} -- árvore.

\textbf{wawa} \textit{v.} -- chorar.

\textbf{wawinhi'umdatõ} \textit{num.} -- sete.

\textbf{wawire} \textit{s.} -- risco pequeno, vírgula.

\textbf{wawi} \textit{s.} -- risco.

\textbf{wawi} \textit{v.} -- riscar.

\textbf{wawã} \textit{v.} -- duvidar, não acreditar, rir-se, achar graça, desconfiar.

\textbf{wawã} \textit{} -- ..; \textbf{dahöröwawã} \textit{s.} eco.

\textbf{wawẽ} \textit{adj.} -- volumoso.

\textbf{wawẽ} \textit{s.} -- sogro, sogra.

\textbf{wazari} \textit{v.} -- misturar, ficar-lhe bem, caber.

\textbf{waza} \textit{s.} -- ponta.

\textbf{wazere} \textit{v.} -- soltar, tirar, livrar.

\textbf{waze} \textit{v. sing.} -- \textbf{wazere} soltar, livrar.

\textbf{wazidi} \textit{s.} -- gemido.

\textbf{wazidi} \textit{v. du./pl.} -- gemer.

\textbf{wazi} \textit{v. sing.} -- \textbf{ĩwazidi} gemer.

\textbf{wazuri'wa} \textit{s.} -- caçador de brancos.

\textbf{wazuri} \textit{v.} -- tirar, rasgar roupa, rasgar.

\textbf{wazu} \textit{v. sing.} -- \textbf{wazuri} tirar, rasgar, rasgar roupa.

\textbf{wazé} \textit{adj.} -- \textbf{ĩwazé} respeitoso, acanhado, retraído; \textbf{wazé di} é respeitoso.

\textbf{wazé} \textit{s.} -- respeito, vergonha.

\textbf{wazöri} \textit{v.} -- capinar.

\textbf{wa} \textit{conj.} -- porque, por causa de = \textbf{-te}; \textbf{wẽ'õ wa} porque não é bom.

\textbf{wa} \textit{conj.} = \textbf{wamhã} -- quando.

\textbf{wa} \textit{posp.} -- \textbf{ĩwa} em dentro de.

\textbf{wa} \textit{pron. pessoa} = \textbf{wa hã} -- eu; \textbf{wa norĩ hã} nós dois, nós todos.

\textbf{wa} \textit{pron. rel.} -- eu que, nós que; \textbf{wa hã, wa mo} eu vou.

\textbf{wa} \textit{s.} -- gordura.

\textbf{wa} -- {PRON} -- {eu}

\textbf{wede na'rada} \textit{s.} -- toco.

\textbf{wede sisa'ridi} \textit{s.} -- forquilha.

\textbf{wede wati'i} \textit{s.} -- bálsamo.

\textbf{wede'rã'uzé} \textit{s.} -- fruta cítrica, citros, limão.

\textbf{wede'rãtõ} \textit{s.} -- toco.

\textbf{wede'u} \textit{s.} -- espinho.

\textbf{wede'wa} \textit{s.} -- médico, enfermeiro.

\textbf{wede'õmo'wa} \textit{s.} -- espinho grande.

\textbf{wede'õmo} \textit{s.} -- \textbf{wede'u} espinho.

\textbf{wedehu} \textit{s.} -- vara.

\textbf{wedehö're} \textit{s.} -- caixa, caixote, gaveta.

\textbf{wedehöböpa} \textit{s.} -- banco de se sentar.

\textbf{wedehöbö} \textit{s.} -- \textbf{wedehöbö} mesa, tábua.

\textbf{wedehöbö} \textit{s.} -- mesa, tábua.

\textbf{wedemapu} \textit{s.} -- cortiça.

\textbf{wedenhamazazu} \textit{s.} -- café.

\textbf{wedenhamri} \textit{s.} -- árvore.

\textbf{wedenhi umo'wa} \textit{s.} -- carpinteiro.

\textbf{wedenhi umozé} \textit{s.} -- carpintaria.

\textbf{wedenhinhihã} \textit{s.} -- cedro.

\textbf{wedenhi} \textit{s.} -- remédio.

\textbf{wedenho nhipada} \textit{s.} -- coxo.

\textbf{wedenhorõto} \textit{s.} -- casca interna da árvore.

\textbf{wedenhorõ} \textit{s.} -- barbante, corda.

\textbf{wedepa nhipti} \textit{s.} -- ramo, galho.

\textbf{wedepaihi} \textit{s.} -- cruz.

\textbf{wedepapo} \textit{s.} -- copa de árvore.

\textbf{wedepare} \textit{s.} -- raiz de árvore.

\textbf{wedepa} \textit{s.} -- raiz.

\textbf{wedepo} \textit{s.} -- tábua.

\textbf{wedepro nho'u} \textit{s.} -- café líquido, café.

\textbf{wedepro} \textit{s.} -- carvão de lenha.

\textbf{wederobre} \textit{s.} -- horto.

\textbf{wedero} \textit{s.} -- bosque, parque, horto.

\textbf{wederã} \textit{s.} -- lança.

\textbf{wedesuire} \textit{s.} -- folha pequena de árvore.

\textbf{wedetede} \textit{s.} -- pau para festa; Aldeia São Domingos Sávio.

\textbf{wedewara} \textit{s.} -- carro de animal, veículo.

\textbf{wedewaré} \textit{s.} -- taquara, bastão, bengala.

\textbf{wedezadazöri} \textit{s.} -- caixa, baú, cofre.

\textbf{wedezaire} \textit{s.} -- carrinho.

\textbf{wedezapu'uzé ĩ'wa} \textit{s.} -- broca.

\textbf{wedezare} \textit{s.} -- mesinha.

\textbf{wedeza} \textit{s.} -- mesa.

\textbf{wedezu nhimizépu} \textit{s.} -- envenenamento.

\textbf{wedezu} \textit{s.} -- feitiço, pó de raiz de árvore para feitiço.

\textbf{wedezé wa'õtõre} \textit{s.} -- comprimido.

\textbf{wedezérére} \textit{s.} -- espinho de unha-de-gato.

\textbf{wedezé} \textit{s.} -- corretivo; \textbf{ti'aiwedezé} adubo químico, fertilizante.

\textbf{wedezé} \textit{s.} -- enfermaria, hospital, ambulatório.

\textbf{wede} \textit{s.} -- árvore, madeira, remédio, medicamento.

\textbf{wedeñorõ} \textit{s} -- corda.

\textbf{wesuirã nho'u} \textit{s.} -- chá.

\textbf{wesuirãze'um'rã} \textit{s.} -- couve.

\textbf{wesuirã} \textit{s.} -- folha, folha verde, verdura.

\textbf{wesunhipsada} \textit{s.} -- cansanção.

\textbf{wesu} \textit{s.} -- folha.

\textbf{we} \textit{adv.} -- para cá; \textbf{we aimorĩ} venha.

\textbf{we} -- {DIR} -- {movimento para cá (ventivo)}

\textbf{we} -- {PRON} -- {ele / ela}

\textbf{wẽ'ẽzé} \textit{} -- \textbf{ĩwẽ'ẽzé} quebrada.

\textbf{wẽ'ẽ} \textit{v. sing.} = \textbf{ẽ} \textit{v. du.} -- quebrar, partir; \textbf{ma ti'ẽ} ele quebrou.

\textbf{wẽzé} \textit{s.} -- \textbf{ĩwẽzé} bondade.

\textbf{wẽ} \textit{adj.} -- \textbf{ĩwẽ} bom, belo; \textbf{sada ĩwẽ} conveniente.

\textbf{wẽ} \textit{s.} -- bondade, clareza, beleza.

\textbf{wi'i} \textit{s.} -- perdiz.

\textbf{wisizé} \textit{s.} -- chegada, vinda.

\textbf{wisi} \textit{v.} -- chegar, vir.

\textbf{wi} \textit{posp.} -- \textbf{ĩwi} de, longe de.

\textbf{wi} \textit{posp} -- {no centro de}

\textbf{wi} \textit{s.} -- \textbf{ĩwi} caroço, semente de caroço, noz.

\textbf{wĩrĩ} \textit{v. sing.} = \textbf{pãrĩ} \textit{v. du.} -- matar; \textbf{ĩro wĩrĩ} apagar a luz.

\textbf{wã'rãpa} \textit{s.} -- lança.

\textbf{wã'rã} \textit{} --- ..; \textbf{wã'rãpa} lança.

\textbf{wãprã} \textit{s.} -- \textbf{ĩwãprã} menos gordo, miudeza de caça.

\textbf{wãrã'u} \textit{s.} -- tatu galinha.

\textbf{wãrãhudu} \textit{s.} -- tatu-bola.

\textbf{wãrãhöbö} \textit{s.} -- tatu.

\textbf{wãrãre} \textit{s.} -- tatu pequeno.

\textbf{wãrãwawẽ} \textit{s. aum.} -- tatu canastra.

\textbf{wãrã} \textit{s.} -- tatu = \textbf{wãrãhöbö}.

\textbf{wãtãwãhö} \textit{s.} -- concha.

\textbf{wãtãwã} \textit{s.} -- caracol, caramujo.

\textbf{wãtö} \textit{v} -- {ser torto}

\textbf{wã} \textit{s.} -- tronco do corpo, corpo.

\textbf{wã} \textit{v.} -- dividir, partir.

\textbf{wéré} \textit{s.} -- pássaro preto, anu preto.

\textbf{wéténhamri} \textit{s.} -- esteira.

\textbf{wétépara} \textit{s.} -- alça de baquité.

\textbf{wẽ'manharĩ'wa} \textit{s.} -- \textbf{ĩwẽ'manharĩ'wa} benfeitor.

\textbf{wẽ'wa} \textit{s.} -- \textbf{ĩwẽ'wa} = \textbf{ĩwẽ'manharĩ'wa} benfeitor.

\textbf{wẽ'õ'wa} \textit{s.} -- \textbf{ĩwe'õ'wa} malfeitor.

\textbf{wẽ'õ} \textit{adj. comp.} -- \textbf{ĩwẽ'õ} mau.

\textbf{wẽ'õ} \textit{vmp.} -- \textbf{ĩwẽ'õ} rejeitar, contrariar, desprezar.

\textbf{wẽpusi} \textit{s.} -- \textbf{ĩwẽpusi} vantagem.

\textbf{wẽre} \textit{adj. dim.} -- \textbf{ĩwẽre} bonito.

\textbf{wẽsãmri} \textit{s.} -- \textbf{ĩwẽsãmri} benevolência.

\textbf{wẽsãmri} \textit{v.} -- \textbf{ĩwẽsãmri} gostar, apreciar, ser benévolo.

\textbf{wẽsã} \textit{v. sing.} -- \textbf{ĩwẽsãmri} gostar, apreciar, ser benévolo.

\textbf{wẽtési} \textit{adv.} -- somente, apenas, exclusivamente; \textbf{ĩwẽtési} eu apenas.


%##########################################
\section*{Z}



\textbf{-zé} \textit{suf.} -- lugar onde se faz, cois.om que se faz.

\textbf{za'a} \textit{v} = \textbf{sa'a} -- ficar de pé, permanecer, elevar, suspender.

\textbf{za'idi s.} -- \textbf{ĩsa'idi} matinho no meio do campo, capoeira.

\textbf{za'idi} \textit{posp.} -- no meio de  \textbf{'rã za'idi} no meio da cabeça, fazer tonsura, cortar cabelo no meio da cabeça.

\textbf{za'ozé} \textit{s.} -- \textbf{ĩsa'õzé} lugar de estar pendurado, lugar de suspender.

\textbf{za'o} \textit{v. du.} -- -- suspender, pendurar, pender, suspender a tampa, elevar.

\textbf{za'rada} \textit{v.} -- enumerar, contar, dispor de gente, levantamento de gente.

\textbf{za'rata} \textit{v} -- enumerar, contar.

\textbf{za'ra} \textit{des.} -- pluralizadora.

\textbf{za'ra} \textit{s.} -- assombração, sombra.

\textbf{za'ra} \textit{v} -- {correr}.

\textbf{za'ra} -- {SFX} -- {plural distributivo (para nomes ou eventos)}

\textbf{za're'wa} \textit{s.} -- quem passa na frente de alguém.

\textbf{za'retö} \textit{s.} -- íngua.

\textbf{za're} \textit{s.} -- axila.

\textbf{za're} \textit{v.} -- \textbf{ĩsa're} \textbf{daza're} passar a frente de.

\textbf{za'ru} \textit{s.} -- pátria, terra nativa, lugar fixo, moradia, clarao, residência, lugar limitado, brinco.

\textbf{za'ry'ry} \textit{s.} -- cócega.

\textbf{za'rãsitédé} \textit{s.} -- caimbra.

\textbf{za'ré} \textit{v.} -- ceder, afastar, conceder, trair.

\textbf{za'rõtõ} \textit{v} -- pular, dançar com pulos, levantar, elevar.

\textbf{za'u'e} \textit{s.} -- jaburu, tuiuiu.

\textbf{za'u're} \textit{s.} -- jaburu.

\textbf{za'u'wa s.} -- último.

\textbf{za'uri'wa} \textit{s.} -- campeão, vencedor.

\textbf{za'uri} \textit{s.} -- sopro, bafo.

\textbf{za'uri} \textit{v.} -- soprar.

\textbf{za'u} \textit{posp.} -- = \textbf{ĩsa'u} atrás de, depois de.

\textbf{za'warizé} \textit{s.} -- cama, maca.

\textbf{za'wari} \textit{v. du.} -- deitar-se.

\textbf{za'wari} \textit{v.} -- \textbf{sa'wari} jogar fora, derramar, espalhar, despejar.

\textbf{za'é s.} -- espaço entre um e outro, garra.

\textbf{za'éré} \textit{s.} -- garra  \textbf{robduri za'éré} bicicleta.

\textbf{za'õmo} \textit{s.} -- cunhado.

\textbf{za'õno} \textit{v.} -- ficar guardado  \textbf{'rãza'õno} parte traseira da cabeça, recordar.

\textbf{za'ẽne} \textit{adj.} -- \textbf{ĩsa'ẽne} grande, bastante  \textbf{aibö za'ẽne} homem grande.

\textbf{zababa} \textit{posp.} -- \textbf{ĩtzababa} na beira de, ao lado de, perto de  \textbf{bödödi zababa} na beira do caminho.

\textbf{zabzé} \textit{s.} -- lugar de colocar algo, de ficar de pé.

\textbf{zabödö} \textit{posp.} -- \textbf{ĩsabödö} debaixo de, inferior, dependendo.

\textbf{zabödö} \textit{v.} -- fazer por gosto, acompanhar, seguir.

\textbf{zada'ro parizé} \textit{s.} -- "apagador do mau hálito", comida da manhã.

\textbf{zada'ro} \textit{s.} -- hálito.

\textbf{zada'rã} \textit{s.} -- pintura preta dos lábios.

\textbf{zada'ré wapré} \textit{s.} -- amigdalite.

\textbf{zada'ré} \textit{posp.} -- \textbf{ĩsadã'ré} ao lado de  \textbf{ö zada'ré} beira d'água.

\textbf{zada'ré} \textit{s.} -- limite, beira.

\textbf{zada'u} \textit{s.} -- saliva.

\textbf{zada'é v.} -- cumprir palavra.

\textbf{zada'õno} \textit{s.} -- coxa.

\textbf{zada'öbö} \textit{v} -- responder, retrucar, atender.

\textbf{zadab're} \textit{s. f.} -- desprezo, "covinha na face".

\textbf{zadai'ré} \textit{s.} -- boca (interior), garganta.

\textbf{zadaihu'u} \textit{v} -- conversar, compreender, combinar, entender-se.

\textbf{zadaihu} \textit{v.} -- compreender, conversar, combinar, entender-se.

\textbf{zadaipro} \textit{s.} -- saliva, escarro, cuspe.

\textbf{zadamrimi} \textit{v.} -- perguntar, pedir.

\textbf{zadapada} \textit{s.} -- bochecha.

\textbf{zadapata wairĩ} \textit{s./v.} -- torcer o rosto.

\textbf{zadapri'rine} \textit{s.} -- arbusto de que se tira fibras para fazer cordas para o pescoço.

\textbf{zadapri'ri} \textit{s.} -- . . . .

\textbf{zadapsy} \textit{s.} -- estalo da língua como sinal de aprovação ou admiração.

\textbf{zadarada} \textit{s.} -- lados debaixo dos braços, tórax.

\textbf{zadari} \textit{s.} -- grito, brado, clamor, berro, grito para chamar, choro de desespero.

\textbf{zadari} \textit{v.} -- gritar, bradar, clamar, berrar, gritar para chamar.

\textbf{zadawa situri} \textit{s./v.} -- conversar barulhentamente, vociferar.

\textbf{zadawa'a'a} \textit{s.} -- gritaria, clamor, aplauso.

\textbf{zadawa'a'a} \textit{v.} -- gritar, aplaudir, clamar, aclamar, cantar, louvar.

\textbf{zadawa'ahu'õ} \textit{adj.} -- "chato", incômodo.

\textbf{zadawahatu} \textit{v.} -- conversar barulhentamente.

\textbf{zadawanhipese} \textit{s.} -- mentira.

\textbf{zadawanhipe} \textit{s.} -- mentira.

\textbf{zadawanhipe} \textit{v.} -- mentir.

\textbf{zadawapara} \textit{s.} -- substituto, delegado, autoridade, em nome de.

\textbf{zadawaza'a} \textit{v.} -- bocejar.

\textbf{zadawa} \textit{s.} -- boca, palavra, abertura.

\textbf{zadaze} \textit{s.} -- cheiro.

\textbf{zadazute nihsu} \textit{s.} -- músculo superior da perna, coxa.

\textbf{zadazu} \textit{s.} -- músculo.

\textbf{zadazöri} \textit{s.} -- caixa, cofre, baú.

\textbf{zada} \textit{posp.} -- para = \textbf{da}, contra  \textbf{ĩsada} para algo.

\textbf{zada} \textit{s.} -- boca, queimadura.

\textbf{zada} \textit{v.} -- queimar, sapecar, cremar.

\textbf{zadutudu} \textit{s.} -- cabelo amarrado como rabo de galo.

\textbf{zadãmĩri} \textit{v.} -- cheirar.

\textbf{zadö} \textit{v.} -- cobrir, encobrir  \textbf{bazadö} \textit{s.} -- cobertura das costas.

\textbf{zahadu} -- espere! depois!.

\textbf{zahi na marĩ mrami} -- conquistar, raptar  conquista.

\textbf{zahi} \textit{adj.} -- corajoso, valente, selvagem.

\textbf{zahi} \textit{s.} -- coragem, valentia, bravura.

\textbf{zahi} \textit{v.} -- zangar-se, ficar valente, corajoso.

\textbf{zahui'wa s.} -- agressor, atacante.

\textbf{zahuré} \textit{des. du. c.} -- os dois, ambos.

\textbf{zahu} \textit{adv.} -- \textbf{ĩsahu} repetindo, outra vez.

\textbf{zahu} \textit{s.} -- agressão.

\textbf{zahu} \textit{v.} -- repetir, agredir, atacar.

\textbf{zahöbö} \textit{s.} -- buraco, cava, cova  \textbf{ẽtẽ zahöbö} caverna, gruta, cavação na pedra.

\textbf{zahöpö} \textit{posp.} -- cercado de  \textbf{ẽtẽ zahöpö} cercado de morros.

\textbf{zahöpö} \textit{posp.} -- debaixo de, \textbf{aibö zahöpö} debaixo do homem.

\textbf{zahöri} \textit{v. du./pl.} = \textbf{sahöri} -- parar.

\textbf{zaihö prézé} \textit{s.} -- baton.

\textbf{zaihö s.} -- lábio, beira  \textbf{ö zaihö} beira da água, rio, córrego.

\textbf{zaihö'rudu} \textit{s.} -- \textbf{ĩzaihö'rudu} decímetro.

\textbf{zaihösu} \textit{s.} -- bigode.

\textbf{zaihöwawire} \textit{s.} -- \textbf{ĩzaihöwawire} milímetro.

\textbf{zaihöwawi} \textit{s.} -- \textbf{ĩzaihöwawi} centímetro.

\textbf{zaihö} \textit{s.} -- \textbf{ĩzaihö} metro, pass.omprido.

\textbf{zaj-wawẽ} \textit{v} -- {ter coxa grande}.

\textbf{zamarĩ'wa} \textit{s.} -- seguidor, companheiro, discípulo, primeiro irmão.

\textbf{zamarĩ} \textit{v.} -- \textbf{ĩsihötö zamarĩ} copiar  cópia.

\textbf{zamarĩ} \textit{v.} = \textbf{samarĩ} -- seguir, acompanhar.

\textbf{zama} = \textbf{zéma} \textit{posp.} -- também, igualmente, com.

\textbf{zama} \textit{v. sing.} = \textbf{zamarĩ} -- seguir, acompanhar, imitar.

\textbf{zamo} \textit{s.} -- \textbf{ĩsamo} cauda de palha de buriti.

\textbf{zanhamri} \textit{v.} -- tecer  \textbf{abazi zanhamri} tecido, tecido de algodão.

\textbf{zani} \textit{s.} -- cativeiro.

\textbf{zani} \textit{v.} = \textbf{sani} -- tirar, retirar, salvar, afugentar, afastar, livrar.

\textbf{zanozé} \textit{s.} -- cinza quente.

\textbf{zano} \textit{v.} -- . . . .

\textbf{zapa'a} \textit{v.} -- interessar-se, participar, respeitar, cuidar de alguém.

\textbf{zapada} \textit{v.} -- ficar sentado de perna cruzada.

\textbf{zapari'wa} \textit{s.} -- guarda, vigia, incentivador.

\textbf{zaparizé} \textit{s.} -- apoio, incentivo, lugar de espera.

\textbf{zapari} \textit{v.} -- apoiar, vigiar, esperar, aguardar, incentivar.

\textbf{zapodo} \textit{adj.} -- \textbf{ĩsapodo} redondo.

\textbf{zapotore} \textit{s.} -- \textbf{ĩsapotore} redondo pequeno, moeda.

\textbf{zapo} \textit{adj.} -- \textbf{ĩsapo} curto, pequeno.

\textbf{zapru} \textit{s.} = \textbf{robzapru} poeira.

\textbf{zapré} \textit{s.} -- pintura vermelha.

\textbf{zapu} \textit{s.} -- \textbf{ĩsapu} furo, buraco  \textbf{dapo're zapu} -- furo da orelha.

\textbf{zapõrĩ} \textit{v.} -- soprar, aspirar, tirar doença.

\textbf{zara} \textit{s.} -- córrego, água, rio  \textbf{zara niwi} do outro lado do córrego  \textbf{zara u} para o outro lado do rio.

\textbf{zaribi} \textit{s.} -- penas de asas  \textbf{si zaribi} penacho.

\textbf{zarina} \textit{posp.} -- atrás, em seguimento, de acordo com.

\textbf{zari} \textit{v.} -- dar de graça, oferecer ajuda com comida.

\textbf{zaro} \textit{v.} -- assar  \textbf{romnhi zaro} -- currasco.

\textbf{zaro} \textit{v.} -- levantar, erguer  \textbf{dazadawa zaro} -- desobedecer, protestar.

\textbf{zarõno} \textit{v.} -- pular, dançar com pulos, levantar, elevar  \textbf{'rãzarõno} levantar a cabeça, soberba.

\textbf{zasi ãna} \textit{adj./posp.} -- generoso, submisso, sem orgulho.

\textbf{zasi'õ} \textit{s.} -- generosidade.

\textbf{zasizé} \textit{s.} -- entrada.

\textbf{zasi} \textit{adj.} -- orgulhoso, amargo.

\textbf{zasi} \textit{s.} -- \textbf{ĩsasi} ninho.

\textbf{zasi} \textit{s.} -- amargura.

\textbf{zasi} \textit{v. du.} -- entrar.

\textbf{zasi} \textit{v.} -- aninhar, fazer ninho.

\textbf{zasu} \textit{s.} -- pelos, penas.

\textbf{zasu} \textit{v.} -- defecar, cagar.

\textbf{zasõmri} \textit{v.} -- pendurar, colocar, suspender.

\textbf{zata} \textit{v} -- \textbf{ĩsada} queimar, assar, sapecar.

\textbf{zate} \textit{v.} -- expulsar, afastar.

\textbf{zati} \textit{posp.} -- \textbf{ĩsati} no fim de.

\textbf{zati} \textit{s.} -- limite, fim, confim.

\textbf{zati} \textit{v.} -- \textbf{ĩsati} confinar.

\textbf{zatõrĩ} \textit{v.} -- mandar embora, dispensar  \textbf{dazatõrĩ we} mandar para cá, mandar de volta, chamar.

\textbf{zawi'wa} \textit{s.} -- amigo, camarada.

\textbf{zawipisudu} \textit{s.} -- aliança.

\textbf{zawire} \textit{v. dim.} -- carinho.

\textbf{zawizé} \textit{s.} -- amor, caridade, amizade.

\textbf{zawi} \textit{v.} -- amar, gostar  \textbf{marĩ zawi} -- conter  \textbf{da wi zawi} recusar, proibir, negar.

\textbf{zawẽrẽ} \textit{v.} -- sonhar, olhar sonhando, ter visão.

\textbf{zazahö marĩ nhitobzé} \textit{s./pron./s.} -- pano que fecha alguma coisa, cortina.

\textbf{zazahö nhtisiwi} \textit{s./posp.} -- por cima da roupa, casaco.

\textbf{zazahö wa'õno} \textit{s.} -- pano cortado, manto para cobrir criança (sem manga), manta.

\textbf{zazahö} \textit{s.} -- roupa, pano, tecido.

\textbf{zaza} \textit{s.} -- que se veste  \textbf{zazahö} roupa  \textbf{da'razaza} chapéu.

\textbf{zazdei'wa} \textit{s.} -- crente, fiel, confiante.

\textbf{zazezé} \textit{s.} -- consentimento, fé.

\textbf{zaze} \textit{v.} -- crer, acreditar, confiar, concordar.

\textbf{zazu} \textit{adj.} -- cinzento  \textbf{ẽtẽza} \textbf{zu pire} pedra cinzenta pesada, chumbo.

\textbf{zazéb 'madö'ö'wa} \textit{s.} -- \textbf{zazéb 'madö'ö'wa} carcereiro.

\textbf{zazé} \textit{s.} -- cadeia, prisão, lugar de pôr gente.

\textbf{zazöri} \textit{v. sing.} \textbf{sahöri} \textit{v. du./pl.} -- parar, interromper, terminar.

\textbf{za} \textit{des. fut.} -- "vai"  \textbf{õ hã}, \textbf{te za mo} ele irá.

\textbf{za} \textit{part} -- {aspecto prospectivo}.

\textbf{za} \textit{pre excl.} -- de espanto, excl. de surpresa.

\textbf{za} \textit{s.} -- coxa.

\textbf{za} \textit{v. du.} -- \textbf{ĩza} colocar, meter  \textbf{zai wa'aba} colocai, metei.

\textbf{ze-te bö} \textit{v.} -- ceder, se assim gostar.

\textbf{zeru ubumroi'wa} \textit{s.} -- cobrador.

\textbf{zeru ubumro} \textit{s.} -- reunião de dinheiro, cobrança, cobrar.

\textbf{zeru waptẽrẽ} \textit{s./v.} -- pedir, exigir dinheiro, cobrança, cobrar.

\textbf{zeru za'razé} \textit{s.} banco (\$).

\textbf{zerupru} \textit{s.} -- dinheiro quebrado, centavo.

\textbf{zerusiromo'õ} \textit{s.} -- cheque.

\textbf{zeru} \textit{s. neo.} -- dinheiro.

\textbf{zidi} \textit{s.} -- brilhar de gente de longe.

\textbf{zidi} \textit{v.} -- \textbf{sidi} fiar.

\textbf{zidi} \textit{v.} -- fiar, brilhar  \textbf{tã we zidi} a chuva ameaça (=brilha).

\textbf{zu'rã'ré} \textit{adj.} -- seco.

\textbf{zu'rã'ré} \textit{v.} -- secar.

\textbf{zu'u'é} \textit{s.} -- traíra.

\textbf{zub'a} \textit{s.} -- \textbf{ĩzub'a} enfeite cerimonial.

\textbf{zubre} \textit{v.} -- \textbf{subre} socar.

\textbf{zubru ahö} \textit{s. adj.} -- supuração.

\textbf{zubru} \textit{s.} -- pus.

\textbf{zubzari} \textit{v. sing.} -- carregar.

\textbf{zubzé} \textit{s.} -- coisa para socar  \textbf{ti'a zubzé} grade.

\textbf{zupe} \textit{s.} -- pó fino.

\textbf{zupru} \textit{s.} -- pus, abcesso.

\textbf{zurizé} \textit{s.} -- lugar de sair, de nascer, de aparecer, de brotar  \textbf{ĩsihötö zurizé} gráfica.

\textbf{zuri} \textit{v.} -- brotar, germinar, distribuir em cada aldeia  \textbf{robzuri} plantar  semear.

\textbf{zusi'wa} \textit{s.} -- quem toca o chão (rito).

\textbf{zusi} \textit{adj. c.} -- \textbf{zu} muito, aglomerado  \textbf{ĩreme zusi} falar muitos.

\textbf{zutu'rã} \textit{s.} -- glande.

\textbf{zuwẽ} \textit{s.} -- canto dos jovens.

\textbf{zu} \textit{ s.} -- -- barulho de trabalho.

\textbf{zu} \textit{ s.} -- \textbf{ĩsu} pó, poeira.

\textbf{zu} \textit{adj.} -- sujo  \textbf{ö zu} água suja, água turva.

\textbf{zu} \textit{v.} = \textbf{da'ãma ĩsiwãrĩ} -- apoiar.

\textbf{zyzy} \textit{s.} -- gafanhoto.

\textbf{zé'a'u'ẽne} \textit{s.} -- tijolo.

\textbf{zé'are} \textit{s.} -- giz.

\textbf{zé'arã zupe} \textit{s.} -- pó fino de barro branco, cal.

\textbf{zé'arã'ubumro} \textit{s.} -- reunião de barro branco, calcário.

\textbf{zé'a} \textit{s.} -- barro, argila.

\textbf{zé're} \textit{s.} -- cavidade para urina da pessoa, bexiga.

\textbf{zé'ubumro} \textit{s.} -- bexiga.

\textbf{zébrézé} \textit{s. sing.} -- = \textbf{dazasizé} \textit{s. du.} -- entrada.

\textbf{zébré} \textit{v. sing.} -- = \textbf{dazasi} \textit{v. du.} -- entrar.

\textbf{zéirépa} \textit{s.} -- muito barro branco  Aldeia São Pedro.

\textbf{zéma} \textit{posp.} -- \textbf{ĩzéma} \textbf{dazéma} também, igualmente, com.

\textbf{zépada} \textit{s.} -- sofrer, comover-se.

\textbf{zépata'wa} \textit{s.} -- sofredor.

\textbf{zépatazé} \textit{s.} -- sofrimento, lugar de sofrimento.

\textbf{zéprabare} \textit{s.} -- pequena corrida de urina, uretra.

\textbf{zéptö'özé} \textit{s.} -- cura, convalescência, lugar ou coisa que cura.

\textbf{zéptö'ö} \textit{v du. pl.} -- -- curar, sarar, convalescer.

\textbf{zéptörö} \textit{s. sing.} -- curar, sarar, convalescer.

\textbf{zéptö} \textit{s.} -- convalescência, cura.

\textbf{zépu ubumrozé} \textit{s.} -- lugar de reunião de pessoas doentes.

\textbf{zépu'u nhomri} \textit{s./v.} -- dar doença para gente, contaminar.

\textbf{zépu} \textit{s.} -- doença.

\textbf{zéré 'rã'ru} \textit{s.} -- cabelo enrolado, pixaim.

\textbf{zéré} \textit{s.} -- cabelo.

\textbf{zétédé} \textit{s.} -- uremia.

\textbf{zé} \textit{s.} -- dor  \textbf{po'rezé} dor de ouvido.

\textbf{zé} \textit{s.} -- urina.

\textbf{zõ} \textit{posp.} -- para, por, em busca de.

\textbf{zö 'rasi'wa} \textit{s.} -- dono do ruído do chocalho, batedor de chocalho, dono do chocalho.

\textbf{zö 'rãsi'wa 'rada} \textit{s.} -- velho batedor de chocalho.

\textbf{zö'rẽne} \textit{v. sing.} = \textbf{hö'rẽne} = \textbf{hösi} \textit{v. du.} -- beber.

\textbf{zö'uhu} \textit{ s.} -- \textbf{ĩzö'uhu} -- abelha.

\textbf{zöbzu} \textit{adj.} -- carrancudo, triste.

\textbf{zöhurure} \textit{s.} -- cotia.

\textbf{zöhuru} \textit{s.} -- cotia.

\textbf{zömhu'atoire} \textit{s.} -- lagarta.

\textbf{zömhu'rãre} \textit{s.} -- formiguinha preta.

\textbf{zömhu'rã} \textit{s.} -- correição (formiga preta).

\textbf{zömhupré} \textit{s.} -- formiga vermelha.

\textbf{zömhuzé} \textit{s.} -- ferrão de formiga e outros.

\textbf{zömhu} \textit{s.} -- formiga (e outro) que tem ferrão.

\textbf{zömorĩ} \textit{v.} -- caçar com paradas demoradas.açar com família por vários dias.

\textbf{zöp-tete} \textit{v} -- {endurecer (de grão)}.

\textbf{zöpronho'u} \textit{s.} -- caldo de pó de semente, café líquido.

\textbf{zöpro} \textit{s.} -- pó de semente, café em pó.

\textbf{zöribi} \textit{s} -- natação = \textbf{daribi}.

\textbf{zösi} \textit{v. du.} -- beber = \textbf{hösi}.

\textbf{zö} \textit{s.} -- \textbf{ĩzö} grão, semente de plantas.

\textbf{zö} \textit{s.} -- chocalho, maracá.


